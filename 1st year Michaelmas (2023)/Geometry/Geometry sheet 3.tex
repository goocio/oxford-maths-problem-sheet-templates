\documentclass[answers]{exam}
\usepackage{../MT2023}
\usepackage{xurl}
\usepackage{hyperref}
\hypersetup{colorlinks=true,urlcolor=blue,}
\urlstyle{same}

\title{Geometry -- Sheet 3\\Conics}
\author{YOUR NAME HERE :)}
\date{Michaelmas Term 2023}


\begin{document}
\maketitle
\begin{questions}

\question%1
Noting that the point $(x_{0}, y_{0})$ is at a distance $|a x_{0}+b y_{0}+c| / \sqrt{a^{2}+b^{2}}$ from the line $a x+b y+c=0$, find the equations in Cartesian co-ordinates of:
\begin{parts}
\part%1a
the ellipse with foci at $(\pm 2,0)$ which passes through $(0,1)$;

\part%1b
the hyperbola with asymptotes $y=\pm 2 x$ and directrices $x=\pm 1$;

\part%1c
the ellipse consisting of all points $P$ such that $|A P|+|B P|=10$, where $A$ is $(3,0)$ and $B$ is $(-3,0)$;

\part%1d
the parabola with directrix $x+y=1$ and focus $(-1,-1)$.
\end{parts}



\question%2
Consider the parabola $C$ with equation $y^{2}=4 a x$ and let point $P=(a t^{2}, 2 a t)$.
\begin{parts}
\part%2a
What is the gradient of $C$ at $P$ in terms of $t$?

\part%2b
The focus $F$ is at $(a, 0)$. Let $l_{1}$ denote the line connecting $P$ and $F$, and $\theta_{1}$ denote the angle between $l_{1}$ and the tangent to $C$ at $P$. Show that \[
	\cos \theta_{1}=\frac{t}{\sqrt{t^{2}+1}}.
\]

\part%2c
Let $l_{2}$ denote the horizontal line through $P$ and $\theta_{2}$ denote the angle between $l_{2}$ and the tangent to $C$ at $P$. Show that $\theta_{1}=\theta_{2}$. (Consequently any light beam emitted from the focus $F$ will reflect horizontally after contact with a parabolic mirror.)
\end{parts}



\question%3
\begin{parts}
\part%3a
The conic $C$ is formed by intersecting the double cone $x^{2}+y^{2}=z^{2}$ with the plane $x+y+z=1$. Show that the point with position vector \[
	\mathbf r(t)=(1+(\sec t-\tan t)/\sqrt2, 1+(\sec t+\tan t)/\sqrt2, -1-\sqrt2\sec t)
\] lies on $C$.

\part%3b
Show that the vectors $\mathbf{e}_{1}=(1 / \sqrt{6}, 1 / \sqrt{6},-\sqrt{2 / 3})$ and $\mathbf{e}_{2}=(-1 / \sqrt{2}, 1 / \sqrt{2}, 0)$ are of unit length, are perpendicular to one another and are parallel to the plane $x+y+z=1$. Show further that \[
	\mathbf{r}(t)=(1,1,-1)+(a \sec t) \mathbf{e}_{1}+(b \tan t) \mathbf{e}_{2}
\] where $a$ and $b$ are positive numbers to be determined.

\part%3c
Show that $C$ has eccentricity $2 / \sqrt{3}$, has foci $(1,1,-1) \pm 2 \mathbf{e}_{1}$ and that the directrices, in parametric form, are $(1,1,-1) \pm 3 \mathbf{e}_{1} / 2+\lambda \mathbf{e}_{2}$.
\end{parts}



\question%4
By rotating the $x y$-axes appropriately, and subsequently completing the squares, show that the curve \[
	6 x^{2}+4 x y+9 y^{2}-12 x-4 y-4=0
\] is an ellipse and find its area.



\question%5
Let $\mathbf{v}$ and $\mathbf{w}$ be independent vectors in $\mathbb{R}^{2}$. Show that \[
	\mathbf{r}(t)=\mathbf{v} \cos t+\mathbf{w} \sin t
\] where $0 \leqslant t<2 \pi$, is a parametrization of an ellipse.



\question%6
(Optional) Let $P$ be a point on an ellipse with foci $F_{1}$ and $F_{2}$. Show that the two lines $F_{1} P$ and $F_{2} P$ make the same angle with the normal to the ellipse. [\emph{This means that if you are playing billiards on an elliptical table, with a pocket on one focus and the ball placed on the other focus, every shot will go in! (Assuming it is hit sufficiently hard.) \href{https://www.youtube.com/watch?v=4KHCuXN2F3I}{Click here for a cool YouTube video showing this.}}] Consider: is there a related statement for a hyperbola?

\end{questions}

\end{document}
