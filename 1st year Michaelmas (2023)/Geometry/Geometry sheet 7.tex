\documentclass[answers]{exam}
\usepackage{../MT2023}

\title{Geometry -- Sheet 7\\Surface area, Isometry}
\author{YOUR NAME HERE :)}
\date{Michaelmas Term 2023}


\begin{document}
\maketitle
\begin{questions}

\question%1
Let $\mathbf{r}$ be a parametrization of a subset of the $x y$-plane in $\mathbb{R}^{3}$ so that \[
	\mathbf{r}(u, v)=(x(u, v), y(u, v), 0), \qquad(u, v) \in U
\] Determine $\mathbf{r}_{u} \wedge \mathbf{r}_{v}$ and show that the area of $\mathbf{r}(U)$ equals \[
	\iint\limits_u\left|\frac{\partial(x, y)}{\partial(u, v)}\right|~\mathrm du~\mathrm dv.
\]



\question%2
Part of a catenoid is formed by rotating the graph of $y=\cosh x$ (where $a \leqslant x \leqslant b$) about the $x$-axis. Calculate the area of the resulting surface.



\question%3
Let $0<\alpha<\pi / 2$ and let $S_{\alpha}$ be the cap of the unit sphere $x^{2}+y^{2}+z^{2}=1$ parametrized as \[
	\mathbf r(\theta, \phi)=(\sin\theta\cos\phi, \sin\theta\sin\phi, \cos\theta)\qquad
	0<\theta<\alpha,\quad 0<\phi<2\pi.
\]
\begin{parts}
\part%3a
Determine $\mathbf{r}_{\theta} \wedge \mathbf{r}_{\phi}$ and show that the surface area of $S_{\alpha}$ equals $2 \pi(1-\cos \alpha)$.

\part%3b
Rederive the result of (a) by considering the cap as part of the graph $z=\sqrt{1-x^{2}-y^{2}}$ and using the formula for a graph's surface area.
\end{parts}



\question%4
Let $\theta$ and $\phi$ denote spherical polar co-ordinates on the unit sphere $x^{2}+y^{2}+z^{2}=1$. The \emph{Albers equal-area conic projection} is defined by \[
	x=\frac1n\sqrt{C-2n\cos\theta}\sin n\phi,\qquad
	y=\rho_0-\frac1n\sqrt{C-2n\cos\theta}\cos n\phi
\] where $n, C, \rho_{0}$ are constants. Describe the image in the plane of a latitude ($\theta=$ const.) and a meridian ($\phi=$ const.) Determine the Jacobian $|\partial(x, y) / \partial(\phi, \theta)|$ and explain why this means the projection preserves area.



\question%5
\begin{parts}
\part%5a search up the wiki page on "first fundamental form"
Let $\mathbf{r}(u, v)$ be a parametrization of a surface in $\mathbb{R}^{3}$ and $\gamma(t)=\mathbf{r}(u(t), v(t))$. Show the following: \[
	\boldsymbol\gamma'(t)=u'\mathbf r_u+v'\mathbf r_v,\qquad
	|\boldsymbol\gamma'(t)|^2=E(u')^2+2Fu'v'+G(v')^2,\qquad
	|\mathbf{r}_{u} \wedge \mathbf{r}_{v}|^2=EG-F^2
\] where $E=\mathbf{r}_{u} \cdot \mathbf{r}_{u}$, $F=\mathbf{r}_{u} \cdot \mathbf{r}_{v}$, $G=\mathbf{r}_{v} \cdot \mathbf{r}_{v}$.

\part%5b
The flat torus in $\mathbb{R}^{4}$ can be parametrized as \[
	\mathbf r(\theta, \phi)=(\cos \theta, \sin \theta, \cos \phi, \sin \phi),\qquad
	(\theta, \phi) \in U=(0,2 \pi) \times(0,2 \pi).
\] Let $\gamma(t)=(\theta(t), \phi(t))$ be a curve in $U$.
\begin{subparts}
\subpart%5bi
Show that the curve $\gamma$, and its image $\mathbf{r}(\gamma)$ in the flat torus, have the same length.

\subpart%5bii
As there is no vector product in $\mathbb{R}^{4}$ we cannot use our current formulae to work out the surface area of the flat torus. What do you think its surface area equals?
\end{subparts}
\end{parts}



\question%6
\begin{parts}
\part%6a
Compute the area of the helicoid \[
	\mathrm{s}(X, Z)=(X \cos Z, X \sin Z, Z)
\] for $A<X<B, 0<Z<C<2 \pi$.

\part%6b
Rederive the result by defining an isometry between the catenoid and helicoid and using the calculation in Question 2. You might find it helpful to use the identities: $\sinh (2 x)=2 \sinh x \cosh x, \cosh (\arsinh x)=\sqrt{1+x^{2}}$.
\end{parts}



\question%7
(Optional) Let $0<a \leqslant b \leqslant c<\pi / 2$ and let $A=(0,0,1)$ and $B=(\sin c, 0, \cos c)$ be points on the unit sphere.
\begin{parts}
\part%7a
Show that there is a point $C$ which is at a distance $b$ from $A$ and at a distance $a$ from $B$ if and only if $a+b \geqslant c$. [Hint: every point at a distance $b$ from $A$ has the form $(\sin b \cos \phi, \sin b \sin \phi, \cos b)$ for some $\phi$.]

\part%7b
\begin{subparts}
\subpart%7bi
Prove the spherical cosine rule which states that, for a spherical triangle with sides $a, b, c$ and opposite angles $\alpha, \beta, \gamma$, \[
	\cos a=\cos b \cos c+\sin b \sin c \cos \alpha.
\]

\subpart%7bii
Show that when $a, b, c$ are small enough, that we can ignore terms of order 3 and above, so that the approximations $\cos x \approx 1-x^{2} / 2$ and $\sin x \approx x$ apply, then the spherical cosine rule approximates the usual cosine rule.
\end{subparts}
\end{parts}

\end{questions}

\end{document}
