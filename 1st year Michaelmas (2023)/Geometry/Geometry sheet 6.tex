\documentclass[answers]{exam}
\usepackage{../MT2023}

\title{Geometry -- Sheet 6\\Surfaces, Arc length}
\author{YOUR NAME HERE :)}
\date{Michaelmas Term 2023}


\begin{document}
\maketitle
\begin{questions}

\question%1
Consider the hyperboloid of one sheet with equation $x^{2}+y^{2}=z^{2}+1$.
\begin{parts}
\part%1a
Show that this hyperboloid can be parametrized as \[
	\mathbf{r}(\theta, \lambda)=(\cos \theta, \sin \theta, 0)+\lambda(\sin \theta,-\cos \theta, 1) \quad 0 \leqslant \theta<2 \pi, \lambda \in \mathbb{R} .
\]

\part%1b
Determine a normal vector at $\mathbf{r}(\theta, \lambda)$ by working out $\mathbf{r}_{\theta} \wedge \mathbf{r}_{\lambda}$.

\part%1c
Determine a normal vector at $\mathbf{r}(\theta, \lambda)$ by working the gradient vector of $f(x, y, z)=x^{2}+y^{2}-z^{2}$ at $\mathbf{r}(\theta, \lambda)$.

\part%1d
What is the tangent plane to the hyperboloid at $\mathbf{r}(\theta, \lambda)$?
\end{parts}



\question%2
\begin{parts}
\part%2a
What is the shortest distance between $(0,1 / \sqrt{2}, 1 / \sqrt{2})$ and $(1 / 2,0, \sqrt{3} / 2)$ as measured on the unit sphere $x^{2}+y^{2}+z^{2}=1$?

\part%2b
Let $\gamma(t)=(\cos \theta(t), \sin \theta(t), z(t))$, where $a \leqslant t \leqslant b$, be a parametrized curve in the cylinder $x^2+y^2=1$.
\begin{subparts}
\subpart%2bi
Show that the curve $\gamma$ has arc length \[
	\int_{a}^{b} \sqrt{\dot{\theta}^{2}+\dot{z}^{2}} \mathrm{~d} t
\]

\subpart%2bii
Deduce that the map $(\theta, z) \to(\cos \theta, \sin \theta, z)$ from $\mathbb{R}^{2}$ to the cylinder is an isometry (when distances are measured on the cylinder).

\subpart%2biii
Find the shortest distance from $(1,0,0)$ to $(0,1,1)$ when measured on the cylinder.
\end{subparts}
\end{parts}



\question%3
Consider the following parameterisation of the cone $x^{2}+y^{2}=z^{2}$: \[
	\mathbf{r}(h, \theta)=(h \cos \theta, h \sin \theta, h) .
\] Does this parameterisation represent an isometry? That is, does a curve in the $(h,\theta)$-plane have the same length when mapped to the cone?



\question%4
\begin{parts}
\part%4a
Consider the following construction of a torus: a circle of radius $a$ is situated in the $x-z$ plane with centre at the point $x=b, z=0$, where $b>a$; and this circle is then rotated about the $z$ axis. Obtain the following parameterisation: \[
	\mathbf r(\theta,\phi)=((b+a\cos\theta)\cos\phi,(b+a\cos\theta)\sin\phi,a\sin\theta),
\] and describe how the values of $\theta$ and $\phi$ relate to position on the torus.

\part%4b
The planar curve $(x(s), y(s)), s \in(0, l)$, for which $x(s)>0$, is rotated around the $y$-axis. Give a parameterisation for the resulting surface.

\part%4c
Bonus: if the curve in (b) were rotated around, say, the line $y=m x$ for some $m>0$, how might you go about constructing a parameterisation? (You needn't do this in detail, just outline the steps.)
\end{parts}



\question%5
Consider two points on a torus with equal values of the angle $\theta$ (as defined in Q4(a)) and different values of $\phi$. Give an arc length parameterisation for the path connecting the points where $\theta$ remains fixed. Show that this is not a geodesic, except at two particular values of $\theta$, which you should determine.



\question%6
\begin{parts}
\part%6a
Let $\mathbf{r}(u, v)$ be a parametrized surface with unit normal $\mathbf{n}(u, v)$. By differentiating $\mathbf{n} \cdot \mathbf{n}=1$, show that $\mathbf{n}_{u}$ and $\mathbf{n}_{v}$ are tangent vectors to the surface. Deduce that \[
	\mathbf{n}_{u} \wedge \mathbf{n}_{v}=K(u, v) \mathbf{r}_{u} \wedge \mathbf{r}_{v}
\] for some scalar function $K(u, v)$, known as the Gaussian curvature.

\part%6b
When $\mathbf{r}(u, v)$ is the sphere of radius $R$ centred at $\mathbf{0}$, so that $\mathbf{r}=R \mathbf{n}$, what is $K(u, v)$?

\part%6c
Let $0<a<b$. Consider the torus parameterisation from Q4(a). Determine $\mathbf{n}(\theta, \phi)=\mathbf{r}_{\theta} \wedge \mathbf{r}_{\phi} /|\mathbf{r}_{\theta} \wedge \mathbf{r}_{\phi}|$ and hence find $K(\theta, \phi)$. Where is $K(\theta, \phi)$ positive, where negative? Can you think of a surface for which $K$ would be everywhere zero?
\end{parts}



\question%7
(Optional) Let $S$ denote the unit sphere $x^{2}+y^{2}+z^{2}=1$. Every point $P=(r \cos \alpha, r \sin \alpha)$ in $\mathbb{R}^{2}$ can be identified with a point $Q$ on $S$ by drawing a line from $(r \cos \alpha, r \sin \alpha, 0)$ to the sphere's north pole $N=(0,0,1)$; this line intersects the sphere at two points $Q$ and $N$. We define a map $f$ from $\mathbb{R}^{2}$ to the sphere $S$ by setting $f(P)=Q$.
\begin{parts}
\part%7a
Show that \[
	Q=\left(\frac{2r\cos\alpha}{1+r^2},\frac{2r\sin\alpha}{1+r^2},\frac{r^2-1}{1+r^2}\right).
\]

\part%7b
What are the spherical polar co-ordinates $\theta, \phi$ of the point $Q$?
\end{parts}

\end{questions}

\end{document}
