\documentclass[answers]{exam}
\usepackage{../MT2023}

\title{Introduction to Complex Numbers}
\author{YOUR NAME HERE :)}
\date{Michaelmas Term 2023}


\begin{document}
\maketitle
\begin{questions}

\question%1
Which of the following quadratic equations require the use of complex numbers to solve them? \[
	3 x^{2}+2 x-1=0 ; \qquad 2 x^{2}-6 x+9=0 ; \qquad-4 x^{2}+7 x-9=0.
\]



\question%2
Show directly that if $z w=0$, where $z, w$ are two complex numbers, then $z=0$ or $w=0$ (or both). Now re-prove this making use of properties of the modulus function.



\question%3
Put each of the following numbers into the form $a+b i$. \[
	(3+2 i)+(2-7 i), \qquad(1+2 i)(3-i), \qquad(1+2 i) /(3-i), \qquad(1+i)^{4}.
\]



\question%4
Find the modulus and argument of each of the following numbers. \[
	1+\sqrt{3} i, \quad(2+i)(3-i), \quad(1+i)^{5}, \quad(1+2 i)^{3} /(2-i)^{3}.
\]



\question%5
Find the two square roots of $-5-12 i$. Hence solve the quadratic equation $z^{2}-(4+i) z+(5+5 i)=0$.



\question%6
On separate Argand diagrams sketch the following sets.
\begin{parts}
\part%6a
$|z|<1$;

\part%6b
$\operatorname{Re}z=3$;

\part%6c
$|z-1|=|z+i|$;

\part%6d
$\arg(z-i)=\frac\pi4$;

\part%6e
$-\frac\pi4<\arg z<\frac\pi4$;

\part%6f
$\operatorname{Re}(z+1)=|z-1|$;

\part%6g
$|z-3-4i|=5$;

\part%6h
$\operatorname{Re}((1+i)z)=1$;

\part%6i
$\operatorname{Im}(z^3)>0$.
\end{parts}



\question%7 in case you haven't seen it before, "cis" is used as shorthand for "cosine plus i sine" by approximately 7 people on the planet because e^(i theta) already exists
Let $z=\operatorname{cis} \theta=e^{i \theta}$ and let $n$ be an integer.
\begin{parts}
\part%7a
Show that $2 \cos \theta=z+z^{-1}$ and that $2 i \sin \theta=z-z^{-1}$.

\part%7b
Using De Moivre's theorem, show further that $2 \cos n \theta=z^{n}+z^{-n}$ and that $2 i \sin n \theta=z^{n}-z^{-n}$.

\part%7c
Deduce that $16 \cos ^{5} \theta=\cos 5 \theta+5 \cos 3 \theta+10 \cos \theta$ and evaluate $\int_{0}^{\pi / 2} \cos ^{5} \theta \mathrm{~d} \theta$.
\end{parts}



\question%8
Let $\zeta=\operatorname{cis}(2 \pi / 5)=e^{2 \pi i / 5}$. Show that $1+\zeta+\zeta^{2}+\zeta^{3}+\zeta^{4}=0$. Show further that \[
	(z-\zeta-\zeta^{4})(z-\zeta^{2}-\zeta^{3})=z^{2}+z-1
\] and deduce that $\cos (2 \pi / 5)=(\sqrt{5}-1) / 4$.



\question%9
Write down the seven roots of $z^{7}+1=0$. By considering the coefficient of $z^{6}$ in the factorization of $z^{7}+1$, show that \[
	\cos \left(\frac{\pi}{7}\right)+\cos \left(\frac{3 \pi}{7}\right)+\cos \left(\frac{5 \pi}{7}\right)=\frac12.
\]



\question%10
\begin{parts}
\part%10a
Let $\zeta=\operatorname{cis}(\pi / 7)=e^{i \pi / 7}$. Simplify the expression $(\zeta-\zeta^{6})(\zeta^{3}-\zeta^{4})(\zeta^{5}-\zeta^{2})$. Hence show that \[
	\cos (\pi / 7) \cos (3 \pi / 7) \cos (5 \pi / 7)=-1 / 8
\]

\part%10b
Given that $\cos 7 \theta=64 \cos ^{7} \theta-112 \cos ^{5} \theta+56 \cos ^{3} \theta-7 \cos \theta$, rederive the result of (a).
\end{parts}



\section*{Optional Extension Questions}

\question%11
\begin{parts}
\part%11a
Let $a, b, c$ be positive real numbers. By expanding $(a+b+c)(a^{2}+b^{2}+c^{2}-a b-b c-c a)$ and considering the second factor as a quadratic in $a$, show that \[
	a^{3}+b^{3}+c^{3} \geqslant 3 a b c \qquad \text{if } a, b, c>0.
\]

\part%11b
Let $\omega^{3}=1, \omega \neq 1$. Show for any real numbers $a, b, c$ that \[
	a^3+b^3+c^3-3abc=(a+b+c)(a+\omega b+\omega^2c)(a+\omega^2b+\omega c)
\] Hence find all real numbers $a, b, c$ which satisfy $a^{3}+b^{3}+c^{3}=3 a b c$.
\end{parts}



\question%12
\begin{parts}
\part%12a
Let $p, q$ be complex numbers. Show that Viète's substitution $z=w-p /(3 w)$ turns the equation $z^{3}+p z=q$ into the quadratic in $w^{3}$.

\part%12b
Use Viète's substitution to solve the cubic $z^{3}-12 z+8=0$.
\end{parts}



\question%13
\begin{parts}
\part%13a
Let $P(z)$ be a polynomial with (possibly repeated) roots $\alpha_{1}, \alpha_{2}, \ldots, \alpha_{k}$. Show that \[
	\frac{P'(z)}{P(z)}=\frac1{z-\alpha_1}+\frac1{z-\alpha_2}+\cdots+\frac1{z-\alpha_k}.
\]

\part%13b
Deduce that if $\operatorname{Im} \alpha_{i}>0$ for each $i$, then $\operatorname{Im} \beta>0$ for any root $\beta$ of $P'(z)$.

\part%13c
Deduce further that if all the roots $\alpha$ of a polynomial $P(z)$ satisfy $|\alpha|<R$ then all the roots $\beta$ of $P'(z)$ satisfy $|\beta|<R$.
\end{parts}



\question%14
\begin{parts}
\part%14a
Show, for any complex number $z$ and positive integer $n$, that \[
	z^{2n}-1=(z^2-1)\prod_{k=1}^{n-1}\left\{z^2-2z\cos\left(\frac{k\pi}n\right)+1\right\}.
\]

\part%14b
Deduce that for any real $\theta$ that \[
	\sin n\theta=2^{n-1}\sin\theta\prod_{k=1}^{n-1}\left\{\cos\theta-\cos\left(\frac{k\pi}n\right)\right\}.
\]

\part%14c
Determine \[
	\prod_{k=1}^{2 n} \cos \left(\frac{k \pi}{2 n+1}\right).
\]
\end{parts}



\question%15
Let $k>0$.
\begin{parts}
\part%15a
Sketch the curve $C_{k}$ with equation $|z+1 / z|=k$.

\part%15b
What are the extreme values of $|z|$ on $C_{k}$?

\part%15c
Show that the curve $C_{2}$ consists of precisely two circles.
\end{parts}

\end{questions}

\end{document}
