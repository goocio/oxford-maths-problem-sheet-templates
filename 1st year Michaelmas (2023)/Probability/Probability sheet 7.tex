\documentclass[answers]{exam}
\usepackage{../MT2023}

\title{Probability -- Sheet 7}
\author{YOUR NAME HERE :)}
\date{Michaelmas Term 2023}


\begin{document}
\maketitle
\begin{questions}

\question%1
Sketch the cumulative distribution function of the following distributions:
\begin{parts}
\part%1a
the (discrete) uniform distribution on $\{1,2, \ldots, n\}$;

\part%1b
the (continuous) uniform distribution on $[a, b]$;

\part%1c
the exponential distribution with parameter 1;

\part%1d
the normal distribution with mean 0 and variance 1.
\end{parts}



\question%2
For each case below, does there exist a constant $c$ such that the given function is a probability density function? If so, find $c$ and find the cumulative distribution function. (In each case, the given function is zero outside the interval $(0,1)$.)
\begin{parts}
\part%2a
$f_{1}(x)=c x$ for $0<x<1$.

\part%2b
$f_{2}(x)=c x^{-1}$ for $0<x<1$.

\part%2c
$f_{3}(x)=c x^{-1 / 2}$ for $0<x<1$.

\part%2d
$f_{4}(x)=c(4 x^{3}-x)$ for $0<x<1$.
\end{parts}



\question%3
Let $U$ be a uniformly distributed random variable on $[0,1]$. Find
\begin{parts}
\part%3a
$\mathbb{E}[U]$ and $\operatorname{var}(U)$;

\part%3b
$\mathbb{P}(U<a \mid U<b)$ for $0<a<b<1$.
\end{parts}



\question%4
Let $X$ be exponentially distributed with parameter $\lambda$.
\begin{parts}
\part%4a
Find $\mathbb{P}(X>x)$.

\part%4b
Find $\mathbb{P}(a \leq X \leq b)$ for $0<a<b$.

\part%4c
Show that $\mathbb{P}(X>a+x \mid X>a)=\mathbb{P}(X>x)$ for $a, x>0$. [\emph{This is the memoryless property of the exponential distribution (compare to Question 3 on Sheet 3).}]

\part%4d
Find $\mathbb{P}(\sin X>\frac{1}{2})$.

\part%4e
Let $c>0$. What is the distribution of the random variable $c X$? [\emph{Try using part (a).}]

\part%4f
For $x \in \mathbb{R}$, let $\lceil x\rceil$ denote the ceiling of $x$; that is, the smallest integer greater than or equal to $x$. Show that the discrete random variable $\lceil X\rceil$ has a geometric distribution, and find its parameter. [\emph{Hint: write the event $\{\lceil X\rceil=k\}$ as $\{X \in I\}$ for some interval $I$.}]
\end{parts}



\question%5 ?? what is a "number" and what's it doing in my maths question
Blood plasma nicotine levels in smokers can be modelled by a normal random variable $X$ with mean 315 and variance $131^2$, the units being nanograms per millilitre.
\begin{parts}
\part%5a
What is the probability that a randomly chosen smoker has nicotine levels lower than 300?

\part%5b
What is the probability that a randomly chosen smoker has nicotine levels between 300 and 500?

\part%5c
If 20 smokers are to be tested what is the probability that at most one has a nicotine level higher than 500?
\end{parts}



\question%6
The radius of a circle is uniformly distributed on $[0, b]$. Find the cumulative distribution function, the probability density function, the expectation and the variance of the random variable representing the area of the circle.



\question%7 very cool question especially for programming enthusiasts and/or computer scientists since this is how computers can sample from different distributions, by "converting" between them
Let $X$ be a continuous random variable taking values in $[a, b]$ with c.d.f. $F_{X}$ which is strictly increasing on $[a, b]$.
\begin{parts}
\part%7a
Show that the random variable $F_{X}(X)$ has a uniform distribution on $[0,1]$.

\part%7b
Let $U$ be a uniform random variable on $[0,1]$. What is the distribution of the random variable $F_{X}^{-1}(U)$, where $F_{X}^{-1}$ is the inverse of $F_{X}$?

\part%7c
Suppose that $U_{1}, U_{2}, \ldots, U_{n}$ are a set of computer-generated pseudo-random numbers (assumed to be drawn from a uniform distribution on $[0,1]$). How would you use them to simulate a random sample $X_{1}, X_{2}, \ldots, X_{n}$ from the distribution with density \[
	f(x)=\mu e^{-\mu x}, \quad x \geq 0 ?
\]
\end{parts}

\end{questions}

\end{document}
