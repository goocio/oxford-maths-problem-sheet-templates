\documentclass[answers]{exam}
\usepackage{../MT2023}

\title{Linear Algebra -- Sheet 4\\Bases, Dimension formula}
\author{YOUR NAME HERE :)}
\date{Michaelmas Term 2023}

\newcommand*{\bump}{\vspace{1em}\phantom{}\vspace{-1.75em}}


\begin{document}
\maketitle
\section*{Main course}
\begin{questions}

\question%1
Find the row rank of each of the following matrices:
\begin{parts}
\part%1a
\bump \[
	\begin{pmatrix}
		2 & 4 & -3 & 0 \\
		1 & -4 & 3 & 0 \\
		3 & -5 & 2 & 1
	\end{pmatrix}
\]

\part%1b
\bump \[
	\begin{pmatrix}
		1 & 2 & 3 & 0 \\
		2 & 3 & 4 & 1 \\
		3 & 4 & 5 & 2
	\end{pmatrix}
\]

\part%1c
\bump \[
	\begin{pmatrix}
		1 & 2 & 3 & 0 \\
		2 & 3 & 4 & 2 \\
		3 & 4 & 5 & 2
	\end{pmatrix}
\]
\end{parts}



\question%2
Let $V$ be a vector space and let $U, W$ be subspaces of $V$. Show that $U \backslash W$ is never a subspace. Find a necessary and sufficient condition for $U \cup W$ to be a subspace.



\question%3
Let $V=\mathbb{R}^{n}$ where $n \geqslant 2$, and let $U$ and $W$ be subspaces of $V$ of dimension $n-1$.
\begin{parts}
\part%3a
Show that if $U \neq W$ then $\dim(U \cap W)=n-2$.

\part%3b
Now suppose that $n \geqslant 3$ and let $U_{1}, U_{2}, U_{3}$ be three distinct subspaces of dimension $n-1$. Must it be true that $\dim\left(U_{1} \cap U_{2} \cap U_{3}\right)=n-3$? Give a proof or find a counterexample.

\part%3c
Show that if $\dim U \leqslant n-2$ then there are infinitely many different subspaces $X$ such that $U \leqslant X \leqslant V$.
\end{parts}



\question%4
Let $V=\mathbb{R}^{4}$, and let \begin{align*}
	X &= \{(x_1,x_2,x_3,x_4) \in V: x_2+x_3+x_4=0\},\\
	Y &= \{(x_1,x_2,x_3,x_4) \in V: x_1+x_2=0=x_3-2x_4\}.
\end{align*} Find bases for $X, Y, X \cap Y$ and $X+Y$, and write down the dimensions of these subspaces.



\question%5
Let $V$ be an $n$-dimensional vector space.
\begin{parts}
\part%5a
Prove that $V$ contains a subspace of dimension $r$ for each $r$ in the range $0 \leqslant r \leqslant n$.

\part%5b
Show that if $U_{0}<U_{1}<\cdots<U_{k} \leqslant V$ (strict containments of subspaces of $V$) then $k \leqslant n$. Show also that if $k=n$ then $\dim U_{r}=r$ for $0 \leqslant r \leqslant k$.

\part%5c
Let $U, W$ be subspaces of $V$ such that $U \leqslant W$. Show that there is a subspace $X$ of $V$ such that $W \cap X=U$ and $W+X=V$.
\end{parts}



\question%6
\begin{parts}
\part%6a
Show that $V=U \oplus W$ if and only if every vector $v \in V$ can be expressed uniquely in the form $v=u+w$ with $u \in U$ and $w \in W$.

\part%6b
Let $V=\mathbb{R}^{3}$ and $U=\{(x_{1}, x_{2}, x_{3}) \in V: x_{1}+x_{2}+x_{3}=0\}$. For each of the following subspaces $W$, either prove that $V=U \oplus W$, or explain why this is not true:
\begin{subparts}
\subpart%6bi
$W=\{(x, 0,-x): x \in \mathbb{R}\}$;

\subpart%6bii
$W=\{(x, 0, x): x \in \mathbb{R}\}$;

\subpart%6biii
$W=\{(x_{1}, x_{2}, x_{3}) \in V: x_{1}=x_{3}\}$.
\end{subparts}
\end{parts}

\end{questions}



\section*{Starter}
\begin{questions}

\question%S1
For each of the following sets $S$ in a vector space $V$, find a basis for $\langle S\rangle$.
\begin{parts}
\part%S1a
$S=\{(1,0,3),(-2,5,4)\} \subseteq \mathbb{R}^{3}$;

\part%S1b
$S=\{(6,2,0,-1),(3,5,9,-2),(-1,0,7,8),(5,5,-1,2)\} \subseteq \mathbb{R}^{4}$;

\part%S1c
$S=\{(7,-3,2,11,2),(0,4,9,-5,16),(20,-13,16,24,38),(1,12,8,-1,0)\} \subseteq \mathbb{R}^{5}$.
\end{parts}



\question%S2
Let $U, W$ be subspaces of a finite-dimensional vector space $V$. Show the following are equivalent:
\begin{parts}
\part%S2a
$V=U \oplus W$;

\part%S2b
every $v \in V$ has a unique expression as $u+w$ where $u \in U$ and $w \in W$;

\part%S2c
$\dim V=\dim U+\dim W$ and $V=U+W$;

\part%S2d
$\dim V=\dim U+\dim W$ and $U \cap W=\left\{0_{V}\right\}$;

\part%S2e
if $u_{1}, \ldots, u_{m}$ is a basis for $U$ and $w_{1}, \ldots, w_{n}$ is a basis for $W$, then $u_{1}, \ldots, u_{m}, w_{1}, \ldots, w_{n}$ is a basis for $V$.
\end{parts}


\question%S3
Find an example to show that it is not the case that if $V=U \oplus W$ then every basis of $V$ is a union of a basis of $U$ and a basis of $W$.

\end{questions}



\section*{Pudding}
\begin{questions}

\question%P1
Find the column rank of each of the matrices in Q1 on this sheet. What do you notice?



\question%P2
Taking $A$ as each of the three matrices in Q1 on this sheet, find all the solutions to $A \mathbf{x}=\mathbf{0}$ in each case. What do you notice?



\question%P3
Let $V=V_{1} \oplus V_{2}$. Take a subspace $U \leqslant V$. Must it be true that $U=U_{1} \oplus U_{2}$ where $U_{1} \leqslant V_{1}$ and $U_{2} \leqslant V_{2}$? (Find a proof or give a counterexample.)

\end{questions}

\end{document}
