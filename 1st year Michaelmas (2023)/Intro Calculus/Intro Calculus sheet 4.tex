\documentclass[answers]{exam}
\usepackage{../MT2023}

\title{Intro Calculus -- Sheet 4}
\author{YOUR NAME HERE :)}
\date{Michaelmas Term 2023}

\newcommand*{\bump}{\vspace{1em}\phantom{}\vspace{-1.75em}}


\begin{document}
\maketitle
\begin{questions}

\question%1
Given that \[
	xs^2+yt^2=1 \qquad \text{and} \qquad x^2s+y^2t=xy-4,
\] where $x=x(s, t)$ and $y=y(s, t)$, find $x_{s}$, $x_{t}$, $y_{s}$ and $y_{t}$ at the point $(x, y, s, t)=(1,-3,2,-1)$.



\question%2
Given that \begin{align*}
	z & =x^{2}+x y, \\
	x^{2}+y^{3} & =s t+5, \\
	x^{3}-y^{2} & =s^{2}+t^{2},
\end{align*} where $x=x(s, t)$ and $y=y(s, t)$, find expressions for $\frac{\partial z}{\partial s}$ and $\frac{\partial z}{\partial t}$. Evaluate these expressions at the point $(x, y, s, t)=(3,1,1,5)$.



\question%3 You'll see the heat equation again in FSPDEs in Hilary
Let $t>0$. Verify that \[ T(x, t)=\frac{A}{\sqrt{t}} \exp \left(\frac{-x^{2}}{4 \kappa t}\right) \] is a solution of the heat equation \[
	\frac{\partial T}{\partial t}=\kappa\frac{\partial^2T}{\partial x^2}.
\] Sketch $T$ as a function of $x$ at two different times $t$.



\question%4
Find general solutions $f(x, y, z)$ of the following PDES:
\begin{parts}
\part%4a
\bump \[
	\frac{\partial^{3} f}{\partial z^{3}}=0;
\]

\part%4b
\bump \[
	\frac{\partial^{3} f}{\partial x \partial y \partial z}=0.
\]
\end{parts}



\question%5
Find general solutions $u(x, y)$ of the following PDES:
\begin{parts}
\part%5a
\bump \[
	y \frac{\partial u}{\partial y}=u;
\]

\part%5b
\bump \[
	\dfrac{\partial u}{\partial x}=2 x y u.
\]
\end{parts}



\question%6
Consider the functions of the form $u(x, t)=e^{\beta t} g(x)$ to find solutions of the equation \[
	\frac{\partial^2u}{\partial x^2}+2\frac{\partial^2u}{\partial x\partial t}+\frac{\partial^2u}{\partial t^2}=0.
\]



\question%7
Find separable solutions, of the form $z(x, y)=X(x) Y(y)$, to the following PDEs:
\begin{parts}
\part%7a
\bump \[
	\frac{\partial z}{\partial y}=y \frac{\partial z}{\partial x};
\]

\part%7b
\bump \[
	x \frac{\partial z}{\partial x}=z+y \frac{\partial z}{\partial y}.
\]
\end{parts}

\end{questions}

\end{document}
