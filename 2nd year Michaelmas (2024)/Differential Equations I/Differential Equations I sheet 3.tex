\documentclass[answers]{exam}
\usepackage{../MT2024}

\title{Differential Equations I -- Sheet 3\\Plane autonomous systems, Semilinear PDEs}
\author{YOUR NAME HERE :)}
\date{Michaelmas Term 2024}
% accurate as of 02/10/2024


\begin{document}
\maketitle
\begin{questions}

\question%1
\begin{parts}
\part%1a
Write the second-order ODE \[
	\ddot{x}-\dot{x}^{2}+x^{2}-x=0
\] as a plane autonomous system. Find and classify the critical points of this system. Analyse each critical point in detail and hence sketch the phase plane including the arrows for the trajectories.

\part%1b
Show that there are periodic orbits of the original ODE. [\emph{Hint: Either: make the transformation $u=x+y, v=x-y$ and solve the resulting equation for $d u / d v$ to write the characteristics as level sets of a function and then argue graphically that these level sets are closed curves; or exploit the symmetry of the phase plane about the $x$ axis to prove the claim.}]
\end{parts}



\question%2
\begin{parts}
\part%2a
Show that the system \[
	\frac{\mathrm{d} x}{\mathrm{d} t}=-y-x(x^{2}+y^{2}),\qquad
	\frac{\mathrm{d} y}{\mathrm{d} t}=x-y(x^{2}+y^{2})
\] has no closed trajectories. Show that in polar coordinates the system becomes \[
	\frac{\mathrm{d} r}{\mathrm{d} t}=-r^{3}, \quad
	\frac{\mathrm{d} \theta}{\mathrm{d} t}=1
\] and sketch the trajectories (you can solve this system explicitly).

\part%2b
Show that the only critical point is the origin. Consider a linear approximation to the $(x, y)$ system in a neighbourhood of the origin and determine its type. Comment on any differences between the approximation and the exact behaviour.
\end{parts}



\question%3
Find $z(x, y)$ explicitly if \[
	-y z_{x}+x z_{y}=\frac{x z}{\sqrt{x^{2}+y^{2}}}
\] and \[
	z(x, 0)=1 \quad \text{for } x \geq 1.
\] Describe the projections of the characteristic curves upon the $(x, y)$-plane. In what region of the plane is $z(x, y)$ determined uniquely?



\question%4
Find in parametric form the characteristics of the differential equation \[
	x z_{x}+y z_{y}=2 z
\] and describe the characteristic projections into the $(x, y)$-plane. Find in explicit form the solution satisfying $z=x^{3}$ on $x+y=1$. In what region of the plane is the solution uniquely determined by the data?



\question%5
Consider the differential equation \[
	z_{x}-y z_{y}=-z
\] with data $z(0, y)=\min (1, y)$ for $-\infty<y<\infty$. (Here $\min (1, y)$ means the lesser of 1 and $y$, so the data is not smooth; draw its graph.) Show that the solution of this problem is the following: \[
	z=\min\{y,e^{-x}\}=\begin{cases}
		e^{-x}&y\geq e^{-x},\\
		y&y<e^{-x}.
	\end{cases}
\] Try to sketch the solution surface.



\question%6
Consider the differential equation \[
	x z_{x}+y z_{y}=(x+y) z
\] with $z=1$ on $y=x^{2}+1 / 4$ for $x \geq 0$.
\begin{parts}
\part%6a
Parametrise the initial curve as $\left(s, s^{2}+1 / 4,1\right), s \geq 0$, and find the characteristics of the differential equation, $x=x(s, t), y=y(s, t), z=z(s, t)$.

\part%6b
Identify the two separate segments of the data curve where the data is Cauchy. Find the domain of definition for each segment. On a sketch show the initial curve with these domains. For each segment solve the first two equations for $s$, and find $z$ explicitly for each segment. Comment on why we need to consider the segments separately.
\end{parts}

\end{questions}

\end{document}
