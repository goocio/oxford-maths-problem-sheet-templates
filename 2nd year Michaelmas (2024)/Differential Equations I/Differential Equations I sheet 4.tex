\documentclass[answers]{exam}
\usepackage{../MT2024}

\title{Differential Equations I -- Sheet 4\\Second order semilinear PDEs}
\author{YOUR NAME HERE :)}
\date{Michaelmas Term 2024}
% accurate as of 02/10/2024


\begin{document}
\maketitle
\begin{questions}

\question%1
Show that the equation \[
	y u_{x x}+(x+y) u_{x y}+x u_{y y}=0
\] is hyperbolic everywhere except on the line $y=x$. Find the characteristic variables, reduce the equation to canonical form, and show that the general solution is \[
	u=\frac{1}{y-x} f(y^{2}-x^{2})+g(y-x).
\]



\question%2
Consider the partial differential equation \[
	e^{2 y} u_{x x}+u_{y}=u_{y y}.
\]
\begin{parts}
\part%2a
Write down the differential equation satisfied by its characteristic curves and show that $\phi=x+e^{y}$ and $\psi=x-e^{y}$ are characteristic variables for the partial differential equation.

\part%2b
Reduce the equation to canonical form and find the solution of the equation for which $u=x$ and $u_{y}=1$ on the line $y=0,0 \leq x \leq 1$.

\part%2c
Sketch the characteristic curves $x+e^{y}=1, x+e^{y}=2, x-e^{y}=-1, x-e^{y}=0$. In what region of the $x, y$-plane is your solution uniquely determined by the initial data? Show this region on your diagram.
\end{parts}



\question%3
Let $D=(a_{1}, a_{2}) \times(0, \tau)$ for some $\tau>0$ and $a_{1}<a_{2}$ and let $f: D \to \mathbb{R}$, $g_{1,2}:[0, \tau] \to \mathbb{R}$ and $u_{0}:[a_{1}, a_{2}] \to \mathbb{R}$ be continuously differentiable functions so that $g_{1,2}(0)=u_{0}(a_{1,2})$.
\begin{parts}
\part%3a
Show that the solution $u=u(x, t)$ of the inhomogeneous heat equation \begin{equation}
	\partial_{t} u-\partial_{x x} u=f\text{ on } D
\end{equation} with initial and boundary values \begin{equation}
	u(x, 0)=u_{0}(x) \text { for } x \in\left[a_{1}, a_{2}\right],\quad
	u(a_{1,2}, t)=g_{1,2}(t) \text { for } t \in[0, \tau]
\end{equation} is unique and depends continuously on the initial and boundary data, if it exists.

\part%3b
Prove that the only non-negative solution of \[
	\partial_{t} u-\partial_{x x} u=-u^{2}
\] with $u(a_{1,2}, t)=0, t \in[0, \tau]$ and $u(x, 0)=0, x \in[a_{1}, a_{2}]$ is the trivial solution $u=0$.

\part%3c
Show that the solution of (1) is also unique if we replace the boundary condition $u(a_{2}, t)=g_{2}(t)$ in (2) by the condition that \[
	\partial_{x} u(a_{2}, t)+u(a_{2}, t)=g_{2}(t)\quad
	\text{for } t \in(0, \tau)
\] [\emph{Hint: Show that if $u_{1}$ and $u_{2}$ are two solutions, then $u=u_{1}-u_{2}$ can not achieve a positive maximum on $x=a_{2}$.}]

\part%3d
Show that if $u$ is a positive function which solves \[
	\partial_{t} u=\partial_{x}(u \partial_{x} u)+u \text { in } D
\] then $u(x, t)$ attains its minimum value on the parabolic boundary \[
	\partial_{P}D=[a_1,a_2]\times\{0\}\cup\{a_{1}\}\times[0,\tau]\cup\{a_{2}\}\times[0,\tau].
\]
\end{parts}



\question%4
Let $D \subset \mathbb{R}^{2}$ be a bounded open set, let $u: \mathbb{R}^{2} \to \mathbb{R}$ be twice continuously differentiable.
\begin{parts}
\part%4a
Show that if $u$ achieves its maximum in a point $(x_{0}, y_{0}) \in D$ then the Hessian matrix at $(x_{0}, y_{0})$ is negative semi-definite, i.e. \[
	A:=\begin{pmatrix}
		u_{x x}(x_{0}, y_{0}) & u_{x y}(x_{0}, y_{0}) \\
		u_{x y}(x_{0}, y_{0}) & u_{y y}(x_{0}, y_{0})
	\end{pmatrix}
\] satisfies $(v_1,v_2)A\binom{v_1}{v_2}\leq0$ for all $(v_{1}, v_{2})\in\mathbb R^2.$ [\emph{Hint: Consider $t \mapsto u((x_{0}, y_{0})+t(v_{1}, v_{2}))$.}]

\part%4b
Suppose that $u$ satisfies \[
	x^{2} u_{x x}-14 x y u_{x y}+(49 y^{2}+1) u_{y y} \geq 0 \text { in } D.
\] Adapt the proof of the Maximum principle for the Laplace operator to show that $u$ achieves its maximum over $\bar{D}$ on the boundary $\partial D$.
\end{parts}

\end{questions}

\end{document}
