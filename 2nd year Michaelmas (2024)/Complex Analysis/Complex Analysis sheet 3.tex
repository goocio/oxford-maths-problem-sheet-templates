\documentclass[answers]{exam}
\usepackage{../MT2024}

\title{Complex Analysis -- Sheet 3\\Cauchy's integral formula, Entire functions, Series expansions}
\author{YOUR NAME HERE :)}
\date{Michaelmas Term 2024}
% accurate as of 15/11/2024


\begin{document}
\maketitle
\begin{questions}

\question%1
Let $\gamma$ be the positively oriented unit circle. Use Cauchy integral formulas to evaluate:
\begin{parts}
\part%1a
$\displaystyle\int_\gamma\frac{e^z\cos z}z~\mathrm dz$;

\part%1b
$\displaystyle\int_\gamma\frac{e^z\cos z}{z^3}~\mathrm dz$.
\end{parts}



\question%2
(Part (a) is known as the \emph{Mean value property}; parts (b) and (c) are known as the \emph{maximum modulus principle}.) Suppose that $f: U \to \mathbb{C}$ is a holomorphic function on a domain $U$.
\begin{parts}
\part%2a
Show that, if $\bar{B}(a, r) \subseteq U$ then \[
	f(a)=\frac{1}{2 \pi} \int_{0}^{2 \pi} f(a+r e^{i t}) \mathrm{~d} t.
\]

\part%2b
Suppose that $a \in U$ is such that $|f(a)|$ is a maximum for $|f|: U \to \mathbb{R}$. Show that $|f|$ must in fact be constant near $a$.

\part%2c
Deduce that $f$ is constant on all of $U$. [\emph{Hint: Consider the set $S=\{z \in U:|f(z)|=|f(a)|\}$.}]
\end{parts}



\question%3
(Extra practice) Suppose $w_1,...,w_n$ are points on the unit circle. Show that there is a point $z$ on the unit circle such that $\prod|z-w_k|\geqslant1$.



\question%4
Suppose that $f$ is holomorphic on $\mathbb{C}$ and that $\Re f(z) \geqslant 0$ for all $z$. Show that $f$ is constant. [\emph{Hint: consider $\exp (-f(z))$.}]



\question%5
Let $f$ be an entire function.
\begin{parts}
\part%5a
Prove that $f$ is a polynomial of degree at most $k$ if and only if there exist real constants $M, R>0$ and an integer $k$ such that \[
	|f(z)| \leqslant M|z|^{k} \quad \text { for }|z|>R.
\]

\part%5b
Let $g(z)$ be an entire function such that $|f(z)| \leqslant|g(z)|$ for all $z \in \mathbb{C}$. Show that that there is some $c \in \mathbb{C}$ so that $f(z)=c g(z)$ for all $z \in \mathbb{C}$. [\emph{Hint: consider $f/g$.}]
\end{parts}



\question%6
(Extra challenge)
Let $f=u+iv$ be a function holomorphic in some domain containing the closed unit disc $\mathbb D$.
\begin{parts}
\part%6a
Show that \[
	u(0)=\frac1{2\pi}\int_0^{2\pi}u(e^{it})~\mathrm dt.
\]

\part%6b
In this question you can assume that for every $z_0\in\mathbb D$ \[
	h_{z_0}(z)\coloneqq\frac{z+z_0}{1+z\overline{z_0}}
\] is a bijection from $\mathbb D$ to itself. Using this map show that \[
	u(z)=\frac1{2\pi}\int_0^{2\pi}u(e^{it})\frac{1-|z|^2}{|e^{it}-z|^2}~\mathrm dt,\qquad
	z\in\mathbb D.
\]

\part%6c
Using the fact that \[
	\frac{|w|^2-|z|^2}{|z-w|^2}=\operatorname{Re}\left(\frac{w+z}{w-z}\right),
\] show that \[
	f(z)=\frac1{2\pi}\int_0^{2\pi}u(e^{it})\frac{e^{it}+z}{e^{it}-z}~\mathrm dt+iv(0).
\]
\end{parts}


\question%7
Suppose that $f:\mathbb C\to\mathbb C$ is \emph{entire} (i.e. holomorphic on the whole complex plane).
\begin{parts}
\part%7a
If $f(1/n)=1/n$ for all $n\in\mathbb N$ must $f(z)=z$ for all $z\in\mathbb C$?

\part%7b
If $f(n)=n$ for all $n\in\mathbb N$ must $f(z)=z$ for all $z\in\mathbb C$?

\part%7c
Show that there must be some $n\in\mathbb N$ such that $f(1/n)\neq1/(n+1)$.
\end{parts}



\question%8
(Extra challenge) Suppose that $f: \mathbb{C} \to \mathbb{C}$ is an entire function, and for each $z_{0} \in \mathbb{C}$ the power series expansion $f(z)=\sum_{n=0}^{\infty} c_{n}(z-z_{0})^{n}$ has at least one $c_{n}$ equal to zero. Prove that $f(z)$ is a polynomial. [\emph{Hint: Note that $n!c_{n}=f^{(n)}(z_{0})$ and that $\mathbb{C}$ is uncountable.}]



\question%9
\begin{parts}
\part%9a
Let $c_{n}$ be a bounded sequence of complex numbers. Show that \[
	\sum_{n=1}^{\infty} c_{n} \frac{z^{n}}{1+z^{n}}
\] is holomorphic at any $z \in \mathbb{C}$ with $|z|<1$. [\emph{Hint: Does the series converge uniformly on some sets?}]

\part%9b
(Extra practice) Show that \[
	\sum_{n=1}^{\infty} \frac{1}{(n-z)^{2}}
\] is holomorphic on $\mathbb{C} \setminus \mathbb{N}$.
\end{parts}



\question%10
Let \[
	F(z)=\frac{2z^2-1}{z^2(z-1)}.
\] Find Laurent expansions for $F$ in:
\begin{parts}
\part%10a
$A_1=\{z:|z-2|<1\}$;

\part%10b
$A_2=\{z:1<|z-2|<2\}$;

\part%10c
$A_3=\{z:|z-2|>2\}$.
\end{parts}



\question%11
(Extra challenge) (The result of this question is known as the \emph{Schwarz reflection principle}.) Let $U$ be a domain in $\mathbb C$ that is symmetric with respect to the real line. Namely, $z\in U$ if and only if $\overline z\in U$. You may assume that its intersection with $\mathbb R$ is a finite disjoint union of intervals. Let $f$ be a function that is holomorphic in $U\cap\{z:\operatorname{Im}z>0\}$, continuous in $U\cap\{z:\operatorname{Im}z\geq0\}$, and its values on the real line are real.
\begin{parts}
\part%11a
Define a function $g$ in $U$ by \[
	g(z)\coloneqq\begin{cases}
		f(z)&z\in U\cap\{z:\operatorname{Im}z\geq0\},\\
		f(\overline z)&z\in U\cap\{z:\operatorname{Im}z<0\}.
	\end{cases}
\] Show that $g$ is holomorphic in $U$.

\part%11b
Formulae a similar reflection principle for reflection with respect to a circle. [\emph{Hint: can you write the reflection with respect to a circle as a complex transformation?}]
\end{parts}

\end{questions}

\end{document}
