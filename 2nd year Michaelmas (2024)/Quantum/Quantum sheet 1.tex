\documentclass[answers]{exam}
\usepackage{../MT2024}

\title{Quantum -- Sheet 1\\Schrödinger equation, Energy}
\author{YOUR NAME HERE :)}
\date{Michaelmas Term 2024}
% accurate as of 15/10/2024

\def\sech{\operatorname{sech}}


\begin{document}
\maketitle

\begin{questions}

\question%1
The potential energy for an electron in a hydrogen atom is \[
	V(r)=-\frac{e^{2}}{4 \pi \epsilon_{0} r}
\] where $-e$ is the charge of the electron, $r$ is its distance from the nucleus, and $\epsilon_{0}$ is a constant. In a circular orbit the electron has angular momentum $L=m v r$, where $m$ is the electron mass and $v$ is its speed. In 1913 Bohr proposed that $L$ is quantized, satisfying $L=n \hbar$ where $n$ is a positive integer.
\begin{parts}
\part%1a
Given that Newton's second law for circular orbits is \[
	m \frac{v^{2}}{r}=V'(r)
\] show that Bohr's quantization implies $r=n^{2} a$, where $a=\frac{4 \pi \epsilon_{0} \hbar^{2}}{m e^{2}}$ is called the Bohr radius.

\part%1b
Show that the total energy $E=\frac{1}{2} m v^{2}+V(r)$ is given by \[
	E=-\frac{\hbar^{2}}{2 m a^{2}} \cdot \frac{1}{n^{2}}
\] [\emph{This successfully reproduces the hydrogen atom energy levels (1.3) in the lecture notes, but a full quantum mechanical treatment will only appear at the end of our course.}]
\end{parts}



\question%2
A particle of mass $m$ moves in the interval $[-a, a]$ where the potential $V=V_{0}$ is constant. Using the stationary state Schrödinger equation show that the energy levels of the system are \[
	E_{n}=V_{0}+\frac{n^{2} \pi^{2} \hbar^{2}}{8 m a^{2}}
\] where $n$ is a positive integer, and find the corresponding normalized wave functions. Show that the wave functions are all either even or odd functions of $x$.



\question%3
A particle of mass $m$ moving on the $x$-axis has a (non-normalized) ground state wave function $\sech^2 x$ with energy $-2 \hbar^{2} / m$.
\begin{parts}
\part%3a
Show that the potential is $V(x)=-\frac{3 \hbar^{2}}{m} \sech^{2} x$.

\part%3b
An excited state wave function for the particle is $\psi(x)=\tanh x \sech x$. What is the energy of this state?
\end{parts}



\question%4
Consider a particle of mass $m$ confined to a box in three dimensions, with potential \[
	V(x, y, z)= \begin{cases}0,&0<x<a, 0<y<b, 0<z<c \\ +\infty&\text{ otherwise}\end{cases}
\] where $(x, y, z)$ are Cartesian coordinates. By separating variables in the stationary state Schrödinger equation, show that the allowed energies of the particle are \[
	E_{n_{1}, n_{2}, n_{3}}=\frac{\pi^{2} \hbar^{2}}{2 m}\left(\frac{n_{1}^{2}}{a^{2}}+\frac{n_{2}^{2}}{b^{2}}+\frac{n_{3}^{2}}{c^{2}}\right)
\] where $n_{1}, n_{2}, n_{3}$ are positive integers, and find the corresponding normalized wave functions. [\emph{You may use the results for the one-dimensional box.}]



\question%5
Consider a particle of mass $m$ moving in two dimensions in the potential \[
	V(x, y)= \begin{cases}V_{0}, & 0<x<a \\ +\infty & \text { otherwise }\end{cases}
\] where $(x, y)$ are Cartesian coordinates. (Note that the potential is independent of $y$.) Determine the allowed energies of the particle, and find the corresponding unnormalized wave functions. [\emph{You may use the results for the one-dimensional box and the one-dimensional free particle.}]



\question%6
A particle of mass $m$ moves on the $x$-axis in a potential $V(x)$, where $V$ is an even function (that is $V(x)=V(-x)$). Let $\psi(x)$ be a normalized wave function satisfying the stationary state Schrödinger equation with energy $E$.
\begin{parts}
\part%6a
Show that $\tilde{\psi}(x) \equiv \psi(-x)$ is also a normalized wave function.

\part%6b
By considering the wave functions $\psi_{ \pm}=\psi \pm \tilde{\psi}$, or otherwise, deduce that there is either an even or an odd wave function (or both) satisfying the same Schrödinger equation.
\end{parts}



\question%7
Suppose that $\Psi(x, t)$ satisfies the one-dimensional time-dependent Schrödinger equation with potential $V(x)$ (assumed real). We define $\rho(x, t)=|\Psi(x, t)|^{2}$ and \[
	j(x, t)=\frac{\mathrm{i} \hbar}{2 m}\left(\Psi \frac{\partial \overline{\Psi}}{\partial x}-\overline{\Psi} \frac{\partial \Psi}{\partial x}\right).
\]
\begin{parts}
\part%7a
Show that, as a consequence of the Schrödinger equation, \[
	\frac{\partial \rho}{\partial t}+\frac{\partial j}{\partial x}=0.
\]

\part%7b
Show further that $j$ vanishes identically if and only if there exists a nowhere zero function $\lambda(t)$ such that $\lambda(t) \Psi(x, t)$ takes only real values.
\end{parts}



\question%8
(Optional.) Verify that the Gaussian wave packet \[
	\Psi(x, t)=\frac{1}{\pi^{1 / 4} \sqrt{1+(\mathrm{i} \hbar t / m)}} \exp \left[-\frac{x^{2}}{2[1+(\mathrm{i} \hbar t / m)]}\right]
\] satisfies the free Schrödinger equation and is normalized for all times $t$.

\end{questions}

\end{document}
