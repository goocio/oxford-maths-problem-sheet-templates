\documentclass[answers]{exam}
\usepackage{../MT2024}
\usepackage{bbm} % for \mathbbm{1} since \mathbb{1} outputs garbage

\title{Quantum -- Sheet 3\\Harmonic oscillators, Hamiltonians}
\author{YOUR NAME HERE :)}
\date{Michaelmas Term 2024}
% accurate as of 15/10/2024


\begin{document}
\maketitle

\begin{questions}

\question%1
Define a linear operator $R$ acting on wave functions $\psi$ on the $x$-axis by \[
	(R \psi)(x)=\psi(-x)
\] This is called the \emph{parity operator}.
\begin{parts}
\part%1a
Show that $R$ is self-adjoint, and $R^{2}=\mathbbm{1}$.

\part%1b
What are the possible eigenvalues of $R$, and how can its eigenspaces be characterized?

\part%1c
Suppose now that a particle of mass $m$ moves under an even potential $V(x)$, so that $V(x)=V(-x)$.
\begin{subparts}
\subpart%1ci
Show that $R$ commutes with the Hamiltonian $H$, i.e. $(R H-H R) \psi=0$ for all $\psi(x)$.

\subpart%1cii
Show that $R \psi$ is an eigenstate of $H$ with energy $E$ if and only if $\psi$ is. By considering $\psi \pm R \psi$, deduce that there is either an even or an odd eigenstate (or both) with energy $E$.
\end{subparts}
\end{parts}



\question%2
Show that for any infinitely differentiable function $\psi(x)$ whose Taylor series converges to $\psi(x)$, one has for all real $s$ \[
	\left(\mathrm{e}^{-\mathrm{i} s P / \hbar} \psi\right)(x)=\psi(x-s)
\] where $P$ is the momentum operator. Deduce that on the subspace of such functions one has the equality of operators \[
	\mathrm{e}^{-\mathrm{i} s P / \hbar} X \mathrm{e}^{\mathrm{i} s P / \hbar}=X-s \mathbbm{1}
\] where $X$ is the position operator and $\mathbbm{1}$ is the identity operator.



\question%3
\begin{parts}
\part%3a
Show that the expectation value $\mathbb{E}_{\psi}(A)=\langle\psi \mid A \psi\rangle$ of an observable $A$ in a state $\psi$ is necessarily real.

\part%3b
Show the converse result: if $\langle\psi \mid A \psi\rangle$ is real for all $\psi$ then $A$ satisfies \[
	\langle\psi_{1} \mid A \psi_{2}\rangle=\langle A \psi_{1} \mid \psi_{2}\rangle
\] for all $\psi_{1}, \psi_{2}$, implying that $A$ is self-adjoint. [\emph{Hint: look at $\psi=\psi_{1} \pm \psi_{2}$ and $\psi=\psi_{1} \pm \mathrm{i} \psi_{2}$.}]
\end{parts}



\question%4
Consider the state space $\mathcal{H}=\mathbb{C}^{3}$, so that a wave function is a three-component column vector $\psi=\left(\psi_{1}(t), \psi_{2}(t), \psi_{3}(t)\right)^{T}$. The Hamiltonian is \[
	H=\hbar \omega\begin{pmatrix}
		1 & 2 & 0 \\
		2 & 0 & 2 \\
		0 & 2 & -1
	\end{pmatrix}
\] with Schrödinger equation $\mathrm{i} \hbar \frac{\mathrm{d} \psi}{\mathrm{d} t}=H \psi$, and stationary state equation $H \psi=E \psi$.
\begin{parts}
\part%4a
Find the stationary states of this quantum system.

\part%4b
Consider the observable \[
	A=\begin{pmatrix}
		1 & 0 & 0 \\
		0 & 0 & 0 \\
		0 & 0 & -1
	\end{pmatrix}
\] and suppose that at time $t=0$ the eigenvalue 1 has just been measured.
\begin{subparts}
\subpart%4bi
Find $\psi(t)$ at subsequent times $t$ by solving the Schrödinger equation.

\subpart%4bii
What is the probability that when $A$ is measured at time $t$ one again obtains the eigenvalue 1?
\end{subparts}
\end{parts}



\question%5
\begin{parts}
\part%5a
Prove Ehrenfest's Theorem: for any observable $A$, \[
	\frac{\mathrm{d}}{\mathrm{d} t}\langle A\rangle=-\frac{\mathrm{i}}{\hbar}\langle[A, H]\rangle+\left\langle\frac{\partial A}{\partial t}\right\rangle
\] where we have denoted expectation value $\langle A\rangle \equiv \mathbb{E}_{\psi}(A)$, and $\psi$ is arbitrary. Note here that $A$ might potentially depend explicitly on time $t$, hence the last term.

\part%5b
Hence show that for the Hamiltonian $H=P^{2} / 2 m+V(X)$ we have \[
	\frac{\mathrm{d}}{\mathrm{d} t}\langle X\rangle=\frac{1}{m}\langle P\rangle, \quad \frac{\mathrm{d}}{\mathrm{d} t}\langle P\rangle=-\left\langle V'(X)\right\rangle
\] and deduce that $m \frac{\mathrm{d}^{2}}{\mathrm{~d} t^{2}}\langle X\rangle=-\left\langle V^{\prime}(X)\right\rangle$. Do you recognize this equation?
\end{parts}



\question%6
The state $\psi=\psi_{n}$ is a normalized eigenvector for the energy level $E=E_{n}=\left(n+\frac{1}{2}\right) \hbar \omega$ of the harmonic oscillator with Hamiltonian $H=P^{2} / 2 m+\frac{1}{2} m \omega^{2} X^{2}$.
\begin{parts}
\part%6a
Show that \[
	E=\frac{1}{2 m} \mathbb{E}_{\psi}(P^{2})+\frac{1}{2}m\omega^{2}\mathbb{E}_{\psi}(X^{2}).
\]

\part%6b
By considering $\left\langle\psi \mid(P \pm \mathrm{i} m \omega X)^{k} \psi\right\rangle$ for $k=1,2$, and using orthogonality of eigenstates, or otherwise, show that \[
	\mathbb{E}_{\psi}(P)=0=\mathbb{E}_{\psi}(X), \quad \mathbb{E}_{\psi}(P^{2})=m^{2} \omega^{2} \mathbb{E}_{\psi}(X^{2})=m E
\]

\part%6c
Deduce that $\Delta_{\psi}(X) \Delta_{\psi}(P)=E / \omega$, and discuss how this relates to Heisenberg's uncertainty principle.
\end{parts}



\question%7
This question analyses the two-dimensional harmonic oscillator, with Hamiltonian \[
	H=\frac{P_{1}^{2}}{2 m}+\frac{P_{2}^{2}}{2 m}+\frac{m \omega_{1}^{2}}{2} X_{1}^{2}+\frac{m \omega_{2}^{2}}{2} X_{2}^{2}.
\] Throughout this problem you do not have to normalize energy eigenstates.
\begin{parts}
\part%7a
Assuming $1 \leq \frac{\omega_{2}}{\omega_{1}}<\frac{3}{2}$ write down the first six energy eigenvalues and eigenstate wave functions of the system. Determine the degeneracy of energy levels when $\omega_{2}=\omega_{1}$.

\part%7b
For a one-dimensional harmonic oscillator the raising and lowering operators are defined as \[
	a_{ \pm}=c_{ \pm} X+\mathrm{i} d_{ \pm} P
\] where $c_{ \pm}$ and $d_{ \pm}$ are real parameters. These operators satisfy the following properties: \[
	[a_{-}, a_{+}]=\mathbbm{1}, \qquad a_{-}^{*}=a_{+}, \qquad H=\hbar \omega(a_{+} a_{-}+\lambda \mathbbm{1})
\] where $\lambda$ is a real number. Determine the parameters $c_{ \pm}, d_{ \pm}, \lambda$.

\part%7c
Introduce two sets of raising and lowering operators $a_{j, \pm}$ for the two-dimensional harmonic oscillator by \[
	a_{j, \pm}=c_{j, \pm} X_{j}+\mathrm{i} d_{j, \pm} P_{j}, \qquad(j=1,2)
\] with $c_{j, \pm}, d_{j, \pm}$ taken from part (b) with the replacement $\omega \rightarrow \omega_{j}$. Compute $\left[a_{j,-}, a_{k,+}\right]$. Express the six energy eigenstates previously found in (a) in terms of the operators $a_{j, \pm}$ acting an appropriate number of times on the vacuum state $\psi_{0,0}$.

\part%7d
Consider the following observable \[
	L_{3}=X_{1} P_{2}-X_{2} P_{1}.
\] Determine $\left[L_{3}, H\right]$ in the special case when $\omega_{2}=\omega_{1}$.

\part%7e
Based on your result above for $\left[L_{3}, H\right]$, in this part you will write down a new basis for the first three energy eigenstates (out of the six considered so far) when $\omega_{2}=\omega_{1}$. To do so, follow the steps below:
\begin{subparts}
\subpart%7ei
Express $L_{3}$ in terms of $a_{j, \pm}$.

\subpart%7eii
Compute the action of $L_{3}$ on the first three energy eigenstates. Based on your results, propose a new basis consisting of simultaneous eigenstates of $H$ and $L_{3}$.

\subpart%7eiii
Write down the wave functions of this new basis in Cartesian coordinates. Convert these formulas to polar coordinates.
\end{subparts}
\end{parts}

\end{questions}

\end{document}
