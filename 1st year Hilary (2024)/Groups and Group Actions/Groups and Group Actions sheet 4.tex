\documentclass[answers]{exam}
\usepackage{../HT2024}

\title{Groups and Group Actions -- Sheet 4\\Modular arithmetic, Cosets, Lagrange's theorem}
\author{YOUR NAME HERE :)}
\date{Hilary Term 2024}
% Accurate as of 05/07/2024

\def\hcf{\operatorname{hcf}}
\def\lcm{\operatorname{lcm}}


\begin{document}
\maketitle
\section*{Main course}
\begin{questions}

\question%1
\begin{parts}
\part%1a
Let $x, n$ be integers with $n \geqslant 2$ and $n$ not dividing $x$. Show that the order $o(\bar{x})$ of $\bar{x} \in \mathbb{Z}_{n}$ is \[
	o(\bar{x})=\frac{n}{\hcf(x, n)}.
\]

\part%1b
Let $G, H$ be finite groups with $g \in G$ and $h \in H$. Show that the order of $(g, h)$ in $G \times H$ is given by \[
	o((g, h))=\lcm\{o(g), o(h)\}.
\]
\end{parts}



\question%2
$\bar{x} \in \mathbb{Z}_{n}$ is said to be a unit if there exists $\bar{y} \in \mathbb{Z}_{n}$ such that $\bar{x} \bar{y}=\overline{1}\pmod n$.
\begin{parts}
\part%2a
Show that the units of $\mathbb{Z}_{n}$ form a group under multiplication. We denote this group $\mathbb{Z}_{n}^{*}$.

\part%2b
Use Bézout's Lemma to show that $\bar{x}$ is a unit of $\mathbb{Z}_{n}$ if and only if $\hcf(x, n)=1$.

\part%2c
List the units in $\mathbb{Z}_{9}$ and write out the Cayley table for $\mathbb{Z}_{9}^{*}$.

\part%2d
Show that $\mathbb{Z}_{9}^{*}$ is cyclic. What are the generators of $\mathbb{Z}_{9}^{*}$?
\end{parts}



\question%3
\begin{parts}
\part%3a
Use Fermat's Little Theorem to compute $5^{15}\pmod7$ and $7^{13}\pmod{11}$.

\part%3b
Use the Fermat-Euler Theorem to compute $4^{43}\pmod{15}$ and $2^{51}\pmod{21}$.

\part%3c
Show that $5^{14}=10\pmod{15}$. [You might try to find $5^{14}$ modulo 3 and modulo 5 first.]
\end{parts}



\question%4
Let $p$ be a prime and let $g, h$ be elements, both of order $p$, in a group $G$.
\begin{parts}
\part%4a
What are the possible orders of $\langle g\rangle \cap\langle h\rangle$?

\part%4b
Show that if $G$ is finite then the number of elements of order $p$ in $G$ is a multiple of $p-1$.

\part%4c
Deduce that a group of order 35 contains an element of order 5 and an element of order 7.
\end{parts}



\question%5
Suppose that every element $x$ in a group $G$ satisfies $x^{2}=e$.
\begin{parts}
\part%5a
Prove that $G$ is Abelian.

\part%5b
Show also that if $H$ is any subgroup of $G$ and $g \in G \setminus H$ then $K=H \cup g H$ is a subgroup of $G$.

\part%5c
Show further that $K$ is isomorphic to $H \times C_{2}$.

\part%5d
Deduce that if $G$ is finite then $G$ is isomorphic to $(\mathbb{Z}_{2})^{n}$ for some non-negative integer $n$.
\end{parts}



\question%6
Let $G_{1}$ and $G_{2}$ be finite groups and let $K \leqslant G_{1} \times G_{2}$.
\begin{parts}
\part%6a
Set $H_{1}=\left\{g \in G_{1}:(g, e) \in K\right\}$ and $H_{2}=\left\{g \in G_{2}:(e, g) \in K\right\}$. Show that \[
	H_{1} \leqslant G_{1} ; \qquad H_{2} \leqslant G_{2} ; \qquad H_{1} \times H_{2} \leqslant K .
\]

\part%6b
Suppose that $\left|G_{1}\right|$ and $\left|G_{2}\right|$ are coprime. Show that $K=H_{1} \times H_{2}$.

\part%6c
Show that this result need not follow if $\left|G_{1}\right|$ and $\left|G_{2}\right|$ are not coprime.
\end{parts}

\end{questions}



\section*{Starter}
\begin{questions}
\question%S1
In this question we work in $\mathbb{Z}_{8}$. For each $a \in \mathbb{Z}_{8}$, find $a^{7}$. How does this relate to Fermat's Little Theorem and to the Fermat-Euler Theorem?



\question%S2
Consider the dihedral group $D_{8}=\left\{e, r, r^{2}, r^{3}, s, r s, r^{2} s, r^{3} s\right\}$ with the notation from lectures. Find all the left cosets of $\langle r\rangle$ in $D_{8}$. Find all the right cosets of $\langle r\rangle$. How do these lists compare? Now repeat for the subgroup $\langle s\rangle$.



\question%S3
For each of the following, give a proof or a counterexample.
\begin{parts}
\part%S3a
A group with order 20 cannot have a subgroup of order 10.

\part%S3b
A group with order 22 cannot have a subgroup of order 10.

\part%S3c
A group with order 10 cannot have a subgroup of order 22.

\part%S3d
A group with order 10 must have a subgroup of order 10.

\part%S3e
A group with order 12 must have a subgroup of order 6.
\end{parts}

\end{questions}



\section*{Pudding}
\begin{questions}
\question%P1
Let $F=2^{32}+1$. Let $p$ be a prime dividing $F$. What is the order of 2 in $\mathbb{Z}_{p}^{*}$? Deduce that $p \equiv 1\pmod{64}$. Use this to show that $F$ is not prime.



\question%P2
We say that $n \geqslant 2$ is a Carmichael number if $n$ is not prime and $a^{n-1} \equiv 1\pmod n$ for all $a$ coprime to $n$. Show that if $n=(6 k+1)(12 k+1)(18 k+1)$ where $k$ is a positive integer such that $6 k+1,12 k+1$ and $18 k+1$ are all prime, then $n$ is a Carmichael number. Use this construction to find two Carmichael numbers.



\question%P3
Let $G$ be a group of order $n$ with a subgroup $H$ of order $n-1$. What can you say about $n$?

\end{questions}

\end{document}
