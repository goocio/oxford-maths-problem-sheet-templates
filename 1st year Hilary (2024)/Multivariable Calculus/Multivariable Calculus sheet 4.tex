\documentclass[answers]{exam}
\usepackage{../HT2024}

\title{Multivariable Calculus -- Sheet 4\\Div, Grad, Curl}
\author{YOUR NAME HERE :)}
\date{Hilary Term 2024}
% Accurate as of 05/07/2024


\begin{document}
\maketitle
\begin{questions}

\question%1
Show directly that $\nabla^{2}(\phi \psi)=\phi \nabla^{2} \psi+2 \nabla \phi \cdot \nabla \psi+\psi \nabla^{2} \phi$ for scalar fields $\phi$ and $\psi$.



\question%2
\begin{parts}
\part%2a
Let $\phi(x, y, z)=y^{2}-x z$ and $\mathbf{f}(x, y, z)=\left(z^{2}, x^{2}, y^{2}\right)$. Find $\nabla \phi$ and $\nabla \cdot \mathbf{f}$.

\part%2b
For the orthonormal basis $\mathbf{e}_{1}=(0,-1,0), \mathbf{e}_{2}=(1,0,-1) / \sqrt{2}, \mathbf{e}_{3}=(1,0,1) / \sqrt{2}$, create new co-ordinates $X, Y, Z$ such that \[
	X \mathbf{e}_{1}+Y \mathbf{e}_{2}+Z \mathbf{e}_{3}=x \mathbf{i}+y \mathbf{j}+z \mathbf{k}
\] Determine $x, y, z$ in terms of $X, Y, Z$.

\part%2c
Find $\Phi, F_{1}, F_{2}, F_{3}$ such that $\Phi(X, Y, Z)=\phi(x, y, z)$ and $F_{1} \mathbf{e}_{1}+F_{2} \mathbf{e}_{2}+F_{3} \mathbf{e}_{3}=f_{1} \mathbf{i}+f_{2} \mathbf{j}+f_{3} \mathbf{k}$. Verify, by direct calculation, that \[
	\nabla \phi=\Phi_{X} \mathbf{e}_{1}+\Phi_{Y} \mathbf{e}_{2}+\Phi_{Z} \mathbf{e}_{3} ; \qquad \nabla \cdot \mathbf{f}=\left(F_{1}\right)_{X}+\left(F_{2}\right)_{Y}+\left(F_{3}\right)_{Z}.
\]
\end{parts}



\question%3
Let $r$ and $\theta$ denote plane polar co-ordinates and set $\mathbf{e}_{r}=(\cos \theta, \sin \theta, 0)$ and $\mathbf{e}_{\theta}=(-\sin \theta, \cos \theta, 0)$. Let $\mathbf{F}(r, \theta)=F_{r} \mathbf{e}_{r}+F_{\theta} \mathbf{e}_{\theta}$ be a vector field. Prove that \[
	\nabla \cdot \mathbf{F}=\frac{1}{r} \frac{\partial}{\partial r}\left(r F_{r}\right)+\frac{1}{r} \frac{\partial F_{\theta}}{\partial \theta}.
\]



\question%4
Let $\mathbf{f}(x, y, z)=\left(y /\left(x^{2}+y^{2}\right),-x /\left(x^{2}+y^{2}\right), 0\right)$ where $(x, y) \neq(0,0)$.
\begin{parts}
\part%4a
Show that $\operatorname{curl} \mathbf{f}=\mathbf{0}$.

\part%4b
Find $\int_{C} \mathbf{f} \cdot \mathrm{d} \mathbf{r}$ for each of the following closed curves $C$.
\begin{subparts}
\subpart%4bi
$C$ is parametrised by $\mathbf{r}(t)=(\cos t, \sin t, 0)$ for $0 \leqslant t \leqslant 2 \pi$.

\subpart%4bii
$C$ is parametrised by \[
	\mathbf{r}(t)=\begin{cases}
		(\cos t, \sin t, t) & 0 \leqslant t \leqslant 4 \pi, \\
		(1,0,8 \pi-t) & 4 \pi \leqslant t \leqslant 8 \pi.
	\end{cases}
\]
\subpart%4biii
$C$ is the square with vertices $(0,1),(1,1),(1,2),(0,2)$ with an anticlockwise orientation.
\end{subparts}

\part%4c
Find a scalar field $\phi$ such that $\mathbf{f}=\nabla \phi$ on $R_{1}=\{(x, y, z): y>0\}$. How does the existence of $\phi$ relate to your answer to 4(b)(iii)?

\part%4d
Show that there does not exist $\psi$ such that $\mathbf{f}=\nabla \psi$ on $R_{2}=\{(x, y, z):(x, y) \neq(0,0)\}$.
\end{parts}



\question%5
\begin{parts}
\part%5a
With $\mathbf{f}$ as in Exercise 4, show that $\operatorname{div} \mathbf{f}=0$.

\part%5b
Suppose that a particle $(x(t), y(t))$ moves according to the flow \[
	\mathrm{d} x / \mathrm{d} t=y /\left(x^{2}+y^{2}\right), \quad \mathrm{d} y / \mathrm{d} t=-x /\left(x^{2}+y^{2}\right) .
\] Show that, on changing to polar co-ordinates $(r, \theta)$ these differential equations become \[
	\mathrm{d} r / \mathrm{d} t=0, \quad \mathrm{~d} \theta / \mathrm{d} t=-1 / r^{2}.
\]

\part%5c
Suppose that particles initially occupy the region $R_{0}=\{(r, \theta): 0<a<r<b, 0<\theta<\alpha<\pi / 2\}$. If the particles move according to the above flow, describe the region $R_{t}$ which they occupy a short time $t$ afterwards. Sketch $R_{0}$ and $R_{t}$, and show that the regions have the same area.
\end{parts}



\question%6
(Optional)
\begin{parts}
\part%6a
In spherical polar co-ordinates \[
	\mathbf{r}(r, \theta, \phi)=(r \sin \theta \cos \phi, r \sin \theta \sin \phi, r \cos \theta),
\] show that we can write \[
	\mathrm{d} \mathbf{r}=h_{r} \mathrm{~d} r \mathbf{e}_{r}+h_{\theta} \mathrm{d} \theta \mathbf{e}_{\theta}+h_{\phi} \mathrm{d} \phi \mathbf{e}_{\phi}
\] where $\mathbf{e}_{r}, \mathbf{e}_{\theta}, \mathbf{e}_{\phi}$ are a right-handed orthonormal basis and $h_{r}, h_{\theta}, h_{\phi}>0$. Show that $h_{r} h_{\theta} h_{\phi}=$ $r^{2} \sin \theta=\partial(x, y, z) / \partial(r, \theta, \phi)$.

\part%6b
More generally $\mathbf{r}\left(u_{1}, u_{2}, u_{3}\right)$ is a parametrization of space by orthogonal curvilinear co-ordinates if \[
	\mathrm{d} \mathbf{r}=h_{1} \mathrm{~d} u_{1} \mathbf{e}_{1}+h_{2} \mathrm{~d} u_{2} \mathbf{e}_{2}+h_{3} \mathrm{~d} u_{3} \mathbf{e}_{3}
\] where $\mathbf{e}_{1}, \mathbf{e}_{2}, \mathbf{e}_{3}$ are a right-handed orthonormal basis. Show in this case that \[
	\frac{\partial(x, y, z)}{\partial\left(u_{1}, u_{2}, u_{3}\right)}=h_{1} h_{2} h_{3}.
\]

\part%6c
Show further for a scalar field $\Phi\left(u_{1}, u_{2}, u_{3}\right)$ that \[
	\nabla \Phi=\frac{1}{h_{1}} \frac{\partial \Phi}{\partial u_{1}} \mathbf{e}_{1}+\frac{1}{h_{2}} \frac{\partial \Phi}{\partial u_{2}} \mathbf{e}_{2}+\frac{1}{h_{3}} \frac{\partial \Phi}{\partial u_{3}} \mathbf{e}_{3} .
\]
\end{parts}

\end{questions}

\end{document}
