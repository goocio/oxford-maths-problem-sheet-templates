\documentclass[answers]{exam}
\usepackage{../HT2024}

\title{Multivariable Calculus -- Sheet 3\\Surface integrals, Flux integrals, Solid angle}
\author{YOUR NAME HERE :)}
\date{Hilary Term 2024}
% Accurate as of 05/07/2024


\begin{document}
\maketitle
\begin{questions}

\question%1
Parameterize the surface of a cone, with base radius $a$ and height $h$, and so express its surface area as a multiple integral. Hence show its surface area equals \[
	\pi a \sqrt{a^{2}+h^{2}}.
\] Re-evaluate this surface area by considering how a sector of a disc may be folded to make a cone.



\question%2
Show that the solid angle at the apex of a cone with semiangle $\alpha$ is $2 \pi(1-\cos \alpha)$. If a sphere has radius $R$ and its centre at distance $D$ from an observer, with $D \gg R$, show that the sphere occupies, as a fraction \[
	\frac{1}{2}\left(1-\frac{\sqrt{D^{2}-R^{2}}}{D}\right) \approx \frac{R^{2}}{4 D^{2}}
\] of the observer's view. Use this to explain how the sun (at radius $7 \times 10^{5} \mathrm{~km}$ and distance $1.5 \times 10^{8} \mathrm{~km}$) and moon (at radius $1.8 \times 10^{3} \mathrm{~km}$ and distance $3.8 \times 10^{5} \mathrm{~km}$) occupy roughly the same amount of the sky.



\question%3
Evaluate \[
	\iint\limits_{\Sigma} \mathbf{F} \cdot \mathrm{d} \mathbf{S}
\] where $\mathbf{F}=\left((x-1) x^{2} y,(y-1)^{2} x y, z^{2}-1\right)$ and $\Sigma$ is the surface of the unit cube $[0,1]^{3}$.



\question%4
Two points are chosen at random on the surface of the sphere $\Sigma$ with equation $x^{2}+y^{2}+z^{2}=a^{2}$. Explain why the mean distance $\mu$ between the points equals the integral \[
	\mu=\frac{1}{4 \pi a^{2}} \iint_{\Sigma} \sqrt{x^{2}+y^{2}+(z-a)^{2}} \mathrm{~d} S
\] and hence determine $\mu$.



\question%5
\begin{parts}
\part%5a
Calculate the surface integrals $\iint_{\Sigma} f \mathrm{~d} S$ and $\iint_{\Sigma} f \mathrm{~d} \mathbf{S}$ where \[
	f(x, y, z)=\left(x^{2}+y^{2}+z^{2}\right)^{2}
\] and \[
	\Sigma=\left\{(x, y, z) \in \mathbb{R}^{3}: x^{2}+y^{2}=z^{2}, y \geqslant 0,0 \leqslant z \leqslant 2\right\}.
\]

\part%5b
Parametrise the various parts of the boundary $\partial \Sigma$ and determine $\int_{\partial \Sigma} f \mathrm{~d} s$ and $\int_{\partial \Sigma} f \mathrm{~d} \mathbf{r}$.
\end{parts}



\question%6
(Optional) A spherical shell $\Sigma$ with equation $x^{2}+y^{2}+z^{2}=1$ has density $\rho(x, y, z) \geqslant 0$. Show that its moment of inertia about an axis through the points $( \pm a, \pm b, \pm c)$ on the shell equals \[
	I(a, b, c)=\iint\limits_{\Sigma}\left(1-(a x+b y+c z)^{2}\right) \rho(x, y, z) \mathrm{d} S.
\] Find this value when $\rho$ is constant. Conversely, if $I(a, b, c)$ is constant for all $(a, b, c) \in \Sigma$, need $\rho(x, y, z)$ be constant?

\end{questions}

\end{document}
