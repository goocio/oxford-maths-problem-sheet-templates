\documentclass[answers]{exam}
\usepackage{../HT2024}

\title{Multivariable Calculus -- Sheet 8\\Gravity, Gauss' flux theorem, Poisson's equation}
\author{YOUR NAME HERE :)}
\date{Hilary Term 2024}
% Accurate as of 05/07/2024


\begin{document}
\maketitle
\begin{questions}

\question%1
\begin{parts}
\part%1a
Two planets, modelled as particles of equal mass $M$, are situated at the points $(a, 0,0)$ and $(-a, 0,0)$ and remain there. Show that the gravitational potential at the point $(x, 0,0)$, where $-a<x<a$, is \[
	\frac{2 G M a}{a^{2}-x^{2}}
\] and find the gravitational field at $(x, 0,0)$.

\part%1b
A space-craft moves along the segment $-a<x<a$ of the $x$-axis under the gravitational attraction of the planets. Write down the equation of motion for the space-craft and show that $x=0$ is a point of unstable equilibrium.

\part%1c
At $x=0$ the space-craft fires its motor briefly so as to acquire a speed $u(>0)$ and heads towards $x=a$. By using energy considerations show that the time taken to reach $x=a$ is \[
	T=\int_{0}^{a} \sqrt{\frac{a^{2}-x^{2}}{u^{2}\left(a^{2}-x^{2}\right)+\frac{4 G M}{a} x^{2}}} \mathrm{~d} x.
\] (You are not required to evaluate the integral.)
\end{parts}



\question%2
A homogeneous straight wire of mass $M$ lies along the $x$-axis from $(-a, 0,0)$ to $(a, 0,0)$. Show that the gravitational potential at the point $(x, y, 0)$, where $y \neq 0$, is \[
	\frac{G M}{2 a}\left[\sinh ^{-1}\left(\frac{a-x}{|y|}\right)+\sinh ^{-1}\left(\frac{a+x}{|y|}\right)\right]
\] and that the $x$-component of the gravitational field is \[
	\frac{G M}{2 a}\left(\frac{1}{d_{1}}-\frac{1}{d_{2}}\right)
\] where $d_{1}$ and $d_{2}$ are the distances from $(x, y, 0)$ to the ends of the wire.



\question%3
Show that the gravitational field at the vertex of a homogeneous circular cone is \[
	\frac{12 G M}{a^{2}} \sin ^{2} \frac{\alpha}{2}
\] where $M$ is the mass of the cone, $a$ is the radius of the base and $\alpha$ is the semi-vertical angle.



\question%4
A hollow spherical shell has internal radius $a$ and external radius $b$, and is made of material of uniform density $\rho$. Find the gravitational field and the gravitational potential in the three regions $0<r<a, a<r<b, r>b$ by using (1) the Flux Theorem, and (2) Poisson's equation.



\question%5
\begin{parts}
\part%5a
A solid hemisphere of uniform density $\rho$ occupies the region \[
	x^{2}+y^{2}+z^{2} \leqslant a^{2}, \quad z \leqslant 0 .
\] Find the gravitational potential due to the hemisphere at the point $(0,0, s)$ where $s>0$.

\part%5b
A uniform rod of density $m$ per unit length lies on the $z$-axis between $(0,0, c)$ and $(0,0, d)$ where $d>c>0$. Show that the force exerted on the rod by the hemisphere is \[
	\psi(c)-\psi(d)
\] where \[
	\psi(\lambda)=\frac{2 \pi G m \rho}{3}\left(\frac{a^{3}+\lambda^{3}-\left(a^{2}+\lambda^{2}\right)^{3 / 2}}{\lambda}\right).
\]
\end{parts}

\end{questions}

\end{document}
