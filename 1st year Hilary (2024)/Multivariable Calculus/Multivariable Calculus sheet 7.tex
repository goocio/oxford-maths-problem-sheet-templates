\documentclass[answers]{exam}
\usepackage{../HT2024}

\title{Multivariable Calculus -- Sheet 7\\Stokes' theorem}
\author{YOUR NAME HERE :)}
\date{Hilary Term 2024}
% Accurate as of 05/07/2024


\begin{document}
\maketitle
\begin{questions}

\question%1
Let $0<a<b$. Verify Stokes' Theorem when $\mathbf{F}=(y, z, x)$ and $\Sigma$ is the upper half of the torus generated by rotating the circle $(x-b)^{2}+z^{2}=a^{2}$ about the $z$-axis.



\question%2
The vector field $\mathbf{F}(\mathbf{R})$ is defined by \[
	\mathbf{F}(\mathbf{R})=\int_{C}|\mathbf{r}-\mathbf{R}|^{2} \mathrm{~d} \mathbf{r}
\] where $\mathbf{r}$ lies on the simple closed curve $C$. Show that there are constant vectors $\mathbf{A}$ and $\mathbf{B}$ such that $\mathbf{F}(\mathbf{R})=\mathbf{R} \wedge \mathbf{A}+\mathbf{B}$. Deduce that \[
	\nabla \wedge \mathbf{F}=-4 \iint\limits_{S} \mathrm{~d} \mathbf{S}
\] where $S$ is any smooth surface spanning $C$.



\question%3
Let $\Sigma$ denote that part of the cone $x^{2}+y^{2}=z^{2}, z>0$ which lies beneath the plane $x+2 z=1$. Let $\mathbf{F}(x, y, z)=x \mathbf{j}$.
\begin{parts}
\part%3a
Show that the projection of $\partial \Sigma$ vertically to the $x y$-plane is an ellipse. Parametrise $\partial \Sigma$ and determine $\int_{\partial \Sigma} \mathbf{F} \cdot \mathrm{d} \mathbf{r}$.

\part%3b
Show that $\mathrm{d} \mathbf{S} \cdot \mathbf{k}=\mathrm{d} x \mathrm{~d} y$ on $\Sigma$ and verify Stokes' Theorem for $\mathbf{F}$ on $\Sigma$.
\end{parts}



\question%4 this question has very nice links to complex analysis, where you'll see that this question's "curl F = 0" and "div F = 0" are effectively encoding the *Cauchy-Riemann equations*, a pair of equations governing complex derivatives. Moreover as in this first part you can use the Cauchy-Riemann equations to show that the real part (and the imaginary part) of any complex differentiable function will satisfy the Laplace equation (i.e. is harmonic), and conversely that if you have a harmonic function u(x,y) then there is a unique *harmonic conjugate* v(x,y) such that f(z)=f(x+iy)=u(x,y)+iv(x,y) is complex differentiable (in this question G is analogous to f(z)).
Let $\mathbf{F}(x, y)=(u(x, y), v(x, y))$ be defined on $\mathbb{R}^{2}$ where $u, v$ have continuous partial derivatives of all orders.
\begin{parts}
\part%4a
Under what condition on $u$ and $v$ is $\operatorname{curl} \mathbf{F}=\mathbf{0}$? Under what condition is $\operatorname{div} \mathbf{F}=0$? Show that if both these conditions hold then $u$ and $v$ are harmonic -- that is, they satisfy Laplace's equation.

\part%4b
Conversely say $U$ is a harmonic function. Use Green's Theorem to explain why \[
	\mathbf{G}(X, Y)=\left(U(X, Y), \int_{C} U_{y} \mathrm{~d} x-U_{x} \mathrm{~d} y\right)
\] is a well-defined function, independent of the choice of curve $C$ from $(0,0)$ to $(X, Y)$. Show that $\operatorname{div} \mathbf{G}=0$ and $\operatorname{curl} \mathbf{G}=\mathbf{0}$.

\part%4c
What is $\mathbf{G}(x, y)$ if (i) $U(x, y)=x^{2}-y^{2}$? (ii) $U(x, y)=e^{x} \cos y$?
\end{parts}



\question%5
(Optional) Let $\mathbf{F}=\mathbf{r} / r^{3}=\mathbf{e}_{r} / r^{2}$, in terms of spherical polar co-ordinates, and let \[
	R_{1}=\mathbb{R}^{3} \backslash\{\mathbf{0}\} \quad \text { and } \quad R_{2}=\left\{(x, y, z) \in \mathbb{R}^{3}: x^{2}+y^{2} \neq 0\right\}.
\]
\begin{parts}
\part%5a
Show that $\operatorname{div} \mathbf{F}=0$.

\part%5b
Show that \[
	\mathbf{f}=-\frac{\cot \theta}{r} e_{\phi}
\] is a vector potential for $\mathbf{F}$ on $R_{2}$ -- that is, show that $\mathbf{F}=\nabla \wedge \mathbf{f}$.

\part%5c
Why is $\mathbf{f}$ not a vector potential for $\mathbf{F}$ on $R_{1}$?
\end{parts}

\end{questions}

\end{document}
