\documentclass[answers]{exam}
\usepackage{../HT2024}

\title{Multivariable Calculus -- Sheet 6\\Divergence theorem}
\author{YOUR NAME HERE :)}
\date{Hilary Term 2024}
% Accurate as of 05/07/2024


\begin{document}
\maketitle
\begin{questions}

\question%1
Let $C$ be a closed, positively oriented curve in $\mathbb{R}^{2}$ bounding a region $D$. Show that \[
	\text { area of } D=\frac{1}{2} \int_{C} x \mathrm{~d} y-y \mathrm{~d} x.
\] Hence find the area of the ellipse $x^{2} / a^{2}+y^{2} / b^{2}=1$.



\question%2
Let $D \subseteq \mathbb{R}^{2}$ be a closed, boundary region with smooth boundary $\partial D$, and $f$ be a smooth function defined in $D$. By applying Green's theorem in the plane with suitable functions $P$ and $Q$, show that \[
	\iint\limits_{D} \nabla^{2} f \mathrm{~d} x \mathrm{~d} y=\int \frac{\partial f}{\partial n} \mathrm{~d} s.
\]



\question%3
Let $R$ be the region $1<a<r<b$, where $r$ is the distance from the origin in $\mathbb{R}^{2}$. Find a solution of the boundary-value problem \[
	\nabla^{2} f+1=0 \text { in } R, \quad \frac{\partial f}{\partial n}+f=0 \text { on } \partial R,
\] which is a function of $r$ only. Show that this is the only solution, even within the class of not necessarily radial functions.



\question%4
The temperature $T(r, \theta)$ in an annulus $a \leqslant r \leqslant b$ satisfies $\nabla^{2} T=1$ inside the annulus. On the inner boundary $\partial T / \partial n=k$, where $k>0$ and the outer boundary is insulated.
\begin{parts}
\part%4a
Use Exercise 2 to show the uniqueness, up to a constant, of any solution to this boundary value problem.

\part%4b
Find all circularly symmetric solutions $T(r)$ to \[
	\nabla^{2} T=\frac{\mathrm{d}^{2} T}{\mathrm{d} r^{2}}+\frac{1}{r} \frac{\mathrm{d} T}{\mathrm{d} r}=1
\] in the annulus.

\part%4c
For what value of $k$ is there a circularly symmetric solution to this boundary value problem? Interpret this value physically.
\end{parts}



\question%5
Let $R$ be the region $x^{2} / a^{2}+y^{2} / b^{2}+z^{2} / c^{2} \leqslant 1$ with boundary $\partial R$ and $a, b, c>0$. Suppose $u(x, y, z)$ satisfies $\nabla^{2} u=-1$ in $R$ and $u=0$ on $\partial R$.
\begin{parts}
\part%5a
Show that the solution $u$ is unique.

\part%5b
Show that the solution $u$ is a quadratic function of $x, y, z$ and evaluate \[
	\iint\limits_{\partial R} \nabla u \cdot \mathrm{d} \mathbf{S}
\]
\end{parts}



\question%6
(Optional) Differentiation under the integral sign relates to the theorem that \[
	\frac{\mathrm{d}}{\mathrm{d} t} \int_{I} f(x, t) \mathrm{~d} x=\int_{I} \frac{\partial f}{\partial t}(x, t) \mathrm{~d} x
\] which holds, under quite general hypotheses, for a function $f(x, t)$ and an interval $I \subseteq \mathbb{R}$.
\begin{parts}
\part%6a
By differentiating with respect to $a$, reproduce a solution to Sheet 1, Exercise 1.

\part%6b
Let $a \in \mathbb{R}$. Determine and solve a differential equation involving \[
	I(a)=\int_{-\infty}^{\infty} e^{-x^{2}} \cos 2 a x \mathrm{~d} x
\] and hence show that $I(a)=\sqrt{\pi} e^{-a^{2}}$.

\part%6c
A compressible fluid of density $\rho(x, t)$ moves with velocity $u(x, t)$ in and out of an interval $I=[\alpha, \beta]$. Explain why \[
	\frac{\mathrm{d}}{\mathrm{d} t} \int_{\alpha}^{\beta} \rho(x, t) \mathrm{~d} x=\rho(\alpha, t) u(\alpha, t)-\rho(\beta, t) u(\beta, t)
\] interpreting each term physically. Hence derive the continuity equation (Sheet 5, Exercise 6).
\end{parts}

\end{questions}

\end{document}
