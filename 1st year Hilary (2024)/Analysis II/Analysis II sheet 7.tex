\documentclass[answers]{exam}
\usepackage{../HT2024}

\title{Analysis II -- Sheet 7\\Taylor's theorem}
\author{YOUR NAME HERE :)}
\date{Hilary Term 2024}
% Accurate as of 05/07/2024


\begin{document}
\maketitle
\begin{questions}

\question%1
Suppose that the real-valued function $f$ is such that the $n^{\text {th }}$ derivative of $f$ exists and is continuous on $[0, h]$ (where $h>0$) and the $n+1^{\text {st }}$ derivative exists on $(0, h)$. Consider the function $G:[0, h] \to \mathbb{R}$ defined by \[
	G(t)=F(t)-\left(\frac{h-t}{h}\right)^{p} F(0),
\] where $F:[0, h] \to \mathbb{R}$ is given by \[
	F(t)=f(h)-f(t)-(h-t) f'(t)-\cdots-\frac{(h-t)^{n}}{n !} f^{n}(t)
\] and $p$ is a constant. By considering the derivative of $G$ and choosing $p$ appropriately, prove that there exist $\theta \in(0,1)$ such that \[
	f(h)=f(0)+h f'(0)+\cdots+\frac{h^{n}}{n !} f^{(n)}(0)+\frac{h^{n+1}}{n!}(1-\theta)^{n} f^{(n+1)}(\theta h).
\]



\question%2
Use the version of Taylor's theorem given in question 1 to show that $\log(1-x)=-\sum_{n=1}^\infty\frac{x^n}n$ for all $x\in[0,1)$.



\question%3
Assume that $f$ is twice differentiable and $|f(x)| \leq 1$ and $|f''(x)| \leq 1$ for all $x \in[0,2]$. By using the Taylor expansions of $f(0)$ and $f(2)$ about the point $x \in[0,2]$, prove that for all $x \in[0,2]$ we have $|f'(x)| \leq 2$.



\question%4
Suppose that $f$ is twice differentiable on $[a, b]$ and $f'(a)=f'(b)=0$. Show that there exists $\xi \in(a, b)$ such that \[
	\left|f''(\xi)\right| \geq \frac{4}{(b-a)^{2}}|f(b)-f(a)|
\] [\emph{Hint: use the triangle inequality $|f(b)-f(a)| \leq\left|f(b)-f\left(\frac{a+b}{2}\right)\right|+\left|f(a)-f\left(\frac{a+b}{2}\right)\right|$ and apply Taylor's formula to $f$ at $a$ and $b$.}]



\question%5
\begin{parts}
\part%5a
Show that for all $x \in \mathbb{R}$, \[
	\left|\frac{d^{n}}{d x^{n}} \frac{1}{1+x^{2}}\right| \leq n !.
\] [\emph{Hint: write $\frac{1}{1+x^{2}}=\frac{A}{1+i x}+\frac{B}{1-i x}$.}]

\part%5b
Prove that for all $-1 \leq x \leq 1$, \[
	\arctan x=x-\frac{x^{3}}{3}+\frac{x^{5}}{5}-\cdots+(-1)^{n} \frac{x^{2 n+1}}{2 n+1}+\cdots.
\]

\part%5c
Deduce that $\frac{\pi}{4}=1-\frac{1}{3}+\frac{1}{5}-\frac{1}{7}+\cdots$.
\end{parts}



\section*{Bonus Questions (optional, for the enthusiasts)}

\question%6
Construct an infinitely differentiable function on $\mathbb{R}$ that is 1 on $[-1,1]$ but 0 outside of $[-2,2]$. [\emph{You may assume that the function defined by $f(x)=e^{-1 / x}$ for $x>0$ and 0 for $x \leq 0$ is infinitely differentiable.}]% only tangentially related but watch freya holmer's video "the continuity of splines" it's so cool



\question%7
(Hard) Suppose $f: \mathbb{R} \to \mathbb{R}$ is infinitely differentiable and $f^{(n)}(x) \geq 0$ for all $x \in \mathbb{R}$. Show that the infinite Taylor expansion of $f$ converges to $f(x)$ for all $x \in \mathbb{R}$.



\question%8
(Hard) Suppose the power series $f(x)=\sum_{n=0}^{\infty} a_{n} x^{n}$ has radius of convergence $R \in(0, \infty]$ and assume $\left|x_{0}\right|<R$. Show that $g(x)=f\left(x_{0}+x\right)$ can be expressed as a power series about $x=0$ with strictly positive radius of convergence.

\end{questions}

\end{document}
