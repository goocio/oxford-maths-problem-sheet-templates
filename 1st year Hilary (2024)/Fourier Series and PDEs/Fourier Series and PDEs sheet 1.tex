\documentclass[answers]{exam}
\usepackage{../HT2024}

\title{Fourier Series and PDEs -- Sheet 1}
\author{YOUR NAME HERE :)}
\date{Hilary Term 2024}
% Accurate as of 05/07/2024


\begin{document}
\maketitle
\begin{questions}

\question%1
\begin{parts}
\part%1a
Consider the boundary value problem for $y(x)$ given by \[
	\frac{\mathrm{d}^{2} y}{\mathrm{~d} x^{2}}+y=0 \quad \text { for } \quad 0<x<L
\] with $y(0)=0$ and $y(L)=0$, where $L$ is a positive constant. The function $y(x)=0$ is a solution. Is this solution unique, or not? How does the answer depend on $L$?

\part%1b
Consider the initial value problem for $y(t)$ given by \[
	\frac{\mathrm{d} y}{\mathrm{~d} t}=2 y^{1 / 2} \quad \text { for } \quad t>0
\] with $y(0)=0$. The function $y(t)=0$ is a solution. Use separation of variables to find a nontrivial solution. Hence, write down a solution that vanishes if and only if $t \leq a$, where $a$ is a positive constant.
\end{parts}



\question%2
\begin{parts}
\part%2a
Use the power series representation of the function $1 /(1-z)$ with $z=R \mathrm{e}^{\mathrm{i} \theta}$, where $R \in[0,1)$ and $\theta \in \mathbb{R}$, to establish an example of a convergent Fourier sine series.

\part%2b
Assuming that the real constants $a_{1}, a_{2}, \ldots$ and $b_{1}, b_{2}, \ldots$ are such that the orders of summation and differentiation may be interchanged, show that \[
	y(x, t)=\sum_{n=1}^{\infty}\left(a_{n} \cos (n t)+b_{n} \sin (n t)\right) \sin (n x)
\] satisfies the wave equation \[
	\frac{\partial^{2} y}{\partial t^{2}}=\frac{\partial^{2} y}{\partial x^{2}} \quad \text { for } \quad 0<x<\pi, t>0
\] and the boundary conditions $y(0, t)=0$ and $y(\pi, t)=0$ for $t>0$. Hence, write down an infinite series solution for $y(x, t)$ that also satisfies the initial conditions \[
	y(x, 0)=\frac{R \sin x}{1-2 R \cos x+R^{2}}, \quad \frac{\partial y}{\partial t}(x, 0)=0 \quad \text { for } 0<x<\pi.
\]
\end{parts}



\question%3
\begin{parts}
\part%3a
Let $f$ be a periodic function of period $2 \pi$. By making the substitution $s=2 \pi+t$ in one of the integrals in the identity \[
	\int_{\alpha}^{\alpha+2 \pi} f(s) \mathrm{~d} s=\int_{\alpha}^{\pi} f(s) \mathrm{~d} s+\int_{\pi}^{\alpha+2 \pi} f(s) \mathrm{~d} s,
\] show that, for any real $\alpha$, \[
	\int_{\alpha}^{\alpha+2 \pi} f(s) \mathrm{~d} s=\int_{-\pi}^{\pi} f(s) \mathrm{~d} s.
\]

\part%3b
Let $g$ be an even function with derivative $g'$ and let \[
	G(x)=\int_{0}^{x} g(s) \mathrm{~d} s.
\]
\begin{subparts}
\subpart%3bi
Use the chain rule to show that $g'$ is an odd function.

\subpart%3bii
By making the substitution $s=-t$, show that $G$ is an odd function.

\subpart%3biii
Deduce that, for any real $\alpha$, \[
	\int_{-\alpha}^{\alpha} g(s) \mathrm{~d} s=2 \int_{0}^{\alpha} g(s) \mathrm{~d} s
\]
\end{subparts}

\part%3c
Show that if $h$ is an odd function then, for any real $\alpha$, \[
	\int_{-\alpha}^{\alpha} h(s) \mathrm{~d} s=0.
\]
\end{parts}



\question%4
Let $f:[0, \pi] \to \mathbb{R}$.
\begin{parts}
\part%4a
Let $f_{e}$ be a periodic function of period $2 \pi$ satisfying \[
	f_{e}(x)= \begin{cases}f(x) & \text { for } 0 \leqslant x \leqslant \pi; \\ f(-x) & \text { for }-\pi<x<0.\end{cases}
\] Show that $f_{e}$ is an even function.

\part%4b
Let $f_{o}$ be a periodic function of period $2 \pi$ satisfying \[
	f_{o}(x)= \begin{cases}f(x) & \text { for } 0 \leqslant x \leqslant \pi; \\ -f(-x) & \text { for }-\pi<x<0.\end{cases}
\] Under what conditions on $f$ is $f_{o}$ an odd function? Justify your answer.
\end{parts}

\end{questions}

\end{document}
