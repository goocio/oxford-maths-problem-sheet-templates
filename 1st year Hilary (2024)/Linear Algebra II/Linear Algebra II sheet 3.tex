\documentclass[answers]{exam}
\usepackage{../HT2024}

\title{Linear Algebra II -- Sheet 3\\Eigenvalues, Eigenvectors, Diagonalizability}
\author{YOUR NAME HERE :)}
\date{Hilary Term 2024}
% Accurate as of 05/07/2024


\begin{document}
\maketitle
\section*{Main course}
\begin{questions}

\question%1
Find the eigenvalues and eigenvectors of each of the following matrices $A_{i}$. In each case, determine whether or not the matrix is diagonalizable and, if so, find an invertible matrix $P$ such that $P^{-1} A_{i} P$ is diagonal.
\begin{parts}
\part%1a
\vspace{2em}\phantom{}\vspace{-3em}\[
	A_{1}=\begin{pmatrix}
		2 & 1 & 2 \\
		0 & 0 & 1 \\
		0 & 1 & 0
	\end{pmatrix};
\]

\part%1b
\vspace{2em}\phantom{}\vspace{-3em}\[
	A_{2}=\begin{pmatrix}
		1 & 1 & 0 \\
		-1 & 3 & 0 \\
		-1 & 4 & -1
	\end{pmatrix};
\]

\part%1c
\vspace{2em}\phantom{}\vspace{-3em}\[
	A_{3}=\begin{pmatrix}
		2 & 1 & 1 \\
		1 & 2 & 1 \\
		1 & 1 & 2
	\end{pmatrix}.
\]
\end{parts}



\question%2
\begin{parts}
\part%2a
Show that the matrix \[
	A=\frac{1}{12}\begin{pmatrix}
		0 & -8 \\
		3 & 10
	\end{pmatrix}
\] is diagonalizable.

\part%2b
Determine $A^{n}$ where $n$ is an integer.

\part%2c
Evaluate the infinite sum $S=I+A+A^{2}+\cdots$ and verify that $S=(I-A)^{-1}$.
\end{parts}



\question%3
Let \[
	A=\begin{pmatrix}
		0 & 1 & 0 & 0 \\
		0 & 0 & 1 & 0 \\
		0 & 0 & 0 & 1 \\
		1 & 0 & 0 & 0
	\end{pmatrix} \quad \text { and } \quad B=\begin{pmatrix}
		x & 1 & 0 & 1 \\
		1 & x & 1 & 0 \\
		0 & 1 & x & 1 \\
		1 & 0 & 1 & x
	\end{pmatrix}.
\]
\begin{parts}
\part%3a
Show that $A$ is not diagonalizable over $\mathbb{R}$ but is diagonalizable over $\mathbb{C}$.

\part%3b
By noting $B=A+A^{-1}+x I$, or otherwise, find the eigenvalues and the determinant of $B$.
\end{parts}



\question%4
\begin{parts}
\part%4a
Let $\mathbf{v}$ be a $\lambda$-eigenvector of an $n \times n$ matrix $A$. Show for any polynomial $p(x)$ that $p(A) \mathbf{v}=p(\lambda) \mathbf{v}$ and hence that $\chi_{A}(A) \mathbf{v}=\mathbf{0}$. Deduce that if $A$ is diagonalizable then $\chi_{A}(A)=0$.

\part%4b
Let $T$ be an upper triangular matrix with diagonal entries $\lambda_{1}, \lambda_{2}, \ldots, \lambda_{n}$. What is $\chi_{T}(x)$? Show that \[
	\left(T-\lambda_{1} I\right) \mathbf{e}_{1}^{T}=\mathbf{0}, \quad\left(T-\lambda_{2} I\right) \mathbf{e}_{2}^{T} \in\left\langle\mathbf{e}_{1}^{T}\right\rangle, \quad\left(T-\lambda_{1} I\right)\left(T-\lambda_{2} I\right) \mathbf{e}_{2}^{T}=\mathbf{0}.
\] Deduce that $\chi_{T}(T)=0$.
\end{parts}



\question%5
Let \[
	A=\begin{pmatrix}
		1 & 1 & 1 \\
		1 & 1 & 0 \\
		1 & 0 & 1
	\end{pmatrix}.
\] Which of the following fields is $A$ diagonalizable over? Justify your answers. In the cases where $A$ is diagonalizable over a finite field, enumerate how many invertible matrices $P$ there are such that $P^{-1} A P$ is diagonal.
\begin{parts}
\part%5a
$\mathbb{R}$,

\part%5b
$\mathbb{Q}$,

\part%5c
$\mathbb{Z}_{3}$,

\part%5d
$\mathbb{Z}_{7}$,

\part%5e
$\mathbb{Z}_{2}$.
\end{parts}

\end{questions}



\section*{Starter}
\begin{questions}

\question%S1
Let $A$ be a square matrix and $c$ a scalar. How do the eigenvalues of $A$ relate to those of $A+c I$? Show that $A$ is diagonalizable if and only if $A+c I$ is diagonalizable.



\question%S2
Find the eigenvalues of the matrix \[
	\begin{pmatrix}
		\sec \alpha \cos \beta & \sec \alpha \sin \beta+\tan \alpha \\
		\sec \alpha \sin \beta-\tan \alpha & -\sec \alpha \cos \beta
	\end{pmatrix}.
\]



\question%S3
Let $V$ denote the space of real $n \times n$ matrices and $S: V \to V$ be the linear map $S(A)=A^{T}$. Show that \[
	\langle A, B\rangle=\operatorname{trace}(B^{T} A)
\] defines an inner product on $V$ and that $\langle S(A), B\rangle=\langle A, S(B)\rangle$ for any $A, B \in V$.

\end{questions}



\section*{Pudding}
\begin{questions}

\question%P1
Let $A$ be an $n \times n$ matrix with $\chi_{A}(x)=x^{n}+c_{n-1} x^{n-1}+\cdots+c_{0}$. Show that \[
	c_{n-2}=\frac{1}{2}(\operatorname{trace}(A)^{2}-\operatorname{trace}(A^{2})).
\]



\question%P2
In lectures it was shown that the matrix \[
	M=\begin{pmatrix}
		5 & -3 & -5 \\
		2 & 9 & 4 \\
		-1 & 0 & 7
	\end{pmatrix}
\] has eigenvalues $9,6,6$ and is not diagonalizable. Find an invertible matrix $P$ such that \[
	P^{-1} M P=\begin{pmatrix}
		9 & 0 & 0 \\
		0 & 6 & 1 \\
		0 & 0 & 6
	\end{pmatrix}.
\]



\question%P3
For $\mathbf{v}=\left(v_{1}, \ldots, v_{n}\right)^{T} \in \mathbb{R}_{\text {col }}^{n}$ we define $\|\mathbf{v}\|=\sqrt{v_{1}^{2}+\cdots+v_{n}^{2}}$ and for an $n \times n$ matrix $A$ we define \[
	\|A\|=\max \{\|A \mathbf{v}\|:\|\mathbf{v}\|=1\}.
\] Prove the following properties where $A, B$ are $n \times n$ matrices, $\lambda \in \mathbb{R}$ and $\mathbf{v} \in \mathbb{R}_{\text {col }}^{n}$:
\begin{parts}
\part%P3a
$\|A\| \geqslant 0$ for all $A$ and $\|A\|=0$ if and only if $A=0$;

\part%P3b
$\|\lambda A\|=|\lambda|\|A\|$;

\part%P3c
$\|A+B\| \leqslant\|A\|+\|B\|$;

\part%P3d
$\|A \mathbf{v}\| \leqslant\|A\|\|\mathbf{v}\|$;

\part%P3e
$\|A B\| \leqslant\|A\|\|B\|$;

\part%P3f
Let $A, S$ be the matrices in question 2 and let $S_{n}=I+A+\cdots+A^{n}$. Show that \[
	\left\|S_{n}-S\right\| \to 0 \quad \text { as } n \to \infty.
\]
\end{parts}

\end{questions}

\end{document}
