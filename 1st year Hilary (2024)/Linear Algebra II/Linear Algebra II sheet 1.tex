\documentclass[answers]{exam}
\usepackage{../HT2024}

\title{Linear Algebra II -- Sheet 1\\Determinants}
\author{YOUR NAME HERE :)}
\date{Hilary Term 2024}
% Accurate as of 05/07/2024


\begin{document}
\maketitle
\section*{Main course}
\begin{questions}

\question%1
Verify directly that $\det A B=\det A \times \det B=\det B A$ where \[
	A=\begin{pmatrix}
		1 & 0 & 2 \\
		0 & 1 & 3 \\
		2 & 0 & 1
	\end{pmatrix} \quad \text { and } \quad B=\begin{pmatrix}
		0 & 0 & 1 \\
		0 & 1 & -1 \\
		2 & 0 & 3
	\end{pmatrix}.
\]



\question%2
Use EROs and ECOs to evaluate the following $4 \times 4$ determinants.
\begin{parts}
\part%2a
\vspace{2em}\phantom{}\vspace{-3.5em}\[
	\begin{vmatrix}
		1 & 2 & 3 & 1 \\
		-2 & 0 & 0 & 2 \\
		6 & 1 & 3 & 4 \\
		0 & 1 & 3 & -1
	\end{vmatrix},
\]

\part%2b
\vspace{2em}\phantom{}\vspace{-3.5em}\[
	\begin{vmatrix}
		1 & 2 & 4 & 2 \\
		2 & -3 & 1 & -3 \\
		7 & -1 & 0 & 1 \\
		1 & -2 & 1 & 0
	\end{vmatrix}.
\]
\end{parts}



\question%3
Show, for any real number $x$, that:
\begin{parts}
\part%3a
\[
	\begin{vmatrix}
		x^{2} & (x+1)^{2} & (x+2)^{2} \\
		(x+1)^{2} & (x+2)^{2} & (x+3)^{2} \\
		(x+2)^{2} & (x+3)^{2} & (x+4)^{2}
	\end{vmatrix}=-8;
\]

\part%3b
\[
	\begin{vmatrix}
		1 & 1 & 1 & 1 \\
		x & 1 & 1 & 1 \\
		x & x & 1 & 1 \\
		x & x & x & 1
	\end{vmatrix}=(1-x)^{3}.
\]
\end{parts}



\question%4
Let $A$ be the $(m+n) \times(m+n)$ matrix below where $U, V, W$ are respectively $m \times m$, $m \times n, n \times n$ matrices. \[
	A=\begin{pmatrix}
		U & V \\
		0 & W
	\end{pmatrix}.
\] Show that $\det A=\det U \times \det W$.



\question%5
(\emph{Vandermonde matrix}) For $n \geqslant 2$ and real numbers $x_{1}, \ldots, x_{n}$ we define \[
	V_{n}=\begin{pmatrix}
		1 & x_{1} & x_{1}^{2} & \cdots & x_{1}^{n-1} \\
		1 & x_{2} & x_{2}^{2} & \cdots & x_{2}^{n-1} \\
		1 & x_{3} & x_{3}^{2} & \cdots & x_{3}^{n-1} \\
		\vdots & \vdots & \vdots & \ddots & \vdots \\
		1 & x_{n} & x_{n}^{2} & \cdots & x_{n}^{n-1}
	\end{pmatrix}.
\]
\begin{parts}
\part%5a
Show that \[
	\det V_{n}=\prod_{i>j}\left(x_{i}-x_{j}\right).
\] [\emph{Hint: begin by adding $-x_{1} \times(k$th column $)$ to $(k+1)$th column for $1 \leqslant k<n$.}]

\part%5b
Deduce that $V_{n}$ is invertible if and only if the $x_{i}$ are distinct.

\part%5c
Let $x_{0}, x_{1}, \ldots, x_{n}, y_{0}, y_{1}, \ldots, y_{n}$ be real numbers with the $x_{i}$ distinct. Show that there is a unique polynomial $p(x)$ of degree $n$ or less such that $p\left(x_{i}\right)=y_{i}$ for each $0 \leqslant i \leqslant n$. Finding such $p(x)$ is called Lagrangian interpolation.

\part%5d
Determine $p(x)$ when $y_{0}=1$ and $y_{i}=0$ for $i \geqslant 1$.
\end{parts}

\end{questions}



\section*{Starter}
\begin{questions}

\question%S1
Give a counter-example to $\det(A+B)=\det A+\det B$.



\question%S2
Show that an orthogonal matrix has determinant 1 or $-1$. Is the converse true?



\question%S3
Prove, by induction, that the determinant of a triangular matrix is the product of its diagonal entries.

\end{questions}



\section*{Pudding}
\begin{questions}

\question%P1
In the case of the determinant \[
	\begin{vmatrix}
		1 & -3 & 2 \\
		0 & 7 & 1 \\
		-5 & 1 & 3
	\end{vmatrix}
\] verify that the following three expressions \[
	a_{21} C_{21}+a_{22} C_{22}+a_{23} C_{23}, \quad a_{12} C_{12}+a_{22} C_{22}+a_{32} C_{32}, \quad a_{13} C_{13}+a_{23} C_{23}+a_{33} C_{33},
\] all agree with the given determinant. Here $C_{i j}$ denotes the $(i, j)$th cofactor.



\question%P2
Let $A$ be a square matrix. Show that if $A$ has a real square root then $\det A \geqslant 0$. Is the converse true?



\question%P3
Define the $n \times n$ determinant $D_{n}$ by \[
	D_{n}=\begin{vmatrix}
		1 & -1 & 0 & 0 & \cdots & 0 \\
		1 & 1 & -1 & 0 & \cdots & 0 \\
		0 & 1 & 1 & -1 & \ddots & \vdots \\
		0 & 0 & 1 & \ddots & \ddots & 0 \\
		\vdots & \vdots & \ddots & \ddots & 1 & -1 \\
		0 & 0 & \cdots & 0 & 1 & 1
	\end{vmatrix}.
\] Show that $D_{1}=1$ and $D_{2}=2$ and that $D_{n+2}=D_{n+1}+D_{n}$ for $n \geqslant 1$. Deduce that $D_{n}$ is the $(n+1)$th Fibonacci number.

\end{questions}

\end{document}
