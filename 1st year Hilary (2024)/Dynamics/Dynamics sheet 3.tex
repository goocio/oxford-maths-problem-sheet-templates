\documentclass[answers]{exam}
\usepackage{../HT2024}

\title{Dynamics -- Sheet 3}
\author{YOUR NAME HERE :)}
\date{Hilary Term 2024}
% Accurate as of 05/07/2024


\begin{document}
\maketitle
\begin{questions}

\question%1
A bead of mass $m$ is attached to the end of a straight spring, where the spring has natural length $a$ and spring constant $k$. The other end of the spring is fixed at the origin $O$. The bead and spring hang directly below $O$, with the spring lying along the vertical line through $O$.
\begin{parts}
\part%1a
Explain why a potential energy function for the bead is \[
	V(x)=\frac{1}{2} k(x-a)^{2}-m g x,
\] where $x$ is the distance of the bead beneath $O$. Find the equilibrium position of the bead.

\part%1b
The bead is released from rest, with the spring at its natural length. Using conservation of energy, find the position of the bead when it is next at rest.
\end{parts}



\question%2
\begin{parts}
\part%2a
A particle moves on the $x$ axis under the influence of a force $F_{a}(x)$ that is inversely proportional to the square of the distance of the particle from the point $x=a>0$. Given that this force is also repulsive, show that a potential energy function is \[
	V_{a}(x)=\frac{\kappa}{|x-a|},
\] where the constant $\kappa>0$.

\part%2b
Suppose now that the particle experiences two such repulsive forces, one from $x=a$ and the other from $x=-a$, so that the total potential energy function is \[
	V(x)=V_{a}(x)+V_{-a}(x).
\] Show that the origin $x=0$ is a stable equilibrium, with the (angular) frequency of small oscillations given by $\omega=2 \sqrt{\kappa / m a^{3}}$.
\end{parts}



\question%3
Consider a unit mass $(m=1)$ particle moving in the potential \[
	V(x)=-\frac{2}{3} x^{3}-2 x^{2} .
\]
\begin{parts}
\part%3a
Show that $x=-2$ is a stable equilibrium, while $x=0$ is an unstable equilibrium.

\part%3b
By sketching the potential argue that there are bounded solutions for energies $E=$ $\frac{1}{2} \dot{x}^{2}+V$ in the range $-\frac{8}{3} \leq E<0$.

\part%3c
Verify that $x(t)=-3 \operatorname{sech}^{2}\left(t-t_{0}\right)$ is a solution with energy $E=0$. Does the particle ever reach $x=0$?

\part%3d
{}[\emph{Optional}] Compare the behaviour of the solution in part (c), near to $x=0$, with the general linearised solution about the unstable equilibrium point $x=0$.
\end{parts}



\question%4
Consider the following differential equation for $\varphi(t)$: \[
	\ddot{\varphi}+\left(\frac{g}{a}-\omega^{2} \cos \varphi\right) \sin \varphi=0 \qquad(*).
\] Here $\varphi$ is a periodic variable, with period $2 \pi$, while $g, a$ and $\omega$ are all positive constants.
\begin{parts}
\part%4a
Find the equilibrium positions for $\varphi$ (i.e. find the constant solutions to ($*$)), and determine their stability as a function of the parameter $s=g / a \omega^{2}$.

\part%4b
Sketch the stable equilibrium values of $\varphi$ as a function of $s>0$.
\end{parts}

\end{questions}

\end{document}
