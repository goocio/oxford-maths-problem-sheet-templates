\documentclass[answers]{exam}
\usepackage{../HT2025}

\title{Differential Equations II -- Sheet 1\\Variation of parameters, Adjoint problems}
\author{YOUR NAME HERE :)}
\date{Hilary Term 2025}
% version uploaded 2024-09-10


\begin{document}
\maketitle
\begin{questions}

\question%1
\textbf{Reduction of order and variation of parameters.} Define the operator \[
	\mathfrak{L} y(x) \equiv x^{2} y''(x)-x(x+2) y'(x)+(x+2) y(x), \quad 1<x<2.
\]
\begin{parts}
\part%1a
Check that $y(x)=x$ is a solution of $\mathfrak{L} y=0$, and use reduction of order to find the general solution.

\part%1b
Solve the following problem by variation of parameters: \[
	\mathfrak{L} y(x)=x^{3}, \quad y(1)=0, \quad y(2)=0 .
\]
\end{parts}



\question%2
\textbf{The Green's function via variation of parameters.}
\begin{parts}
\part%2a
Use variation of parameters to solve the problem \[
	y''(x)-2 y'(x)+2 y(x)=f(x), \quad y(0)=0, \quad y\left(\frac{\pi}{2}\right)=0,\tag{$\star$}
\] where $f$ is a given continuous function. Show that the solution can be written in the form \[
	y(x)=\int_{0}^{\pi / 2} g(x, \xi) f(\xi) ~\mathrm{d} \xi,
\] for a function $g$ (the \emph{Green's function}) which you should determine.

\part%2b
Evaluate the integral when $f(x)=e^{x}$ and check that the resulting expression for $y$ does indeed satisfy $(\star)$.
\end{parts}



\question%3
\textbf{Variation of parameters for an Initial Value Problem.} Consider the inhomogeneous ODE \[
	\mathfrak{L} y(x) \equiv P_{2}(x) y''(x)+P_{1}(x) y'(x)+P_{0}(x) y(x)=f(x) \tag{$\star$}
\] for $x>0$, subject to initial conditions $y(0)=y'(0)=0$. Suppose that the homogeneous ODE $\mathfrak{L} y=0$ has linearly independent solutions $y_{1}$ and $y_{2}$ satisfying $y_{1}(0)=0$ and $y_{2}'(0)=0$. Use variation of parameters to construct the solution to $(\star)$, and determine the Green's function $g$ such that \[
	y(x)=\int_{0}^{\infty} g(x, \xi) f(\xi) ~\mathrm{d} \xi .
\]



\question%4
\textbf{Adjoint problems.} For each of the problems below, use the adjoint relation $\langle\mathfrak{L} y, w\rangle \equiv\langle y, \mathfrak{L}^{*} w\rangle$ to determine the differential operator and boundary conditions for the adjoint problem. In each case state whether the operator and/or the full system is self-adjoint.
\begin{parts}
\part%4a
$\mathfrak{L} y=y'', \qquad 2 y(0)+y'(0)=0, \qquad y(1)+y'(1)=0$.

\part%4b
$\mathfrak{L} y=y'', \qquad 2 y(0)+y'(1)=0, \qquad y(1)+y'(0)=0$.

\part%4c
$\mathfrak{L} y=y''''-y', \qquad y'(0)-y''(0)=0, \qquad y'''(0)=0, \qquad y(1)=0, \qquad y'(1)-y'''(1)=0$.
\end{parts}



\question%5
\textbf{Sturm-Liouville form.} Consider the general second order eigenvalue problem \[
	\mathfrak{L} y(x) \equiv A(x) y''(x)+B(x) y'(x)+C(x) y(x)=\lambda y(x), \qquad a<x<b \tag{$\star$}
\] where $A(x), B(x), C(x)$ are given functions with $A(x) \neq 0$ for $x \in[a, b]$.
\begin{parts}
\part%5a
Use an integrating factor to show that $(\star)$ can always be put into Sturm-Liouville form, \[
	\hat{\mathfrak{L}} y(x) \equiv-\left(p(x) y'(x)\right)'+q(x) y(x)=\lambda r(x) y(x)
\] where $p(x), q(x), r(x)$ should be determined in terms of $A(x), B(x), C(x)$.

\part%5b
Show that the operator $\hat{\mathfrak{L}}$ is self adjoint.

\part%5c
If $\mathfrak{L} y(x)=f(x)$ for some function $f$, what is the equivalent Sturm-Liouville problem?
\end{parts}



\question%6
\textbf{FAT and existence.} Determine the parameter values $(A, B)$ that yield existence of a solution for each of the following inhomogeneous BVPs.
\begin{parts}
\part%6a
For $0 \leq x \leq 2 \pi$: \[
	y''(x)+y(x)=A \sin x+B \cos x+2 \sin \left(x+\frac{\pi}{3}\right)+\sin ^{3} x, \qquad y(0)=y(2 \pi), \qquad y'(0)=y'(2 \pi).
\] [\emph{Hint: Note the problem is fully self-adjoint.}]

\part%6b
For $0 \leq x \leq 1$: \[
	y''(x)+2 y'(x)+y(x)=1, \qquad y'(0)+y(0)=A, \qquad y'(1)+y(1)=3.
\] [\emph{Hint: First find the homogeneous adjoint problem and note that it has solution $w(x)=e^{x}$.}]
\end{parts}

\end{questions}

\end{document}
