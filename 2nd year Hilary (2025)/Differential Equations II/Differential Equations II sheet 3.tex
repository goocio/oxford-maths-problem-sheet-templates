\documentclass[answers]{exam}
\usepackage{../HT2025}

\title{Differential Equations II -- Sheet 3\\Bessel functions, Legendre functions}
\author{YOUR NAME HERE :)}
\date{Hilary Term 2025}
% version uploaded 2024-09-10


\begin{document}
\maketitle
\begin{questions}

\question%1
\textbf{Frobenius method.} Consider the differential equation \[
	(x-1) y''(x)-x y'(x)+y(x)=0 .
\] Show that $x=1$ is a regular singular point, and determine the appropriate from of the series expansions about $x=1$ for two linearly independent solutions. By explicitly computing the coefficients in the series, find closed from expressions for the two linearly independent solutions.



\question%2
\textbf{The point $x=\infty$.} Consider the differential equation \[
	x^{3} y''(x)+y(x)=0 .\tag{$\star$}
\]
\begin{parts}
\part%2a
Use the transformation of variables $x=1 / t$ to show that $(\star)$ has a regular singular point at $x=\infty$ and determine the indicial exponents.

\part%2b
Obtain the first Frobenius solution in the form of an infinite series in powers of $t$, solving explicitly for the coefficients.

\part%2c
Find the form of the second Frobenius solution and obtain (but do not attempt to solve) a recurrence relation for the coefficients in the series.

\part%2d
(Optional) Suppose we seek a particular solution to $(\star)$ with the leading behaviour $y(x) \sim x$ as $x \to \infty$. By considering the magnitude of the various terms in the two series solutions found above, determine the next largest term in the expansion of $y(x)$ for large positive $x$.
\end{parts}



\question%3
\textbf{Bessel functions.} Consider \emph{Bessel's equation} (of order $n$): \[
	x^{2} y''(x)+x y'(x)+(x^{2}-n^{2}) y(x)=0,\tag{$\star$}
\] for integer $n \geq 0$.
\begin{parts}
\part%3a
Find the indicial exponents $\alpha_{1}, \alpha_{2}$ (with $\operatorname{Re} \alpha_{1}>\operatorname{Re} \alpha_{2}$) for the local series expansion of $(\star)$ about $x=0$.

\part%3b
Determine the series $y(x)=\sum_{k=0}^{\infty} a_{k} x^{k+\alpha_{1}}$ that solves $(\star)$, giving the coefficients $a_{k}$ in closed form. Find $a_{0}$ such that the series is the expansion of the Bessel functions of first kind, \[
	J_{n}(x)=\left(\frac{x}{2}\right)^{n} \sum_{k=0}^{\infty} \frac{\left(-x^{2} / 4\right)^{k}}{k!(k+n)!}.\tag{$\#$}
\]

\part%3c
Using $(\#)$, show that the following recursion relation is true for all integers $n \geq 0$: \[
	J_{n+1}(x)=\frac{2 n}{x} J_{n}(x)-J_{n-1}(x) .
\]

\part%3d
For any integer $n \geq 0$, show that \[
	\int_{0}^{1} x[J_{n}(\alpha x)]^{2} \mathrm{~d} x=\frac{1}{2}[J_{n}'(\alpha)]^{2},
\] where $\alpha$ is any zero of $J_{n}$. [\emph{Hint: Substitute $z=\alpha x$, integrate by parts, and use the fact that $J_{n}$ satisfies Bessel's equation.}]
\end{parts}



\question%4
\textbf{Bessel functions in a Sturm-Liouville problem.}
\begin{parts}
\part%4a
Determine the bounded eigenfunctions $y_{k}$ and eigenvalues $\lambda_{k}$ of the following singular Sturm=Liouville problem on $0 \leq x \leq 1$: \[
	-(x y'(x))'=\lambda x y(x), \qquad y(1)=0.
\] [\emph{Hint: Use a change of variables of the form $r=\beta x$.}]

\part%4b
Use (a) to obtain the eigenfunction expansion for the bounded solution of the following inhomogeneous problem on $0 \leq x \leq 1$: \[
	(x y'(x))'=x, \qquad y(1)=0 .
\] [\emph{You may leave the coefficients $c_{k}$ in your final answer in terms of integrals containing Bessel functions.}]
\end{parts}



\question%5
\textbf{Legendre functions and associated Legendre functions.} Consider Legendre's equation \[
	(1-x^{2}) y''(x)-2 x y'(x)+\left(\ell(\ell+1)-\frac{m^{2}}{1-x^{2}}\right) y(x)=0
\] and let $P_{\ell}^{m}(x)$ denote the solution for integers $0 \leq m \leq \ell$. Show that \[
	\int_{-1}^{1} P_{k}^{m}(x) P_{\ell}^{m}(x) ~\mathrm{d} x= \begin{cases}
		0 & \text { if } \ell \neq k \\[1em]
		\displaystyle\frac{2}{(2 k+1)} \frac{(k+m)!}{(k-m)!} & \text { if } \ell=k
	\end{cases}
\] [\emph{You may use without proof Rodrigues' formula for $P_{\ell}^{m}(x)$ given in lectures, and also the identity \[
	\int_{-1}^{1}\left(1-x^{2}\right)^{\ell} ~\mathrm{d} x=\frac{2^{2 \ell+1}(\ell!)^{2}}{(2 \ell+1)!}
\] (or for extra fun try to show this as well).}]

\end{questions}

\end{document}
