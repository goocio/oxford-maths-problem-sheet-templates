\documentclass[answers]{exam}
\usepackage{../HT2025}

\title{Differential Equations II -- Sheet 2\\Green's functions, Eigenfunctions}
\author{YOUR NAME HERE :)}
\date{Hilary Term 2025}


\begin{document}
\maketitle
\begin{questions}

\question%1
\textbf{Computing Green's functions.} Obtain the Green's function for the following operators using the delta-function construction:
\begin{parts}
\part%1a
$\mathfrak{L} y=-y', \quad 0<x<1, \quad y(0)-y'(1)=0, \quad y(0)+y(1)=0$.

\part%1b
$\mathfrak{L} y=y'-y, \quad 0<x<2 \pi, \quad y(0)-y(2 \pi)=0, \quad y'(0)-y'(2 \pi)=0$.

\part%1c
In (b), what goes wrong if we change the operator to $\mathfrak{L} y=y'+y$ (for the same boundary conditions)?
\end{parts}



\question%2
\textbf{Green's functions for an Initial Value Problem.} Reconsider the IVP from Sheet 1 Q3, \[
	\mathfrak{L} y(x) \equiv P_{2}(x) y'(x)+P_{1}(x) y'(x)+P_{0}(x) y(x)=f(x)
\] for $x>0$, subject to initial conditions $y(0)=y'(0)=0$. Recall that $y_{1}$ and $y_{2}$ are linearly independent solutions to the homogeneous problem $\mathfrak{L} y=0$ satisfying $y_{1}(0)=0$ and $y_{2}'(0)=0$. State the ODE and initial conditions satisfied by the Green's function $g(x, \xi)$ in terms of a delta function, solve this problem for $g$, and show that this approach reproduces the expression found by variation of parameters on Sheet 1.



\question%3
\textbf{Eigenfunction expansion.}
\begin{parts}
\part%3a
Find the general solution of the Cauchy-Euler equation \[
	x^{2} y'(x)+3 x y'(x)+(1+\alpha) y(x)=0,
\] where $\alpha$ is a given positive constant.

\part%3b
Use (a) to determine the eigenvalues $\lambda_{j}$ and eigenfunctions $y_{j}$ of the self-adjoint problem \[
	-(x^{3} y'(x))'=\lambda x y, \qquad y(1)=0, \qquad y(e)=0 .
\]

\part%3c
Obtain the eigenfunction expansion for the solution of the inhomogeneous problem \[
	(x^{3} y'(x))'=x, \qquad y(1)=0, \qquad y(e)=0,
\] giving the coefficients explicitly (i.e. compute the integrals).
\end{parts}



\question%4
\textbf{Eigenvalue expansion - two routes.}
\begin{parts}
\part%4a
Consider the following eigenvalue problem on $0 \leq x \leq 1$: \[
	\mathfrak{L} y \equiv y'+2 y'+y=\lambda y, \qquad y'(0)+y(0)=0, \qquad y'(1)+y(1)=0 .
\] Compute the eigenvalues $\lambda_{k}$, eigenfunctions $y_{k}$, and the adjoint eigenfunctions $w_{k}$.

\part%4b
Under what condition on $f$ does a solution exist for the inhomogeneous problem \[
	\mathfrak{L} y(x)=f(x) \quad(0<x<1), \qquad y'(0)+y(0)=0, \qquad y'(1)+y(1)=0 ?
\]

\part%4c
Assuming that the condition in (b) is satisfied, obtain the coefficients in an eigenfunction expansion $y(x)=\sum_{k}^{\infty} c_{k} y_{k}(x)$.

\part%4d
Convert the problem in (b) to the equivalent Sturm-Liouville problem and show that the eigenfunction expansion of the solution to that problem matches what you found in part (c).
\end{parts}



\question%5
\textbf{Legendre's equation and the Fredholm Alternative.} Consider bounded solutions of the eigenvalue problem \[
	\mathfrak{L} y(x) \equiv\left(1-x^{2}\right) y'(x)-2 x y'(x)=\lambda y(x), \qquad-1<x<1 \tag{$\star$}
\]
\begin{parts}
\part%5a
Use the inner product relation to compute $\mathfrak{L}^{*}$ and show that the boundary terms vanish identically. Why are no boundary conditions given for $(\star)$?

\part%5b
Convert $(\star)$ to Sturm-Liouville form. What orthogonality relation do the eigenfunctions satisfy?

\part%5c
 Verify that $y_{0}(x)=1$ is an eigenfunction for $\lambda_{0}=0$. For the inhomogeneous problem $\mathfrak{L} y(x)=f(x)$ to be solvable for $y$, what condition must $f$ satisfy?

\part%5d
Consider the equation $\mathfrak{L} y(x)=-2 x$. Explain via the Fredholm Alternative why this problem should have a non-unique solution. Show that \[
	y=x+A \log \left(\frac{1+x}{1-x}\right)+B
\] is a solution for any values of $A$ and $B$. What can you conclude about the constant $A$?

\part%5e
Find the general solution of $\mathfrak{L} y=1$. Does this match your reasoning in (c)?
\end{parts}



\question%6
(Optional) \textbf{Green's function and eigenfunctions for Sturm-Liouville.} Consider the Sturm-Liouville problem \[
	\mathfrak{L} y \equiv-(p y')'+q y=f, \qquad a<x<b
\] where $p(x) \neq 0$ on $a<x<b$, with the boundary conditions \[
	\mathfrak{B}_{\ell} y \equiv y(a)=0, \qquad \mathfrak{B}_{r} y \equiv y(b)=0
\] Let $y_{\ell}$ and $y_{r}$ satisfy $\mathfrak{L} y_{\ell}=0, \mathfrak{B}_{\ell} y_{\ell}=0$, and $\mathfrak{L} y_{r}=0, \mathfrak{B}_{r} y_{r}=0$, respectively, and let $y_{k}$ be eigenfunctions that satisfy $\mathfrak{L} y_{k}=\lambda_{k} y_{k}$ with $\mathfrak{B}_{\ell} y_{k}=\mathfrak{B}_{r} y_{k}=0$.
\begin{parts}
\part%6a
Use variation of parameters to derive the following expression for the Green's function: \[
	g(x, \xi)= \begin{cases}
		\displaystyle\frac{-y_{\ell}(x) y_{r}(\xi)}{W(\xi) p(\xi)} & a<x<\xi<b \\[1em]
		\displaystyle\frac{-y_{\ell}(\xi) y_{r}(x)}{W(\xi) p(\xi)} & a<\xi<x<b
	\end{cases}\tag{$\star$}
\] where $W=y_{\ell} y_{r}'-y_{\ell}' y_{r}$ is the Wronskian.

\part%6b
Re-derive equation $(\star)$ by constructing the Green's function satisfying $\mathfrak{L}_{x} g(x, \xi)=\delta(x-\xi)$.

\part%6c
Find an alternative expression for the Green's function in terms of an eigenfunction expansion $g(x, \xi)=\sum_{k} c_{k}(\xi) y_{k}(x)$.

\part%6d
Show that the two expressions agree by expanding $(\star)$ directly in an eigenfunction expansion and showing that the coefficients match, i.e. write the expression in $(\star)$ as $\sum_{k} d_{k}(\xi) y_{k}(x)$ and show that $d_{k}=c_{k}$.
\end{parts}

\end{questions}

\end{document}
