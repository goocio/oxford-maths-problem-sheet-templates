\documentclass[answers]{exam}
\usepackage{../HT2025}
\RequirePackage{bbm} % for \mathbbm, gives \mathbbm{1} for indicator function

\title{Integral Transforms -- Sheet 2\\Convolution and inversion theorem, Fourier transform}
\author{YOUR NAME HERE :)}
\date{Hilary Term 2025}
% version uploaded 2024-07-26


\begin{document}
\maketitle
\begin{questions}

\question%1
The life time $T$ of a particular brand of light bulb is modelled as follows. There is a probability $p$ of the light-bulb blowing immediately (so that $T=0$); given that the light bulb does not blow immediately, the probability of it having life time $\tau$ or less is $1-e^{-\lambda \tau}$ (where $\lambda>0$).
\begin{parts}
\part%1a
Write down the cumulative distribution function, $F_{T}(t)$, of $T$.

\part%1b
Write down the (generalized) probability density function $f_{T}(t)$ of $T$.

\part%1c
What is the expectation of $T$?

\part%1d
Write down the characteristic function of $T$, that is $\mathbb{E}(e^{i s T})=\hat{f}_{T}(-s)$.
\end{parts}



\question%2
The \emph{Laguerre polynomials} $L_{n}(x)$ are defined by \[
	L_{n}(x)=e^{x} \frac{\mathrm{d}^{n}}{\mathrm{d} x^{n}}(x^{n} e^{-x}).
\] Show that $\overline{L_{n}}(p)=n!(p-1)^{n} p^{-n-1}$ and hence determine $L_{n}(x)$ for $1 \leqslant n \leqslant 4$.



\question%3
Solve using Laplace transform methods the following differential and integral equations:
\begin{parts}
\part%3a
$f'(x)+f(x)=\mathbbm{1}_{[0,1]}(x),\qquad f(0)=0$;

\part%3b
$f'(x)-2 \int_{0}^{x} f(t) e^{t-x} \mathrm{~d} t=e^{2 x},\qquad f(0)=0$.
\end{parts}



\question%4
Find the inverse Laplace transform of $(p^{3}+1)^{-1}$:
\begin{parts}
\part%4a
using partial fractions;

\part%4b
using the inversion formula;

\part%4c
using term-by-term inversion of power series. [\emph{Hint for (c): to find $\sum_{n=0}^{\infty} z^{3 n} /(3 n)!$, let $\omega=e^{2 \pi i / 3}$ so that $\omega^{3}=1$, note that $1+\omega+\omega^{2}=0$, and consider $e^{z}+e^{\omega z}+e^{\omega^{2} z}$. You can adapt this technique to find the sum you need in (c).}]
\end{parts}



\question%5
In lectures it was shown that the Fourier transform of $f=\mathbbm{1}_{[-1.1]}$ is $\hat{f}(s)=2 \sin s / s$. Determine the convolution $f * f$ and hence evaluate \[
	\int_{-\infty}^{\infty} \frac{\sin ^{2} x}{x^{2}} \mathrm{~d} x.
\]



\question%6
The function $u(x, t)$ is defined for $x \in \mathbb{R}$ and $t>0$ and solves the following boundary value problem \[
	\frac{\partial u}{\partial t}=k \frac{\partial^{2} u}{\partial x^{2}}, \qquad u(x, 0)=g(x).
\] Show that the Fourier transform $\hat{u}(s, t)$ of $u$ in the $x$ variable satisfies \[
	\frac{\partial \hat{u}}{\partial t}=-k s^{2} \hat{u}, \qquad \hat{u}(s, 0)=\hat{g}(s)
\] Deduce that \[
	\hat{u}(s, t)=\hat{g}(s) e^{-k s^{2} t}
\] and hence write down the solution $u(x, t)$ as a convolution.



\question%7
The function $u(x, y)$ is defined for $x \geqslant 0, y \geqslant 1$ and solves the following boundary value problem \[
	y \frac{\partial u}{\partial y}+\frac{\partial u}{\partial x}=1, \qquad u(x, 1)=1=u(0, y)
\] Show that the Laplace transform $\bar{u}(p, y)$ of $u$ in the $x$ variable satisfies \[
	y \frac{\partial \bar{u}}{\partial y}+p \bar{u}=\frac{1}{p}+1, \qquad \bar{u}(p, 1)=\frac{1}{p} .
\] Show further that $\bar{u}(p, y)=p^{-2}+p^{-1}-p^{-2} y^{-p}$ and deduce that \[
	u(x, y)= \begin{cases}
		1+x & \text{if } e^{x}<y \\
		1+\log y & \text{if } e^{x} \geqslant y
	\end{cases}
\]

\end{questions}

\end{document}
