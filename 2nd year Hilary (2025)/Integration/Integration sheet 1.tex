\documentclass[answers]{exam}
\usepackage{../HT2025}

\title{Integration -- Sheet 1\\Lebesgue measure}
\author{YOUR NAME HERE :)}
\date{Hilary Term 2025}
% version uploaded 2024-10-05


\begin{document}
\maketitle

\begin{questions}

\question%1
Find $\liminf_{n \to \infty} a_n$ and $\limsup_{n \to \infty} a_n$ when
\begin{parts}
\part%1a
$a_n=\exp (-\cos n)$,

\part%1b
$a_n=\exp \left(n \sin \left(\frac{n \pi}{2}\right)\right)+\exp \left(\frac{1}n \cos \left(\frac{n \pi}{2}\right)\right)$,

\part%1c
$a_n=\cosh \left(n \sin \left(\left(\frac{n^{2}+1}n\right) \frac{\pi}{2}\right)\right)$.
\end{parts}



\question%2
Let $(a_n)$ and $(b_n)$ be bounded real sequences. Prove that:
\begin{parts}
\part%2a
If $a_n \leqslant b_n$ for all $n$ then $\limsup_{n \to \infty} a_n \leqslant \limsup_{n \to \infty} b_n$;

\part%2b
$\limsup_{n \to \infty}(a_n+b_n) \leqslant \limsup_{n \to \infty} a_n+\limsup_{n \to \infty} b_n$;

\part%2c
There is a subsequence $\left(a_{n_{r}}\right)_{r \geqslant 1}$ of $(a_n)$ such that $\lim_{r \to \infty} a_{n_{r}}=\limsup_{n \to \infty} a_n$;

\part%2d
If $\left(a_{k_{r}}\right)_{r \geqslant 1}$ is any convergent subsequence of $(a_n)$, then $\lim_{r \to \infty} a_{k_{r}} \leqslant \limsup_{n \to \infty} a_n$.
\end{parts}



\question%3
\begin{parts}
\part%3a
Let $E=\mathbb{Q} \cap[0,1]$. Show that there exists a sequence $(x_n)_{n \geqslant 1}$ in $[0,1]$ such that the sets $E+x_n\coloneqq\left\{y+x_n: y \in E\right\}$ $(n=1,2, \ldots)$ are disjoint. Show that \[
	0 \leqslant \sum_{n=1}^{k} \chi_{E}(x-x_n) \leqslant \chi_{[0,2]}(x), \quad x \in \mathbb{R},\ k \in \mathbb{N}.
\]

\part%3b
Let $V$ be a vector space of functions from $\mathbb{R}$ to $\mathbb{R}$, and $\phi: V \to \mathbb{R}$ be a linear functional with the following properties: \begin{itemize}
  \item For any bounded interval $I \subset \mathbb{R}$ with endpoints $a$ and $b, \chi_{I} \in V$ and $\phi(\chi_{I})=b-a$.
  \item If $f \in V$ and $f(x) \geqslant 0$ for all $x \in \mathbb{R}$, then $\phi(f) \geqslant 0$.
  \item If $f \in V, a \in \mathbb{R}$ and $f_{a}(x)=f(x-a)$, then $f_{a} \in V$ and $\phi(f_{a})=\phi(f)$.
\end{itemize} If $\chi_{E} \in V$, show that $\phi\left(\chi_{E}\right)=0$.
\end{parts}



\question%4
Let $C$ be the Cantor set.
\begin{parts}
\part%4a
Explain, in as much detail as you think is appropriate, why \[
	C=\left\{\sum_{n=1}^{\infty} a_n 3^{-n}: a_n=0 \text{ or } 2\right\}.
\]

\part%4b
Prove that $C$ is uncountable, for example by either (or all) of the following methods:
\begin{subparts}
\subpart%4bi
Adapting Cantor's proof, via decimal expansions, that $[0,1]$ is uncountable;

\subpart%4bii
Constructing a surjection of $C$ onto $[0,1]$ --- think about binary expansions in $[0,1]$;

\subpart%4biii
Prove that $C+C=[0,2]$ and deduce that $C$ is uncountable.
\end{subparts}
\end{parts}



\question%5
\begin{parts}
\part%5a
Show that the set of all real numbers which have a decimal expansion not containing the digit 4 is null. [\emph{Consider first numbers between 0 and 1.}]

\part%5b
Show that if $A$ is null and $B$ is countable, then $A+B$ is null.

\part%5c
Show that if $A$ is null and $f: \mathbb{R} \to \mathbb{R}$ has a continuous derivative, then $f(A)$ is null. [\emph{You may wish to first consider the case when $A \subset[0,1]$ and use the fact that $f'$ is bounded on $[0,1]$.}]
\end{parts}



\question%6
Let $A, B$ and $A_n$ be subsets of $\mathbb{R}$, and $x, \alpha \in \mathbb{R}$. Prove the following:
\begin{parts}
\part%6a
$m^{*}(A+x)=m^{*}(A)$;

\part%6b
$m^{*}(\alpha A)=|\alpha| m^{*}(A)$;

\part%6c
$m^{*}(A \cup B) \leqslant m^{*}(A)+m^{*}(B)$;

\part%6d
$m^{*}\left(\bigcup_{n=1}^{\infty} A_n\right) \leqslant \sum_{n=1}^{\infty} m^{*}\left(A_n\right)$.
\end{parts}



\question%7
Prove the following:
\begin{parts}
\part%7a
Any null set is (Lebesgue) measurable;

\part%7b
Any interval is measurable;

\part%7c
If $E$ and $F$ are measurable and $x, \alpha \in \mathbb{R}$, then $E+x, \alpha E$ and $E \cup F$ are measurable;

\part%7d
If $E_n$ are disjoint measurable subsets of $\mathbb{R}$, then $\bigcup_{n=1}^{\infty} E_n$ is measurable and $m^{*}\left(\bigcup_{n=1}^{\infty} E_n\right)=\sum_{n=1}^{\infty} m^{*}\left(E_n\right)$. [\emph{Hint: first use the previous part to show that for each $n$, we have $m^{*}(A) \geqslant \sum_{r=1}^n m^{*}\left(A \cap E_{r}\right)+m^{*}\left(A \setminus \bigcup_{r=1}^{\infty} E_{r}\right)$.}]
\end{parts}



\question%8
(Optional) Let $G$ be an open subset of $\mathbb{R}$. For $x, y \in G$, let $I_{x, y}$ be the closed (or open, if you prefer) interval between $x$ and $y$, so $I_{x, x}=\{x\}$ (or $\emptyset$). Define a relation $\sim$ on $G$ by $x \sim y$ if and only if $I_{x, y} \subset G$.
\begin{parts}
\part%8a
Show that $\sim$ is an equivalence relation on $G$.

\part%8b
Show that each equivalence class is an open interval. [\emph{To show that $A$ is an interval, it is sufficient to check that, if $x, y \in A$ then $I_{x, y} \subset A$.}]

\part%8c
Show that there are (at most) countably many equivalence classes.

\part%8d
Deduce that $G$ is the union of (at most) countably many, disjoint open intervals, and in particular $G$ is measurable.
\end{parts}

\end{questions}

\end{document}
