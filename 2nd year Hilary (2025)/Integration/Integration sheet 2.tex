\documentclass[answers]{exam}
\usepackage{../HT2025}

\title{Integration -- Sheet 2\\Measurable functions, Lebesgue integral}
\author{YOUR NAME HERE :)}
\date{Hilary Term 2025}
% version uploaded 2024-10-05


\begin{document}
\maketitle

\begin{questions}

\question%1
Let $(\omega_{n})$ be a sequence of non-negative real numbers. For a subset $E$ of $\mathbb{N}$, let \[
	\mu_{\omega}(E)=\sum_{n \in E} \omega_{n}.
\]
\begin{parts}
\part%1a
Show that $\mu_{\omega}$ is a measure on $(\mathbb{N}, \mathcal{P}(\mathbb{N}))$.

\part%1b
Now let $\nu$ be any measure on $(\mathbb{N}, \mathcal{P}(\mathbb{N}))$, and define $\omega_{n}=\nu(\{n\})$. Show that $\nu(E)=\mu_{\omega}(E)$ for all subsets $E$ of $\mathbb{N}$.
\end{parts}



\question%2
Let $(\Omega, \mathcal{F}, \mu)$ be any measure space. If $(A_{n})$ is a decreasing sequence of sets in $\mathcal{F}$ and $\mu(A_{1})<\infty$, prove that \[
	\mu\left(\bigcap_{n=1}^{\infty} A_{n}\right)=\lim _{n \to \infty} \mu(A_{n}) .
\] Is this still true if $\mu(A_{1})=\infty$?



\question%3
\begin{parts}
\part%3a
Let $b \in \mathbb{R}$. Show that $(-\infty, b)=\bigcup_{n=1}^{\infty}\left(\mathbb{R} \setminus\left(b-\frac{1}{n}, \infty\right)\right)$. Deduce that if $\mathcal{F}$ is a $\sigma$-algebra on $\mathbb{R}$ containing the intervals $(a, \infty)$ for each $a \in \mathbb{R}$, then $\mathcal{F}$ contains all open intervals, and hence all open subsets of $\mathbb{R}$.

\part%3b
Let $f: \mathbb{R} \to \mathbb{R}$ be a function. Show that \[
	\mathcal{G}\coloneqq\left\{G \subset \mathbb{R}: f^{-1}(G) \in \mathcal{M}_{\mathrm{Leb}}\right\}
\] is a $\sigma$-algebra. Deduce that if $f^{-1}(a, \infty) \in \mathcal{M}_{\text {Leb }}$ for every $a$, then $f^{-1}(G) \in \mathcal{M}_{\text {Leb }}$ for every $G \in \mathcal{M}_{\mathrm{Bor}}$.
\end{parts}



\question%4
\begin{parts}
\part%4a
Let $I$ be an interval of positive length, let $a \in I, f, g: I \to \mathbb{R}$ be functions such that $f(x)=g(x)$ a.e., and suppose that $f$ and $g$ are continuous at $a$. Show that $f(a)=g(a)$.

\part%4b
Is $\chi_{\mathbb{Q}}$ continuous a.e.? Does there exist a continuous function $g$ such that $\chi_{\mathbb{Q}}=g$ a.e.?

\part%4c
Is $\chi_{(0, \infty)}$ continuous a.e.? Does there exist a continuous function $g$ such that $\chi_{(0, \infty)}=g$ a.e.? [\emph{Use (a).}]
\end{parts}



\question%5
Let $f, g$ be measurable functions from $\mathbb{R}$ to $\mathbb{R}$, and $h: \mathbb{R} \to \mathbb{R}$ be continuous. Recall from lectures that $f+g$ and $h \circ f$ are measurable. Prove that the following functions are measurable. [\emph{Complicated constructions are not required. Everything can be quickly deduced from the information from lectures recalled above, plus a couple of simple formulae.}]
\begin{parts}
\part%5a
$f^{2}: x \mapsto f(x)^{2}$;

\part%5b
$f g: x \mapsto f(x) g(x)$;

\part%5c
$|f|: x \mapsto|f(x)|$;

\part%5d
$\max (f, g): x \mapsto \max (f(x), g(x))$.
\end{parts}



\question%6
\begin{parts}
\part%6a
Suppose that $g$ is a measurable function and $f=g$ a.e. Show that $f$ is measurable.

\part%6b
Suppose that $f$ is continuous a.e.. Show that there is a sequence of step functions $(\phi_n)$ such that $f=\lim_{n \to \infty} \phi_{n}$ a.e.. Deduce that $f$ is measurable.
\end{parts}



\question%7
Let $f: \mathbb{R} \to[-\infty, \infty]$ be an integrable function, and let $\alpha>0$.
\begin{parts}
\part%7a
Show that \[
	m(\{x:|f(x)| \geqslant \alpha\}) \leqslant \frac{1}{\alpha} \int|f| .
\]

\part%7b
Deduce that
\begin{subparts}
\subpart%7bi
$f(x) \in \mathbb{R}$ a.e.\\
\subpart%7bii
If $\int|f|=0$, then $f(x)=0$ a.e..
\end{subparts}
\end{parts}



\question%8
(Optional) Let $(\Omega, \mathcal{F}, \mu)$ be a measure space, $\Omega_{*}$ be a set, and $f: \Omega \to \Omega_{*}$ be a function. Let \[
	f_{*}(\mathcal{F})=\{G \subset \Omega_{*}: f^{-1}(G) \in \mathcal{F}\}, \qquad
	(f_{*} \mu)(G)=\mu(f^{-1}(G)).
\]
\begin{parts}
\part%8a
Show that $(\Omega_{*}, f_{*}(\mathcal{F}), f_{*} \mu)$ is a measure space.

\part%8b
Now let $(\Omega, \mathcal{F}, \mu)=(\mathbb{R}, \mathcal{M}_{\mathrm{Bor}}, m)$, and $\Omega_{*}=\mathbb{R}$. Determine $f_{*}(\mathcal{M}_{\mathrm{Bor}})$ and $f_{*} m$ when
\begin{subparts}
\subpart%8bi
$f(x)=\tan x$ if $\cos x \neq 0$, and $f(x)=0$ if $\cos x=0$;

\subpart%8bii
$f(x)=\arctan x($ taking values in $(-\pi / 2, \pi / 2))$.
\end{subparts}
\end{parts}

\end{questions}

\end{document}
