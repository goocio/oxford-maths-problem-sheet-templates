\documentclass[answers]{exam}
\usepackage{../HT2025}

\title{Fluids and Waves -- Sheet 3\\Conformal maps, Vortices}
\author{YOUR NAME HERE :)}
\date{Hilary Term 2025}


\begin{document}
\maketitle
\begin{questions}

\section*{A: Fundamental Principles}

\question%1
\textbf{Conformal maps.} Define the term \emph{conformal map}. Write down conformal maps from the wedge $0<\arg (z)<\alpha$, from the strip $-a<y<a$ and from the semi-infinite strip $0<y<a, x>0$ onto the upper half-plane. Find all the points at which each map is not conformal.



\question%2
\textbf{Preservation under conformal mapping.} Show that the strength of a source or of a vortex is unaltered by a conformal transformation of the complex plane.



\section*{B: Applications}

\question%3
Fluid occupies the region between two plane rigid boundaries at $y= \pm b$, and there is a line vortex of strength $\Gamma$ at the origin $(x, y)=(0,0)$. Find the complex potential for the resulting flow by using
\begin{parts}
\part%3a
the method of images;

\part%3b
the conformal mapping $\zeta=e^{\alpha z}$ with suitably chosen $\alpha>0$.
\end{parts}



\question%4
\begin{parts}
\part%4a
Show that the circle $|\zeta|=a$ is mapped to a line segment \[
	S=\{z: \operatorname{Im} z=0,-2 a \leq \operatorname{Re} z \leq 2 a\}
\] along the real- $z$-axis by the Joukowski transformation \[
	z=\zeta+\frac{a^{2}}{\zeta}.
\]

\part%4b
Deduce that the exterior of the line segment $S$ is mapped to the exterior of the circle $|\zeta|=a$ by the transformation \[
	\zeta=\frac{1}{2}\left(z+\sqrt{z^{2}-4 a^{2}}\right) .\tag{$\star$}
\] Carefully define the function $\sqrt{z^{2}-4 a^{2}}$ and determine where the mapping $(\star)$ is conformal.
\end{parts}



\question%5
\begin{parts}
\part%5a
Show that the complex potential for a uniform stream of magnitude $U$ aligned at an angle $\alpha$ to the real-$\zeta$-axis with circulation $\Gamma$ around a stationary circular cylinder of radius $a$ centred on the origin is given by \[
	W(\zeta)=U\left(\zeta e^{-i \alpha}+\frac{a^{2} e^{i \alpha}}{\zeta}\right)-\frac{i \Gamma}{2 \pi} \log \zeta
\]

\part%5b
Hence find the complex potential $w(z)$ for flow past a flat plate at angle of incidence $\alpha$ in a uniform stream of magnitude $U$ with circulation $\Gamma$. Deduce that the velocity at the trailing edge of the plate is finite only if the circulation satisfies the \emph{Kutta condition} \[
	\Gamma+4 \pi U a \sin \alpha=0.
\]

\part%5c
Hence use the Kutta-Joukowski Lift Theorem to find the drag and lift forces experienced by the plate.
\end{parts}



\question%6
Two vortices, of strengths $\Gamma_{1}$ and $\Gamma_{2}$, are at the points $z=z_{1}$ and $z=z_{2}$ respectively in the complex plane.
\begin{parts}
\part%6a
Write down the equations of motion for the position vectors $z_{1}(t)$ and $z_{2}(t)$ if the vortices are free to move. Assuming that $\Gamma_{1}+\Gamma_{2} \neq 0$, show that $\mathrm{d} Z / \mathrm{d} t=\mathrm{d} a / \mathrm{d} t=0$, where \[
	Z=\frac{\Gamma_{1} z_{1}+\Gamma_{2} z_{2}}{\Gamma_{1}+\Gamma_{2}}
\] is the centroid of the two vortices, and $a=|z_{1}-z_{2}|$ is the distance between them.

\part%6b
Deduce that each vortex moves in a circle centred on $Z$, with angular velocity \[
	\Omega=\frac{\Gamma_{1}+\Gamma_{2}}{2 \pi a^{2}}
\] What happens in the exceptional case where $\Gamma_{1}+\Gamma_{2}=0$?
\end{parts}



\question%7
Fluid occupies the region $x^{2}+y^{2}>a^{2}$ outside a circular obstacle of radius $a$.
\begin{parts}
\part%7a
By using the Circle Theorem, find the resulting complex potential when a vortex of strength $\Gamma$ is placed at $(x, y)=(b, 0)$, where $b>a$ (assuming there to be no circulation about the obstacle).

\part%7b
Explain why the vortex will move in a circle of radius $b$ with angular velocity of magnitude \[
	\Omega=\frac{\Gamma a^{2}}{2 \pi b^{2}(b^{2}-a^{2})} .
\]
\end{parts}


\question%8
Fluid occupies the quadrant $x>0, y>0$ bounded by two rigid boundaries along the $x$- and $y$-axes. Find the complex potential for the flow caused by a vortex at a point $z=c=a+i b$ in the fluid. If the vortex is free to move, show that it follows a path on which \[
	\frac{1}{x^{2}}+\frac{1}{y^{2}}=\text{constant.}
\]



\question%9
\relax[\emph{Harder}] Fluid occupies the semi-infinite channel $\{z: \operatorname{Re} z>0,-a<\operatorname{Im} z<a\}$.
\begin{parts}
\part%9a
Show that the flow induced by a line vortex of strength $\Gamma>0$ at the point $z=d \in \mathbb{R}^{+}$ has complex potential \[
	w(z)=\frac{i \Gamma}{2 \pi}\left\{-\log \left[\sinh \left(\frac{\pi z}{2 a}\right)-\sinh \left(\frac{\pi d}{2 a}\right)\right]+\log \left[\sinh \left(\frac{\pi z}{2 a}\right)+\sinh \left(\frac{\pi d}{2 a}\right)\right]\right\}.
\]

\part%9b
Show that the velocity components satisfy \[
	u-i v=\frac{i \Gamma}{4 a}\left\{\operatorname{cosech}\left(\frac{\pi(z+d)}{2 a}\right)-\operatorname{cosech}\left(\frac{\pi(z-d)}{2 a}\right)\right\}
\]

\part%9c
Deduce that, if the vortex is free to move, it will instantaneously travel downwards with speed $\frac\Gamma{4a}\operatorname{cosech}\frac{\pi d}a$.
\end{parts}

\end{questions}

\end{document}
