\documentclass[answers]{exam}
\usepackage{../HT2025}

\title{Fluids and Waves -- Sheet 2\\Complex potential, Blasius' theorem}
\author{YOUR NAME HERE :)}
\date{Hilary Term 2025}
% version uploaded 2025-01-10


\begin{document}
\maketitle
\begin{questions}

\question%1
\textbf{An example of the velocity potential.} An impermeable sphere has a time-varying radius, $R(t)=a(1+\cos \Omega t)$, which induces a time-dependent irrotational, incompressible flow. The resulting velocity potential $\phi(r, t)$ satisfies Laplace's equation, $\nabla^{2} \phi=0$, which in spherically-symmetric coordinates reads \[
	0=\frac{1}{r^{2}} \frac{\partial}{\partial r}\left(r^{2} \frac{\partial \phi}{\partial r}\right)
\]
\begin{parts}
\part%1a
Explain why appropriate boundary conditions for $\phi(r, t)$, are $\phi_{r}(R, t)=\dot{R}$ and $|\nabla \phi| \to 0$ as $r \to \infty$; solve for $\phi(r, t)$.

\part%1b
By using an appropriate form of Bernoulli's principle and neglecting any body force, calculate the pressure, $p[R(t), t]$, on the surface of the sphere (measured relative to the pressure at $\infty$ ).

\part%1c
Show that the time average of the surface pressure, $\overline{p-p_\infty}=\frac{\Omega}{2 \pi} \int_{0}^{2 \pi / \Omega} (p[R(t), t]-p_\infty) ~\mathrm{d} t$, is non-zero.
\end{parts}



\question%2
A line source of strength $Q$ is at $z=a$, and a line sink of the same strength is at $z=-a$, where $a \in \mathbb{R}^{+}$.
\begin{parts}
\part%2a
Write down the complex potential $w(z)$. Find $\mathrm{d} w / \mathrm{d} z$, locate any stagnation points and sketch the streamlines.

\part%2b
Now let $a \to 0$ and $Q \to \infty$ while keeping the product $a Q$ fixed. This gives the flow due to a doublet. Show that its complex potential is $\mu / z$, where $\mu$ is to be found in terms of $a$ and $Q$. Show that the streamlines are circles through the origin with centres on the $y$-axis.
\end{parts}



\question%3
The velocity field $\boldsymbol{u}=(Q / 2 \pi r) \boldsymbol{e}_{r}$, in terms of plane polar coordinates $(r, \theta)$, corresponds to a line source if $Q>0$ or a line sink if $Q<0$.
\begin{parts}
\part%3a
Show that it is irrotational and incompressible for $r>0$. Find the velocity potential and streamfunction, and show that the complex potential is \[
	w=\frac{Q}{2 \pi} \log z.
\] Explain why the streamfunction is a multi-valued function of position.

\part%3b
Fluid occupies the region $x>0$ and there is a plane rigid boundary at $x=0$. Find the complex potential for the flow due to a line source at the point $(d, 0)$, where $d>0$, and show that the pressure at $x=0$ decreases to a minimum at $|y|=d$ and thereafter increases with $y$.
\end{parts}



\question%4
Incompressible inviscid fluid occupies the region $y>0$, and there is an impermeable wall at $y=0$. There is a uniform flow, speed $U$, in the positive $x$-direction, and a line source of strength $Q$ at $(0, a)$, where $a>0$. Find the complex potential $w(z)$ and calculate $\mathrm{d} w / \mathrm{d} z$. Let $\beta=Q / 2 \pi U a$. Show that if $\beta>1$ there are two stagnation points, both on the wall, while if $\beta<1$ there is only one, in the fluid, a distance $a$ from the origin. Try to sketch the streamlines in either case.



\question%5
\textbf{Blasius' Theorem.} Consider the steady, two-dimensional, irrotational flow of a fluid with constant density $\rho$ past a closed body $B$.
\begin{parts}
\part%5a
If $p$ and $w$ are the pressure and complex potential of the flow, and $z=x+i y$, show that the force exerted on $B$ by the fluid is $(F_{x}, F_{y})$, where \[
	F_{x}+i F_{y}=\oint_{\partial B} pi ~\mathrm{d} z=-\frac{i\rho}{2} \oint_{\partial B}\left|\frac{\mathrm{d} w}{\mathrm{d} z}\right|^2 \mathrm{~d} z
\] Explain why the integral on the right-hand side is not amenable to calculation via Cauchy's Theorem as it stands.

\part%5b
By taking the complex conjugate, or otherwise, deduce \emph{Blasius' Theorem}: \[
	F_{x}-iF_{y}=\frac{i\rho}{2} \oint_{\partial B}\left(\frac{\mathrm{d} w}{\mathrm{d} z}\right)^{2} \mathrm{~d} z
\]

\part%5c
Show that the moment (about the centre of $B$) exerted on $B$ by the fluid may similarly be written \[
	M=\Re\left[-\frac{\rho}{2} \oint_{\partial B} z\left(\frac{\mathrm{d} w}{\mathrm{d} z}\right)^{2} \mathrm{~d} z\right] .
\]
\end{parts}



\question%6
\textbf{Flow inside a cylinder: a modified Circle Theorem.} Modify the proof of Milne-Thomson's Circle Theorem to show that if $f(z)$ has all of its singularities lying in $|z|<a$ then the function \[
	w(z)=f(z)+\overline{f\left(\frac{a^{2}}{\bar{z}}\right)}
\] has:
\begin{parts}
\part%6a
the same singularities as $f$ within $|z|<a$;

\part%6b
the circle $|z|=a$ as a streamline.
\end{parts}



\question%7
Inviscid, incompressible fluid occupies the region $x^{2}+y^{2}>a^{2}$ outside a rigid circular cylinder of radius $a$. There is a line source of strength $Q$ at $(b, 0)$, where $b>a$, and there is also a circulatory flow around the cylinder as if due to a line vortex of strength $\Gamma$ at the origin.
\begin{parts}
\part%7a
Explain why the complex potential is \[
	w(z)=\frac{Q}{2 \pi} \log (z-b)+\frac{Q}{2 \pi} \log \left(\frac{a^{2}}{z}-b\right)-\frac{i \Gamma}{2 \pi} \log z .
\]

\part%7b
Calculate $\mathrm{d} w / \mathrm{d} z$ and use Blasius' Theorem to find the force components $(F_{x}, F_{y})$ on the cylinder.\\
Adding the results, we have \[
  F_x-iF_y=-\rho\frac{Q}{2\pi}\left[-\frac{Qa^2}{b(b^2-a^2)}+\frac{i\Gamma}{b}\right]
\] and hence \[
  F_x=\frac{\rho Q^2a^2}{2\pi b(b^2-a^2)},\quad F_y=\frac{\rho Q\Gamma}{2\pi b}.
\] Note that the horizontal component of the force is positive, independent of the sign of $Q$ -- the cylinder is pulled towards the source/sink. However, the sign of the vertical component of the force depends on the sign of the product $Q\Gamma$: when there is positive circulation and a source, the fluid velocity is highest (and the pressure is lowest) above the cylinder so that the force acts to `suck' the cylinder vertically upwards. When there is negative circulation (and a source), the highest velocity is at the bottom of the cylinder and the direction of the force reverses.
\end{parts}



\question%8
A two-dimensional irrotational incompressible flow has streamfunction $\psi=A(x-c) y$, where $A$ and $c$ are real constants.
\begin{parts}
\part%8a
Write down (without further calculation), the corresponding complex potential. (\emph{This is unique up to an additive constant. Why?})

\part%8b
A circular cylinder of radius $a<\abs{c}$ is introduced, its centre being at the origin. Find the force exerted on the cylinder by the resulting flow.
\end{parts}

\end{questions}

\end{document}
