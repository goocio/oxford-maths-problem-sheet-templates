\documentclass[answers]{exam}
\usepackage{../HT2025}

\title{Statistics -- Sheet 4\\Prior and Posterior distributions}
\author{YOUR NAME HERE :)}
\date{Hilary Term 2025}


\begin{document}
\maketitle

\begin{questions}

\question%1
Suppose $X_{1}, \ldots, X_{n}$ are independent, each having a geometric distribution with probability mass function $f(x \mid \theta)=(1-\theta)^{x} \theta$ for $x=0,1, \ldots$. Suppose that the prior for $\theta$ is a $\operatorname{Beta}(a, b)$ density. Find the posterior distribution of $\theta$.



\question%2
Let $\theta>0$ be an unknown parameter and let $c>0$ be a known constant. Conditional on $\theta$, suppose $X_{1}, \ldots, X_{n}$ are independent each with probability density function \[
	f(x \mid \theta)=\theta c^{\theta} x^{-(\theta+1)}, \quad x \geqslant c
\] and suppose the prior for $\theta$ is a $\operatorname{Gamma}(\alpha, \beta)$ density. Find the posterior distribution of $\theta$.



\question%3
Let $r \geqslant 1$ be a known integer and let $\theta \in[0,1]$ be an unknown parameter. The negative binomial distribution with index $r$ and parameter $\theta$ has probability mass function \[
	f(x \mid \theta)=\binom{x+r-1}{x}(1-\theta)^{x} \theta^{r} \quad \text { for } x=0,1, \ldots
\] Let $\theta$ have a $\operatorname{Beta}(a, b)$ prior density and suppose, given $\theta$, that $X_{1}, \ldots, X_{n}$ are independent each with the above negative binomial distribution.
\begin{parts}
\part%3a
Show that the posterior density is also a Beta density.

\part%3b
Explain how to construct a $100(1-\alpha) \%$ equal-tailed credible interval for $\theta$. Will this interval be a highest posterior density interval?
\end{parts}


\question%4
Suppose that $X$ has a $N(\theta, \phi)$ distribution, where $\phi$ is known, Suppose also that the prior distribution for $\theta$ is $N(\theta_{0}, \phi_{0})$, where $\theta_{0}$ and $\phi_{0}$ are known.
\begin{parts}
\part%4a
Find the posterior distribution of $\theta$ given $X=x$.

\part%4b
Show that the posterior mean of $\theta$ always lies between the prior mean and the observed value $x$.

\part%4c
Construct a $100(1-\alpha) \%$ highest posterior density interval for $\theta$.

\part%4d
Let $\phi=2, \theta_{0}=0$ and $\phi_{0}=2$.
\begin{subparts}
\subpart%4di
Suppose the observed value is $x=4$. What are the mean and variance of the resulting posterior distribution? Sketch the prior, likelihood, and posterior on a single set of coordinate axes.

\subpart%4dii
Repeat (i) assuming $\phi_{0}=18$. Explain any resulting differences. Which of these two priors would likely have more appeal for a frequentist statistician?
\end{subparts}
\end{parts}



\question%5
Let $X$ be the number of heads when a coin with probability $\theta$ of heads is flipped $n$ times.
\begin{parts}
\part%5a
When the prior is $\pi(\theta)$, the prior predictive distribution for $X$ (the predictive distribution before observing any data) is given by \[
	P(X=k)=\int_{0}^{1} P(X=k \mid \theta) \pi(\theta)~\mathrm d \theta, \quad k=0,1, \ldots, n
\] Find the prior predictive distribution when $\pi(\theta)$ is uniform on $(0,1)$.

\part%5b
Suppose you assign a $\operatorname{Beta}(a, b)$ prior for $\theta$, and then you observe $x$ heads out of $n$ flips. Show that the posterior mean of $\theta$ is always lies between your prior mean, $a /(a+b)$, and the observed relative frequency of heads, $x / n$.

\part%5c
Show that, if the prior distribution on $\theta$ is uniform, then the posterior variance is always less than the prior variance.

\part%5d
Give an example of a $\operatorname{Beta}(a, b)$ prior distribution and values of $x, n$ for which the posterior variance is larger than the prior variance. (Try $x=n=1$.)
\end{parts}



\question%6
A coin, with probability $\theta$ of heads, is flipped $n$ times and $r$ heads are observed.
\begin{parts}
\part%6a
If the prior for $\theta$ is a uniform distribution on $(0,1)$, what is the probability that the next flip is a head?

\part%6b
Can you generalise to the case where $\theta$ has a $\operatorname{Beta}(a, b)$ prior and where we wish to find the probability of getting $k$ heads from $m$ further flips?
\end{parts}


\question%7
\begin{parts}
\part%7a
Let $X \sim N(\theta, \sigma_{0}^{2})$, where $\sigma_{0}^{2}$ is known. Find the Jeffreys' prior for $\theta$.

\part%7b
Let $X \sim N(\mu_{0}, \sigma^{2})$, where $\mu_{0}$ is known. Find the Jeffreys' prior for $\sigma$.

\part%7c
Let $X$ be Poisson with parameter $\lambda$. Find the Jeffreys' prior for $\lambda$. Check that the posterior distribution of $\lambda$ given $X=x$ is proper, but that the Jeffreys' prior is not.
\end{parts}



\question%8
Suppose $X$ is the number of successes in a binomial experiment with $n$ trials and probability of success $\theta$. Either $H_{0}: \theta=\frac{1}{2}$ or $H_{1}: \theta=\frac{3}{4}$ is true. Show that the posterior probability that $H_{0}$ is true is greater than the prior probability for $H_{0}$ if and only if \[
	x \log 3<n \log 2.
\]



\question%9
Let $X \sim \operatorname{Binomial}(n, \theta)$, where the prior for $\theta$ is uniform on $(0,1)$. Suppose that we wish to compare the hypotheses $H_{0}: \theta \leqslant \frac{1}{2}$ and $H_{1}: \theta>\frac{1}{2}$.
\begin{parts}
\part%9a
What are the prior odds of $H_{0}$ relative to $H_{1}$?

\part%9b
Find an expression for the posterior odds of $H_{0}$ relative to $H_{1}$.

\part%9c
If we observe $X=n$, find the Bayes factor $B$ of $H_{0}$ relative to $H_{1}$.

\part%9d
Check that $B \to 0$ as $n \to \infty$. Why is this expected?
\end{parts}



\question%10
Suppose we have a random sample $X_{1}, \ldots, X_{n}$ from a Poisson distribution with mean $\theta$. Suppose we wish to test the hypothesis $H_{0}: \theta=\theta_{0}$ against $H_{1}: \theta \neq \theta_{0}$ and that, under $H_{1}$, the prior distribution $\pi(\theta \mid H_{1})$ for $\theta$ is given by \[
	\pi(\theta \mid H_{1})=\frac{\beta^{\alpha}}{\Gamma(\alpha)} \theta^{\alpha-1} e^{-\beta \theta}, \quad \theta>0.
\]
\begin{parts}
\part%10a
Calculate the Bayes factor of $H_{0}$ relative to $H_{1}$.

\part%10b
When $n=6, \sum x_{i}=19, \theta_{0}=2$, find the numerical value of the Bayes factor (i) when $\alpha=4$ and $\beta=\frac{2}{3}$, and (ii) when $\alpha=36$ and $\beta=6$. Compare and interpret the values of the Bayes factor in cases (i) and (ii).
\end{parts}

\end{questions}

\end{document}
