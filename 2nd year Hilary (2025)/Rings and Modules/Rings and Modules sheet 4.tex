\documentclass[answers]{exam}
\usepackage{../HT2025}

\title{Rings and Modules -- Sheet 4\\Free modules}
\author{YOUR NAME HERE :)}
\date{Hilary Term 2025}


\begin{document}
\maketitle

\begin{questions}

\question%1
Give proofs or counterexamples for the following over a Euclidean domain.
\begin{parts}
\part%1a
In a free module, can a linearly independent set always be extended to a basis?

\part%1b
In a free module, is a minimal spanning set necessarily a basis?

\part%1c
Is the rank of a proper submodule of a finite-rank free module always strictly less than the rank of the module?
\end{parts}



\question%2
\begin{parts}
\part%2a
Given a ring $R$, considered as an $R$-module, what are the submodules of $R$?

\part%2b
Show that $\mathbb{Q}$ is not finitely generated as a $\mathbb{Z}$-module.

\part%2c
Show that $M_{1}=\mathbb{R}[x] /\langle x\rangle$ and $M_{2}=\mathbb{R}[x] /\langle x-1\rangle$ are isomorphic as rings but not as $\mathbb{R}[x]$-modules.
\end{parts}



\question%3
Let $M$ be a finitely-generated module over an integral domain $R$.
\begin{parts}
\part%3a
By considering a maximal linearly independent subset of a generating set $X$, show that $M$ contains a free submodule $F$ such that $M / F$ is a torsion module.

\part%3b
Show that for a finitely generated torsion module $N$ the ideal \[
	\operatorname{Ann}_{R}(N)=\{r \in R: r . n=0 \text { for all } n \in N\}
\] is nonzero.

\part%3c
Deduce that a finitely-generated torsion-free module over a PID is free.
\end{parts}



\question%4
Find the cyclic decomposition of the Abelian group with generators $a, b, c$ and relations $2 a-16 b-8 c=0$ and $4 a+24 b+8 c=0$.



\question%5
Up to isomorphism, how many Abelian groups are there of order 120?



\question%6
Find the rational canonical form of \[
	\begin{pmatrix}
		1 & 1 & 0 & 0 \\
		0 & 1 & 0 & 0 \\
		2 & 3 & -1 & 4 \\
		1 & 1 & -1 & 3
	\end{pmatrix}
\] and its minimum and characteristic polynomials. What is the Jordan canonical form for this matrix?



\question%7
Let $R=\mathbb{Z}[x]$ and consider the ideal $M=\langle 2, x\rangle$ as an $R$-module. Show that $M$ is not the direct sum of cyclic $R$-modules. [\emph{You may find it helpful to observe that the cyclic submodules of $R$ are principal ideals.}]

\end{questions}

\end{document}
