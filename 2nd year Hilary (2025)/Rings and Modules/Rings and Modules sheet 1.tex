\documentclass[answers]{exam}
\usepackage{../HT2025}

\title{Rings and Modules -- Sheet 1\\Rings, Ideals}
\author{YOUR NAME HERE :)}
\date{Hilary Term 2025}
% version uploaded 2025-01-12


\begin{document}
\maketitle

\begin{questions}

\question%1
Let $C(\mathbb{R})$ denote the set of continuous functions $\mathbb{R} \to \mathbb{R}$.
\begin{parts}
\part%1a
Show that it is a ring with the usual operations of addition and multiplication of functions. What are the zero-divisors?

\part%1b
How do your answers change if we consider instead the ring $\mathcal{O}$ of holomorphic functions $\mathbb{C} \to \mathbb{C}$?
\end{parts}



\question%2
Show that the set of rational numbers with an odd denominator is a subring of $\mathbb{Q}$. What are the units? What are the ideals?



\question%3
Show that a finite integral domain is a field.



\question%4
Which of the following $I_{k}$ are ideals in the polynomial ring $\mathbb{R}[x]$? If $I_{k}$ is an ideal, find a generator, i.e. an element $p_{k}$ such that $I_{k}=\left\langle p_{k}\right\rangle$.
\begin{parts}
\part%4a
$I_{1}=\{f(x) \in \mathbb{R}[x]: f(2)=0\}$,

\part%4b
$I_{2}=\{f(x) \in \mathbb{R}[x]: f'(2)=0\}$,

\part%4c
$I_{3}=\{f(x) \in \mathbb{R}[x]: f(2)=f(3)=0\}$,

\part%4d
$I_{4}=\{f(x) \in \mathbb{R}[x]: f(2)=f'(2)=0\}$.
\end{parts}



\question%5
Let $R$ be a commutative ring with 1 such that the only ideals are $\{0\}$ and $R$ itself. Show that $R$ is a field.



\question%6
Let $I$ and $J$ be ideals in a commutative ring $R$. Show that $I J \subseteq I \cap J$ and that we have equality if $I+J=R$ (i.e. if $I, J$ are coprime). Does equality hold in general?



\question%7
Define \[
	\mathbb{H}=\left\{\begin{pmatrix}
		x & y \\
		-\bar{y} & \bar{x}
	\end{pmatrix}: x, y \in \mathbb{C}\right\}.
\] Show that $\mathbb{H}$ is a ring, with usual matrix addition and multiplication. Is it commutative? Show that every nonzero element has a multiplicative inverse.



\question%8
\begin{parts}
\part%8a
Find all solutions $x \in \mathbb{Z}$ to the simultaneous congruences $x \equiv 8 \bmod 17$ and $x \equiv 3 \bmod 9$.

\part%8b
Find a general formula for solving the simultaneous congruences \[
	x \equiv a_{i} \bmod n_{i} \quad: \quad i=1, \ldots r
\] where $\left(n_{i}, n_{j}\right)=1$ for $i \neq j$.
\end{parts}

\end{questions}

\end{document}
