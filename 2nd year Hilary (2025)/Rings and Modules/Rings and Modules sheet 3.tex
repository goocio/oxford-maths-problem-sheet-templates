\documentclass[answers]{exam}
\usepackage{../HT2025}

\title{Rings and Modules -- Sheet 3\\Generators, Modules}
\author{YOUR NAME HERE :)}
\date{Hilary Term 2025}


\begin{document}
\maketitle

\begin{questions}

\question%1
Factorise the following elements into irreducibles.
\begin{parts}
\part%1a
$36 x^{3}-24 x^{2}-18 x+12$ in $\mathbb{Z}[x]$,

\part%1b
$x^{6}-1$ in $\mathbb{Z}_{7}[x]$,

\part%1c
$32+9 i$ in $\mathbb{Z}[i]$.
\end{parts}



\question%2
Let $\mathbb{A}$ denote the set of complex numbers that are algebraic over $\mathbb{Q}$, that is, they are roots of a not identically zero polynomial $f(x) \in \mathbb{Q}[x]$.
\begin{parts}
\part%2a
Let $\alpha$ be algebraic and suppose that $m(x)=x^{n}+a_{n-1} x^{n-1}+\ldots+a_{1} x+a_{0}$ is a polynomial of least degree in $\mathbb{Q}[x]$ that has $\alpha$ as a root. Show that the field $\mathbb{Q}(\alpha)$ generated by $\mathbb{Q}$ and $\alpha$ (i.e. the intersection of all subfields of $\mathbb{C}$ containing $\mathbb{Q}$ and $\alpha$) has a $\mathbb{Q}$ - basis \[
	\{1, \alpha, \alpha^{2}, \ldots \alpha^{n-1}\}
\] and hence is a degree $n$ extension of $\mathbb{Q}$.

\part%2b
Show that if $E$ is a finite degree extension of $\mathbb{Q}$ then every element of $E$ is algebraic over $\mathbb{Q}$.

\part%2c
Deduce using the tower law that $\mathbb{A}$ is a subfield of $\mathbb{C}$.

\part%2d
Show that $\mathbb{A}$ has infinite degree as a field extension of $\mathbb{Q}$.
\end{parts}



\question%3
Let $R$ be a commutative ring with 1, and let $GL_{n}(R)$ denote the set of $n \times n$ matrices with entries in $R$ which are invertible over $R$. Show that $GL_{n}(R)$ is the group of $n \times n$ matrices $A$ over $R$ with $\det A$ a unit in $R$.



\question%4
\begin{parts}
\part%4a
Show that the set $\{6,10,15\}$ generates $\mathbb{Z}$ as a $\mathbb{Z}$-module, but no proper subset of it generates $\mathbb{Z}$.

\part%4b
For which values of $a$ in the Gaussian integers $\mathbb{Z}[i]$ do \[
	\{(2,1),(2+i, a)\}
\] form a basis for $\mathbb{Z}[i] \oplus \mathbb{Z}[i]$?
\end{parts}



\question%5
Let $M$ be an $R$-module ($R$ an integral domain) and let $M^{\mathrm{tor}}$ denote the set of torsion elements of $M$. Show that $M^{\mathrm{tor}}$ is a submodule of $M$ and that $M / M^{\mathrm{tor}}$ is torsion-free.



\question%6
Consider the quotient $\mathbb{Z}$-module \[
	M=\frac{\mathbb{Z}^{3}}{\langle(3,3,1),(2,2,2)\rangle}.
\]
\begin{parts}
\part%6a
Is $M$ free?

\part%6b
Show that $\langle(3,3,1),(2,2,2)\rangle=\langle(1,1,-1),(0,0,4)\rangle$.

\part%6c
Find an element of infinite order in $M$.

\part%6d
Find a nonzero element of finite order in $M$.

\part%6e
Show that $M \cong \mathbb{Z} \oplus \mathbb{Z}_{4}$.
\end{parts}



\question%7
Recall that a (commutative) ring is \emph{Noetherian} if whenever we have an ascending chain of ideals \[
	I_{1} \subseteq I_{2} \subseteq I_{3} \subseteq \ldots
\] there exists an $N \in \mathbb{N}$ such that $I_{N}=I_{N+1}=\cdots$.
\begin{parts}
\part%7a
We saw that a PID is Noetherian. Generalise this by proving that if every ideal in the ring is finitely generated then the ring is Noetherian. Does the converse hold?

\part%7b
What would an appropriate definition of Noetherian be for modules?
\end{parts}

\end{questions}

\end{document}
