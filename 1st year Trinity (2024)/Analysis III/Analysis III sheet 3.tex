\documentclass[answers]{exam}
\usepackage{../TT2024}

\title{Analysis III -- Sheet 3}
\author{YOUR NAME HERE :)}
\date{Trinity Term 2024}
% accurate as of 25/06/2024


\begin{document}
\maketitle
\begin{questions}

\question%1
Evaluate $\int_{2}^{5} \frac{\mathrm d x}{\sqrt{x-1}}$, explaining carefully which results from the course you are using.



\question%2
Recall that $\cos : \mathbb{R} \rightarrow \mathbb{R}$ and $\sin : \mathbb{R} \rightarrow \mathbb{R}$ are defined by \[
	\cos (x)=\sum_{n=0}^{\infty}(-1)^{n} \frac{x^{2 n}}{(2 n) !}
	\qquad\text{and}\qquad
	\sin (x)=\sum_{n=0}^{\infty}(-1)^{n} \frac{x^{2 n+1}}{(2 n+1) !}.
\] Recall that there is a unique $p \in[0,2]$ such that $\cos p=0$ and $\sin p=1$. One can then define $\pi$ to be $2 p$. Show that \[
	\int_{0}^{1} \frac{\mathrm d x}{1+x^{2}}=\frac\pi4
\] (\emph{This question relies on Problem Sheet 6 Exercise 3 of the 2022 Analysis II course. That problem sheet is now available on the course website. You may use anything from that exercise in your answer.})



\question%3
For the following functions $f_{n}$, in which cases does the sequence $\left(f_{n}\right)$ converge uniformly on $[0,1]$? In which cases is it true that $\lim _{n \rightarrow \infty} \int_{0}^{1} f_{n}=\int_{0}^{1} \lim _{n \rightarrow \infty} f_{n}$?
\begin{subparts}
\subpart $n x^{n}(x-1)$,
\subpart $\displaystyle\frac{x}{1+n x^{2}}$,
\subpart $n^{2} x e^{-n x^{2}}$.
\end{subparts}



\question%4
Let $f_{n}(t)=\frac{n}{n+t}$ for $t \geqslant 0$. Show that, for each $x>0$, the sequence $\left(f_{n}\right)_{n=1}^{\infty}$ converges uniformly on $[0, x]$. By considering $\int_{0}^{x} f_{n}$ deduce that $\left(1+\frac{x}{n}\right)^{n} \rightarrow e^{x}$ as $n \rightarrow \infty$ for all $x \in \mathbb{R}$. (\emph{You may use simple facts about $\log$ and $\exp$ without proof.})

\end{questions}

\end{document}