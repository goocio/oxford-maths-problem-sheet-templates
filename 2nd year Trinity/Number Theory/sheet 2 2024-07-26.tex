\documentclass[answers]{exam}
\usepackage{../preamble}

\title{Number Theory -- Sheet 2}
\author{YOUR NAME HERE :)}
\date{Trinity Term 2025}


\begin{document}
\maketitle
\begin{questions}

\question%1
Suppose that $p$ and $q=2 p+1$ are both odd primes. Explain why (a) $2 p$ is a quadratic non-residue of $q$ and (b) $q$ has $p-1$ primitive roots. Show that the primitive roots of $q$ are precisely the quadratic non-residues of $q$, other than $2 p$.



\question%2
Prove that if $n$ has a primitive root then it has $\phi(\phi(n))$ of them.



\question%3
Let $p$ be an odd prime. Show that every element in $\mathbb{Z} / p \mathbb{Z}$ can be written as the sum of two squares.



\question%4
Do there exist integer solutions to the equation $x^2 \equiv 251 \bmod 779$? [\emph{Note that $779=19 \times 41$.}]



\question%5
Does the equation $x^2+10 x+15 \equiv 0 \bmod 45083$ have any integer solutions? [\emph{Note that $45083$ is prime.}]


\question%6
Use the Fermat method to factorise 9579, without using a calculator.



\question%7
For any integer $n \geqslant 2$, let $F_{n}=2^{2^n}+1$ be the $n$th "Fermat number". Suppose that $p$ is a prime factor of $F_n$.
\begin{parts}
\part%7a
Show that $\operatorname{ord}_p(2)=2^{n+1}$.

\part%7b
Show that \[
	2^{(p-1) / 2} \equiv 1 \bmod p
\]

\part%7c
Deduce that $p=1+2^{n+2} k$ for some $k \in \mathbb{N}$.

\part%7d
Hence, or otherwise, verify that $F_4=65537$ is prime.
\end{parts}



\question%8
\begin{parts}
\part%8a
Using the Fermat method, factorise 2881, and hence find $\phi(2881)$.

\part%8b
A message has been encrypted using RSA and the encoding $01 \leftrightarrow A$, $02 \leftrightarrow B$, $03 \leftrightarrow C$, etc. with exponent $e=5$ and modulus $n=2881$. The message is 235221380828. What is the plain-text message? I suggest using a free online modular exponentiation calculator, which you can find by a google search for those terms.
\end{parts}



\question%9
Let $p \geqslant 7$ be a prime. Show that every nonzero element of $\mathbb{Z} / p \mathbb{Z}$ is a sum of two non-zero squares.

\end{questions}

\end{document}
