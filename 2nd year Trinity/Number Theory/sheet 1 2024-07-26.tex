\documentclass[answers]{exam}
\usepackage{../preamble}

\title{Number Theory -- Sheet 1}
\author{YOUR NAME HERE :)}
\date{Trinity Term 2025}


\begin{document}
\maketitle
\begin{questions}

\question%1
Let $a$ be a positive integer and suppose that in its decimal expansion it has 7 digits: $a=a_0+10 a_1+\cdots+1000000 a_6$. Show that $a$ is divisible by 7 if and only if $a_0+3 a_1+2 a_2-a_3-3 a_4-2 a_5+a_6$ is divisible by 7.



\question%2
Find a positive integer $x$ such that $x \equiv 3(\bmod 4), 2 x \equiv 5(\bmod 9)$ and $7 x \equiv 1(\bmod 11)$.



\question%3
Find the smallest positive integer $x$ such that $x \equiv 11(\bmod 59)$ and $x \equiv$ $29(\bmod 71)$.



\question%4
Show that $2^{340} \equiv 1(\bmod 341)$. Comment on this in connection with Fermat's Little Theorem.



\question%5
Let $n=(6 t+1)(12 t+1)(18 t+1)$ with $6 t+1,12 t+1$ and $18 t+1$ all prime numbers. Prove that \[
	a^{n-1} \equiv 1(\bmod n)
\] whenever $(a, n)=1$. Comment on this in connection with Fermat's Little Theorem.


\question%6
Show that if $x$ is an integer then $x^{10} \in\{-1,0,1\}(\bmod 25)$.



\question%7
For which $N$ is the following true: if you take an $N$ digit number, reverse its digits and then add the result to the original number, you always get a multiple of 11?



\question%8
Find all primes $p$ for which the $\operatorname{map} \phi: \mathbb{Z} / p \mathbb{Z} \to \mathbb{Z} / p \mathbb{Z}$ defined by $\phi(x)=$ $x^{13}$ is a group homomorphism.



\question%9
Find all four-digit numbers $N$ such that, when written in decimal, the last four digits of any power of $N$ are the same as the digits of $N$.



\question%10
For each of the following properties, show that there are infinitely many positive integers $n$ which do not have that property.
\begin{parts}
\part%10a
$n$ is the sum of at most 3 squares;

\part%10b
$n$ is the sum of at most 8 sixth powers;

\part%10c
$n$ is the sum of at most 11 tenth powers;

\part%10d
$n$ is the sum of at most 15 fourth powers;

\part%10e
$n$ is the sum of at most 7 (positive) seventh powers.
\end{parts}

\end{questions}

\end{document}
