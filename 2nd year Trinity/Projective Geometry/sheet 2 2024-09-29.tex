\documentclass[answers]{exam}
\usepackage{../preamble}

\title{Projective Geometry -- Sheet 2}
\author{YOUR NAME HERE :)}
\date{Trinity Term 2025}


\begin{document}
\maketitle
\begin{questions}

\question%1
Write down the dual of Pappus' Theorem.



\question%2
Let $P_0, P_1, P_2, P_3$ be four distinct points in a projective plane $\mathbb{P}(V)$. Show that $P_0, P_1, P_2, P_3$ are in general position if and only if the lines $P_0 P_1, P_1 P_2, P_2 P_3, P_3 P_0$ are in general position in $\mathbb{P}(V^{*})$.



\question%3
Use general position arguments to show that given five points in the projective plane, such that no three are collinear, there is a unique conic through these five points.



\question%4
Let $C, D$ be conics in a projective plane $\mathbb{P}(V)$, where $V$ is a 3 -dimensional real vector space, and suppose that $C \cap D=\{p_1, p_2, p_3, p_4\}$, where $p_1, \ldots, p_4$ are distinct points in $\mathbb{P}(V)$.
\begin{parts}
\part%4a
Show that $p_1, \ldots, p_4$ are in general position. Prove that there exist homogeneous coordinates $[x_0: x_1: x_2]$ on $\mathbb{P}(V)$ for which \[
	p_1=[1: 1: 1], \quad p_2=[1:-1: 1], \quad p_3=[1: 1:-1], \quad p_4=[1:-1:-1] .
\]

\part%4b
Show that any conic through $p_1, \ldots, p_4$ has equation \[
	\lambda x_0^2+\mu x_1^2+\nu x_2^2=0
\] where $\lambda+\mu+\nu=0$.

\part%4c
Find four projective transformations $\tau$ of $\mathbb{P}(V)$ that form a group, and for which $\tau(C)=C$ and $\tau(D)=D$.
\end{parts}



\question%5
Let $F\left(x_0, x_1, x_2\right)$ be a homogeneous polynomial of degree $n$. Let $\mathcal{C}$ be the set of points $\mathbf{a}=[a_0, a_1, a_2]$ in $\mathbb{R P}^2$ such that $F(a_0, a_1, a_2)=0$. Provided that $\nabla F(\mathbf{a}) \neq \mathbf0$, the tangent line to $\mathcal{C}$ at $\mathbf{a}=[a_0, a_1, a_2]$ is the line \[
	x_0 \frac{\partial F}{\partial x_0}(\mathbf{a})+x_1 \frac{\partial F}{\partial x_1}(\mathbf{a})+x_2 \frac{\partial F}{\partial x_2}(\mathbf{a})=0
\] in $\mathbb{R P}^2$ and $\mathbf{a}$ is said to be singular if $\nabla F(\mathbf{a})=\mathbf0$.
\begin{parts}
\part%5a
Show that a lies on the tangent line to a.

\part%5b
Given a $3 \times 3$ symmetric real matrix $B$ its associated conic is the set of solutions to the equation $\mathbf{x}^{T} B \mathbf{x}=0$ where $\mathbf{x}=[x_0: x_1: x_2]$ and the conic is said to be singular if $B$ is singular. Show that a conic is singular if and only if it has a singular point.

\part%5c
Sketch the curves $y^2=x^3$ and $y^2=x^2(x+1)$ in $\mathbb{R}^2$. What singular points do these curves have? Show that $y=x^3$ has a singular point at infinity.
\end{parts}



\question%6
Find all rational numbers $x, y$ such that $x^2+y^2-x y=1$.



\question%7
Let $V$ be a 3-dimensional real vector space and suppose that $L_0, L_1, L_2, L_3$ are four lines in the projective plane $\mathbb{P}(V)$ all intersecting in a common point $x$. Explain why:
\begin{parts}
\part%7a
if $L$ is a line in $\mathbb{P}(V)$ that does not pass though $x$, but intersects $L_{i}$ in a point $x_{i}$ (so $x_0, x_1, x_2, x_3$ are four distinct collinear points), then the cross-ratio $\left(x_0 x_1: x_2 x_3\right)$ is independent of the choice of $L$;

\part%7b
the cross-ratio defined in (a) equals the cross-ratio $(L_0 L_1: L_2 L_3)$ formed by regarding $L_0, L_1, L_2, L_3$ as collinear points of the dual projective plane $\mathbb{P}(V^{*})$.
\end{parts}

\end{questions}

\end{document}
