\documentclass[answers]{exam}
\usepackage{../preamble}

\title{Projective Geometry -- Sheet 1}
\author{YOUR NAME HERE :)}
\date{Trinity Term 2025}


\begin{document}
\maketitle
\begin{questions}

\question%1
\begin{parts}
\part%1a
If we identify $(x, y)$ with $[1: x: y]$, what is the point at infinity shared by all lines of the form $y=m x+c$, where $m$ is fixed?

\part%1b
Under the same identification, find any points at infinity of the conic $4 x^2+y^2-8 x-6 y-4 x y+9=0$. Classify the conic.

\part%1c
Show that those projective transformations in $P G L(3, \mathbb{F})$ which map the line at infinity to itself form a subgroup of $\operatorname{PGL}(3, \mathbb{F})$ which is isomorphic to \[
	A G L(2, \mathbb{F})=\left\{\mathbf{x} \mapsto A \mathbf{x}+\mathbf{b} \mid A \in G L(2, \mathbb{F}), \mathbf{b} \in \mathbb{F}_{\mathrm{col}}^2\right\}
\] Which of these mappings fix the line at infinity pointwise?
\end{parts}



\question%2
\begin{parts}
\part%2a
Let $\mathbb{P}\left(U_1\right)$ and $\mathbb{P}\left(U_2\right)$ be two non-intersecting lines in the 3-dimensional projective space $\mathbb{R} \mathbb{P}^3=\mathbb{P}(\mathbb{R}^4)$. Show that $\mathbb{R}^4=U_1 \oplus U_2$.

\part%2b
Deduce that three pairwise non-intersecting lines in $\mathbb{R} \mathbb{P}^3$ have infinitely many transversals, i.e. projective lines meeting all three.
\end{parts}



\question%3
Suppose that three lines $L, M, N$ in $\mathbb{P}^{n}$ intersect in pairs. Prove that the three lines are either concurrent (have a common point) or coplanar.



\question%4
\begin{parts}
\part%4a
List the elements of $\operatorname{PGL}\left(2, \mathbb{F}_2\right)$. What is the order of $P G L(2, \mathbb{F})$ if $|\mathbb{F}|=q$?

\part%4b
By considering the action of $P G L\left(2, \mathbb{F}_2\right)$ on $\mathbb{F}_2 \mathbb{P}^1$, show that $P G L\left(2, \mathbb{F}_2\right) \cong S_3$. Is $P G L\left(2, \mathbb{F}_3\right) \cong S_4$? Is $P G L\left(2, \mathbb{F}_5\right) \cong S_6$?
\end{parts}



\question%5
Let $a, b, c, d$ be four distinct points in $\mathbb{C}$. Show that $a, b, c, d$ lie on a circline if and only if the cross-ratio $(a b: c d)$ is real.



\question%6
We say $x_0, x_1$ and $x_2, x_3$ are harmonically separated if $(x_0 x_1: x_2 x_3)=-1$, where the $x_i$ are distinct points in a projective line. Let $a, b, c, d$ be four points in general position in the real projective plane $\mathbb{R} \mathbb{P}^2$ and let $e, f, g$ be the diagonal points, i.e. $e=a c \cap b d, f=a b \cap c d, g=a d \cap b c$. Let $g e$ meet $a b$ at $h$. Prove that $a, b$ and $h, f$ are harmonically separated.



\question%7
\begin{parts}
\part%7a
Let $\tau \in P G L(2, \mathbb{C})$, other than the identity. Show that $\tau$ fixes either one or two points. Show that this need not be true over other fields.

\part%7b
If $\tau$ fixes two points, show that there is an inhomogeneous co-ordinate $z$ on $\mathbb{C P}^1$ with respect to which \[
	\tau(z)=\lambda z, \quad \lambda \neq 0,1
\] Is the same true over other fields?

\part%7c
Let $A_1, A_2, A_3$ be three distinct points in $\mathbb{C P}^1$ and let $n \geqslant 3$ be an integer. Show that there is $\tau \in P G L(2, \mathbb{C})$ such that $\tau(A_1)=A_2, \tau\left(A_2\right)=A_3$ and $\tau$ has order $n$.
\end{parts}


\question%8
Use the strategy outlined in the lectures to prove Pappus's Theorem: \emph{Let $A, B, C$ and $A', B', C'$ be two collinear triples of distinct points in the projective plane. Then the three intersection points $A B' \cap A' B$, $B C' \cap B' C$, and $C A' \cap C' A$ are collinear.} Proceed by the following steps.
\begin{parts}
\part%8a
Prove the theorem in the degenerate case when $A, B, C', B'$ are not in general position.

\part%8b
If these 4 points are in general position, explain why without loss of generality we may take them to be \[
	A=[1,0,0], \quad B=[0,1,0], \quad C'=[0,0,1], \quad B'=[1,1,1] .
\]
\end{parts}



\question%9
(Optional) Every line in the real affine plane $\mathbb{R}^2$ can be written in the form $a x+b y+c=0$ where $a, b$ are not both zero. Of course, $\lambda a x+\lambda b y+\lambda c=0$ is an equation of the same line where $\lambda \neq 0$. Hence the space of lines can be identified with \[
	M=\frac{\mathbb{R}^2 \setminus\{(0,0)\} \times \mathbb{R}}{\mathbb{R}^*}
\] Identify $M$ as a subspace of $\mathbb{R}^2$. What is the topology of $M$?

\end{questions}

\end{document}
